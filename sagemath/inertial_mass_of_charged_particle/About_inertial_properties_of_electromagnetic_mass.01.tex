\documentclass{article}
\usepackage[T2A]{fontenc}
\usepackage[utf8]{inputenc}
\usepackage[english, russian]{babel}

\begin{document}

\title{Об инерционных свойствах электромагнитной массы}

\author{А.Ю.Дроздов}

\date{Jan 30, 2020}


%========================================================================

\section{Постановка задачи}



Пусть частица с заданным распределением объёмной плотности электрического заряда $\rho \left( r \right)$ приобретает ускорение $\overrightarrow{a}$ . Найти силу, действующую на распределённый в объёме электрический заряд этой частицы со стороны электрического поля самоиндукции.

Для решения этой задачи электрическое поле самоиндукции $\overrightarrow{E}=-\frac{1}{c}\frac{\partial \overrightarrow{A}}{\partial t}$
выразим исходя из выражения векторного потенциала Лиенара-Вихерта \cite{LL2} $\overrightarrow{A}=\frac{\overrightarrow{v}}{c}\frac{q}{R-\frac{\overrightarrow{v}\cdot \overrightarrow{R}}{c}}=\frac{\overrightarrow{v}}{c}\frac{q}{{{R}^{*}}}$ дифференцирование которого приводит к выражению \cite{rustot}

\[\overrightarrow{E}=\frac{dq}{{{R}^{*}}^{2}}\left\{ \frac{\overrightarrow{v}}{c}\left( \frac{R}{{{R}^{*}}}\left( \frac{{{v}^{2}}}{{{c}^{2}}}-\frac{\overrightarrow{a}\cdot \overrightarrow{R}}{{{c}^{2}}}-1 \right)+1 \right)-\frac{\overrightarrow{a}R}{{{c}^{2}}} \right\}\]

В системе СИ перед этим выражением появляется множитель $\frac{1}{4\pi {{\varepsilon }_{0}}}=\frac{{{\mu }_{0}}{{c}^{2}}}{4\pi }=\frac{{{c}^{2}}}{{{10}^{7}}}$

Направляя в сферической системе координат вектор ускорения вдоль оси $z$  запишем

${{R}^{*}}=R-\frac{v}{c}\left( {{z}_{a}}-{{z}_{q}} \right)=R-\frac{v}{c}\left( {{r}_{a}}\cos \left( {{\theta }_{a}} \right)-{{r}_{q}}\cos \left( {{\theta }_{q}} \right) \right)$

\[\overrightarrow{a}\cdot \overrightarrow{R}=a\left( {{z}_{a}}-{{z}_{q}} \right)=a\left( {{r}_{a}}\cos \left( {{\theta }_{a}} \right)-{{r}_{q}}\cos \left( {{\theta }_{q}} \right) \right)\]

Где следуя Тамму \cite{tamm}, индексом $q$ обозначены координаты заряда, а индексом $a$ обозначены координаты точки наблюдения

\[\overrightarrow{E}=\int\limits_{{{V}_{q}}}\\
{\left\{ \frac{\overrightarrow{v}}{c}\left( \frac{R}{{{R}^{*}}}\left( \frac{{{v}^{2}}}{{{c}^{2}}}-\frac{\overrightarrow{a}\cdot \overrightarrow{R}}{{{c}^{2}}}-1 \right)+1 \right)-\frac{\overrightarrow{a}R}{{{c}^{2}}} \right\}\frac{\rho \left( {{r}_{q}} \right)}{{{R}^{*}}^{2}}\ }d{{V }_{q}}\]


Действующая на заряд со стороны электрического поля самоиндукции инерционная сила равна

$\overrightarrow{{{F}_{z}}}=\int\limits_{{{r}_{a}}}\int\limits_{{{\varphi}_{a}}}\int\limits_{{{\theta}_{a}}}{\overrightarrow{{{E}_{z}}}\rho \left( {{r}_{a}} \right){{r}_{a}}^{2}\sin \left( {{\theta }_{a}} \right)}\ d{{\theta }_{a}}d{{\varphi }_{a}}d{{r}_{a}}$


Из приведенных формул видно, что сила инерции электромагнитной массы зависит от вида функции распределения плотности заряда в пространстве, а также от скорости и ускорения заряда.

\section{Приближение малых скоростей без учёта запаздывания}

В приближении малых скоростей ${}^{v}/{}_{c}\ll 1$  и малых ускорений $a{{r}_{0}}\ll {{c}^{2}}$ и при игнорировании запаздывания
\[\overrightarrow{E}=\int\limits_{{{r}_{q}}}\int\limits_{{{\varphi}_{q}}}\int\limits_{{{\theta}_{q}}}\\
{\left\{ -\frac{\overrightarrow{a}R_{0}}{{{c}^{2}}} \right\}\frac{\rho \left( {{r}_{q}} \right){{r}_{q}}^{2}\sin \left( {{\theta }_{q}} \right)}{{{R}_{0}}^{2}}\ }d{{\theta }_{q}}d{{\varphi }_{q}}d{{r}_{q}}\]
где ${R}_{0}$ расстояние от точки источника заряда к точке наблюдения без учёта запаздывания.

Откуда
${{F}_{z}}=-\frac{\overrightarrow{a}}{{{c}^{^{2}}}}\int\limits_{{{V}_{a}}}{\int\limits_{{{V}_{q}}}{\frac{\rho \left( {{r}_{q}} \right)\rho \left( {{r}_{a}} \right)}{R_{0}}}}\ d{{V}_{q}}d{{V}_{a}}$
Сопоставляя с законами Ньютона для электромагнитной массы получаем выражение
$m=\frac{1}{{{c}^{^{2}}}}\int\limits_{{{V}_{a}}}{\int\limits_{{{V}_{q}}}{\frac{\rho \left( {{r}_{q}} \right)\rho \left( {{r}_{a}} \right)}{R_{0}}}}\ d{{V}_{q}}d{{V}_{a}}$ (в системе сгс) и соответственно $m=\frac{{{\mu }_{0}}}{4\pi }\int\limits_{{{V}_{a}}}{\int\limits_{{{V}_{q}}}{\frac{\rho \left( {{r}_{q}} \right)\rho \left( {{r}_{a}} \right)}{R_{0}}}}\ d{{V}_{q}}d{{V}_{a}}$ в системе СИ
Рассчитаем теперь электромагнитную массу равномерно заряженной сферы радиуса ${{r}_{0}}$

Поскольку расстояние между координатами заряда и точки наблюдения ${{R}_{0}}=\left|\overrightarrow{r_{q}} - \overrightarrow{r_{a}}\right|$ находится в знаменателе, то в сферической системе координат можно применить разложение по сферическим гармоникам следующего вида  \cite{flugge} если $\left( {{r}_{q}}<{{r}_{a}} \right)$ то

$\frac{1}{\left| \overrightarrow{{{r}_{q}}}-\overrightarrow{{{r}_{a}}} \right|}=\frac{1}{{{r}_{a}}}\sum\limits_{l=0}^{\infty }{{{\left( \frac{{{r}_{q}}}{{{r}_{a}}} \right)}^{l}}{{P}_{l}} \cos \left( \gamma  \right)}$

и если $\left( {{r}_{a}}<{{r}_{q}} \right)$ то

$\frac{1}{\left| \overrightarrow{{{r}_{q}}}-\overrightarrow{{{r}_{a}}} \right|}=\frac{1}{{{r}_{q}}}\sum\limits_{l=0}^{\infty }{{{\left( \frac{{{r}_{a}}}{{{r}_{q}}} \right)}^{l}}{{P}_{l}} \cos \left( \gamma  \right)}$

В данной формуле ${{P}_{l}} \cos \left( \gamma  \right)$ это полиномы Лежандра аргумент которых $\gamma$ есть угол между векторами ${{r}_{q}}$  и ${{r}_{a}}$. Применяя формулу, известную как теорему сложения

${{P}_{l}}\cos \left( \gamma  \right)=\frac{4\pi }{2l+1}\sum\limits_{m=-l}^{l}{Y_{l,m}^{*}\left( {{\theta }_{a}},{{\varphi }_{a}} \right)}\ {{Y}_{l,m}}\left( {{\theta }_{q}},{{\varphi }_{q}} \right)$

получаем способ аналитического вычисления интеграла инертной электромагнитной массы.

Пусть заряд представляет собой сферу, равномерно заряженную по всему объёму тогда плотность заряда составит $\rho \left( r \right)=\frac{e}{{}^{4}/{}_{3}\pi {{r}_{0}}^{3}}$

Производя вычисления для интеграла электромагнитной массы получено значение 

$m =\frac{1}{{{c}^{^{2}}}}\frac{6}{5}\frac{e^2}{{{r}_{0}}}$ (сгс) и
$m =\frac{{{\mu }_{0}}}{4\pi }\frac{6}{5}\frac{e^2}{{{r}_{0}}}$ (СИ)

Пусть заряд представляет собой сферическую поверхность с равномерным поверхностным распределением заряда по поверхности сферы, равным $\sigma=\frac{e}{4\pi {{r}_{0}}^{2}})$ тогда для вычисления инертной электромагнитной массы потребуется формула
$m=\frac{1}{{{c}^{^{2}}}}\int\limits_{{{S}_{a}}}{\int\limits_{{{S}_{q}}}{\frac{\sigma \left( {{r}_{q}} \right)\sigma \left( {{r}_{a}} \right)}{R}}}\ d{{S}_{q}}d{{S}_{a}}$

Вычисления по которой дают следующий результат

$m =\frac{1}{{{c}^{^{2}}}}\frac{e^2}{{{r}_{0}}}$ (сгс) и
$m =\frac{{{\mu }_{0}}}{4\pi }\frac{e^2}{{{r}_{0}}}$ (СИ)

В работе \cite{misyuchenko} приводится способ вычисления инертной электромагнитной массы электрона исходя из коэффициента самоиндукции сферы и производной тока сферы по времени. Результат вычислений электромагнитной массы электрона авторы приводят следующий $m =\frac{{{\mu }_{0}}}{8\pi }\frac{e^2}{{{r}_{0}}}$ (СИ) то есть в два раза меньший, чем полученный в данной работе.

Анализ показывает, что приведенная авторами формула коэффициента самоиндукции сферы $L =\frac{{{\mu }_{0}}{r}_{0}}{2\pi }$ (СИ) в 2 раза занижена. Автором данной работы было произведено вычисление коэффициента самоиндукции сферической поверхности с равномерно распределённым поверхностным зарядом. Получена формула $L =\frac{{{\mu }_{0}}{r}_{0}}{\pi }$ (СИ) .

Далее авторы  работы \cite{misyuchenko} пишут, "Отметим тот важный факт, что выведенная из закона самоиндукции масса полностью совпадает с Эйнштейновской массой $m=\frac{U}{c^2}$ , если под полной энергией электрона U понимать собственную энергию его электрического поля."

Однако с учётом исправленного значения коэффициента самоиндукции сферы данное утверждение становится неверным. 

Русская википедия в статье "Классический радиус электрона"  на момент написания данной работы даёт следующее определение:

Классический радиус электрона равен радиусу полой сферы, на которой равномерно распределён заряд, если этот заряд равен заряду электрона, а потенциальная энергия электростатического поля ${U}_{0}$  полностью эквивалентна половине массы электрона (без учета квантовых эффектов):

${\displaystyle U_{0}={\frac {1}{2}}{\frac {1}{4\pi \varepsilon _{0}}}\cdot {\frac {e^{2}}{r_{0}}}={\frac {1}{2}}m_{0}c^{2}}$.

Возникает закономерный вопрос: а чему соответствует вторая половина массы электрона?

Тамм \cite{tamm} для собственной электрической энергии заряженного шара радиуса $a$ находит $W=\frac{e^2}{2a}$ если заряд распределён на поверхности шара и $W=\frac{3e^2}{5a}$ если заряд распределён по всему объёму шара. Появление коэффициента $\frac{1}{2}$  в формуле энергии системы зарядов Тамм объясняет тем, что "в сумму энергия каждой пары зарядов входит дважды, так, например, в ней встретится как член ${e_{1}}{e_{1}}/{R_{12}}$ так и равный ему член ${e_{2}}{e_{1}}/{R_{21}}$".

Однако в задаче, рассмотренной в самом начале данной работы при нахождении силы, действующей на распределённый в объёме электрический заряд этой частицы со стороны электрического поля самоиндукции, подобные рассуждения неприменимы. Поскольку хотя и каждая пара зарядов $d{e}_{1}$ и $d{e}_{2}$ при совместном поступательном движении создаёт две одинаковые как по виду формулы, так и по значинию силы самоиндукции, они обе должны быть включены в общую электромагнитную инерцию, поскольку одна из них - это сила самоиндукции действующая на заряд $d{e}_{1}$  со стороны поля заряда $d{e}_{2}$ , а вторая это сила самоиндукции действующая на заряд $d{e}_{2}$  со стороны поля заряда $d{e}_{1}$.

Таким образом потенциальная энергия электростатического поля модели электрона в виде полой сферы 
${U}_{0} =\frac{1}{2}\frac{e^2}{{{r}_{0}}}$ (сгс)
при том что инертная масса электрона этой модели равна
$m =\frac{1}{{{c}^{^{2}}}}\frac{e^2}{{{r}_{0}}}$ (сгс).

В любой другой модели потенциальная энергия электростатического поля заряженной частицы равна
${U}_{0}=\frac{1}{2}\int\limits_{{{V}_{a}}}{\int\limits_{{{V}_{q}}}{\frac{\rho \left( {{r}_{q}} \right)\rho \left( {{r}_{a}} \right)}{R_{0}}}}\ d{{V}_{q}}d{{V}_{a}}$ (сгс)
тогда как инертная масса 
$m=\frac{1}{{{c}^{^{2}}}}\int\limits_{{{V}_{a}}}{\int\limits_{{{V}_{q}}}{\frac{\rho \left( {{r}_{q}} \right)\rho \left( {{r}_{a}} \right)}{R_{0}}}}\ d{{V}_{q}}d{{V}_{a}}$ (сгс).

Таким образом соотношение потенциальной энергией электростатического поля и энергией массы покоя электрона, приведенное в русской википедии подтверждается, но вопрос чему соответствует вторая половина энергии массы покоя электрона остаётся открытым.


\section{Учёт запаздывания}

Чтобы учесть запаздывание следует решить систему уравнений

$s=v\left( t-t' \right)+\frac{a}{2}{{\left( t-t' \right)}^{2}}$
и
$R=c\left( t-t' \right)$

Учитывая, что по теореме косинусов

${{R}^{2}}={{R}_{0}}^{2}+{{s}^{2}}-2{{R}_{0}}s\cos \left( \alpha  \right)={{R}_{0}}^{2}+{{s}^{2}}-2{{R}_{0}}s\frac{{{z}_{q'}}-{{z}_{a'}}}{{{R}_{0}}}$

уравнение для вычисления запаздывающего момента принимает вид

$c^{2}\left( t-t' \right)^{2}={{R}_{0}}^{2}+{{s}^{2}}+2s\left( {{z}_{a'}}-{{z}_{q'}} \right)$

\section{Приближение малых скоростей с учётом запаздывания}
Решение этой системы имеет весьма сложный вид, но если мы исследуем вопрос какова будет инертная масса покоя, то при решении этой системы мы можем положить $v = 0$. В этом случае для нахождения запаздывания нужно будет решить уравнение

$-\frac{1}{4} \, a^{2} \mathit{dt}^{4} + c^{2} \mathit{dt}^{2} - a \mathit{dt}^{2} {\left(z_{a} - z_{q}\right)} - R_{0}^{2} = 0$

где
$(t-t') = dt$

Это уравнение имеет 4 решения, но физически приемлемый смысл при положительном ускорении имеет решение

% $\mathit{dt} = -\frac{\sqrt{2 \, c^{2} - 2 \, a \mathit{dz} + 2 \, \sqrt{-R_{0}^{2} a^{2} + c^{4} - 2 \, a c^{2} \mathit{dz} + a^{2} \mathit{dz}^{2}}}}{a}$

% $\mathit{dt} = \frac{\sqrt{2 \, c^{2} - 2 \, a \mathit{dz} + 2 \, \sqrt{-R_{0}^{2} a^{2} + c^{4} - 2 \, a c^{2} \mathit{dz} + a^{2} \mathit{dz}^{2}}}}{a}$

% $\mathit{dt} = -\frac{\sqrt{2 \, c^{2} - 2 \, a \mathit{dz} - 2 \, \sqrt{-R_{0}^{2} a^{2} + c^{4} - 2 \, a c^{2} \mathit{dz} + a^{2} \mathit{dz}^{2}}}}{a}$

$\mathit{dt} = \frac{\sqrt{2 \, c^{2} - 2 \, a \mathit{dz} - 2 \, \sqrt{-R_{0}^{2} a^{2} + c^{4} - 2 \, a c^{2} \mathit{dz} + a^{2} \mathit{dz}^{2}}}}{a}$

где
$\mathit{dz} = z_{a'} - z_{q'}$


В приближении малых скоростей ${}^{v}/{}_{c}\ll 1$ но при учете запаздывания
\[\overrightarrow{E}=\int\limits_{{{r}_{q}}}\int\limits_{{{\varphi}_{q}}}\int\limits_{{{\theta}_{q}}}\\
{\left\{ -\frac{\overrightarrow{a}R}{{{c}^{2}}} \right\}\frac{\rho \left( {{r}_{q}} \right){{r}_{q}}^{2}\sin \left( {{\theta }_{q}} \right)}{{{R}^{*}}^{2}}\ }d{{\theta }_{q}}d{{\varphi }_{q}}d{{r}_{q}}\]
 Откуда
\[{{F}_{z}}=-\frac{\overrightarrow{a}}{{{c}^{^{2}}}}\int\limits_{{{V}_{a}}}{\int\limits_{{{V}_{q}}}{\frac{\rho \left( {{r}_{q}} \right)\rho \left( {{r}_{a}} \right)}{R}}}\ d{{V}_{q}}d{{V}_{a}}\]
где
$\mathit{R} = c\frac{\sqrt{2 \, c^{2} - 2 \, a \mathit{dz} - 2 \, \sqrt{-R_{0}^{2} a^{2} + c^{4} - 2 \, a c^{2} \mathit{dz} + a^{2} \mathit{dz}^{2}}}}{a}$

\section{Результат численного интегрирования. Зависимость электромагнитной инертной массы от ускорения}

Пусть заряд представляет собой сферу, равномерно заряженную по всему объёму.
Для численного интегрирования была использована программа Cuba-4.2, интегратор VEGAS.
Значения радиуса сферы и ее заряда заданы равными единице $r_{0} = 1.0$ и $q = 1.0$.
Численное интегрирование этим итегратором для формулы не учитывающей запаздывание дает результат
$1.20057324 +- 0.00117095$ что вполне соответствует аналитическому решению
$m =\frac{1}{{{c}^{^{2}}}}\frac{6}{5}\frac{e^2}{{{r}_{0}}}$ (сгс) и
$m =\frac{{{\mu }_{0}}}{4\pi }\frac{6}{5}\frac{e^2}{{{r}_{0}}}$ (СИ)

При интегрировании с учётом запаздывания скорость света была установлена равной $c = 1.0$.

Результаты интегрирования при различных ускорениях сведены в таблицу.

\begin{table}[]
\begin{tabular}{lllll}
 & a      & VEGAS RESULT             &  &  \\
 & 0      & 1.20057324 +- 0.00117095 &  &  \\
 & 0.0001 & 1.20057145 +- 0.00117094 &  &  \\
 & 0.001  & 1.20057159 +- 0.00117101 &  &  \\
 & 0.01   & 1.20056853 +- 0.00117642 &  &  \\
 & 0.1    & 1.19887814 +- 0.00117946 &  &  \\
 & 0.2    & 1.19315830 +- 0.00116342 &  &  \\
 & 0.3    & 1.18205714 +- 0.00116337 &  &  \\
 & 0.4    & 1.15008831 +- 0.00112842 &  &  \\
 & 0.5    & 1.09737467 +- 0.00108701 &  &  \\
 & 0.6    & 1.03839069 +- 0.00101328 &  &  \\
 & 0.7    & 0.98022447 +- 0.00101643 &  &  \\
 & 0.8    & 0.92625001 +- 0.00104227 &  &  \\
 & 0.9    & 0.87671199 +- 0.00105610 &  &  \\
 & 1.0    & 0.83227193 +- 0.00107483 &  &  \\
 & 2.0    & 0.56319872 +- 0.00103720 &  &  \\
 & 5.0    & 0.32020001 +- 0.00083749 &  &  \\
 & 10.0   & 0.20964075 +- 0.00070045 &  &  \\
 & 100.0  & 0.05666584 +- 0.00038991 &  &  \\
 & 1000.0 & 0.00130377 +- 0.00006038 &  &  \\
 & 1250.0 & 0.00081384 +- 0.00003357 &  &  \\
 & 1500.0 & 0.00000082 +- 0.00000028 &  &  \\
 & 1750.0 & 0.00000017 +- 0.00000006 &  &  \\
 & 2000.0 & 0.00000010 +- 0.00000005 & 9.51624e-08 +- 4.99632e-08 &  \\
 & 2100.0 & 0.00000001 +- 0.00000000 & 9.04571e-09 +- 3.96347e-09 &  \\
\end{tabular}
\end{table}

Электромагнитная масса уменьшается с ускорением. При достижении определённого предела электромагнитная масса практически исчезает. Эта закономерность выявляется при учёте запаздывания. Физически это означает что при достижении колосальных ускорений частица ускользает от действия своего поля.

\section{Приближение малых ускорений с учётом запаздывания}

В приближении малых ускорений $a{{r}_{0}}\ll {{c}^{2}}$ при учете запаздывания


\[\overrightarrow{E}=\int\limits_{{{V}_{q}}}\\
{\left\{ \frac{\overrightarrow{v}}{c}\left( \frac{R}{{{R}^{*}}}\left( \frac{{{v}^{2}}}{{{c}^{2}}}-1 \right)+1 \right)-\frac{\overrightarrow{a}R}{{{c}^{2}}} \right\}\frac{\rho \left( {{r}_{q}} \right)}{{{R}^{*}}^{2}}\ }d{{V }_{q}}\]

Откуда

${{F}_{z}}=-\frac{\overrightarrow{a}}{{{c}^{^{2}}}}\int\limits_{{{V}_{a}}}{\int\limits_{{{V}_{q}}} \\
{\left\{ \frac{\overrightarrow{v}}{c}\left( \frac{R}{{{R}^{*}}}\left( \frac{{{v}^{2}}}{{{c}^{2}}}-1 \right)+1 \right)-\frac{\overrightarrow{a}R}{{{c}^{2}}} \right\} } \\
{\frac{\rho \left( {{r}_{q}} \right)\rho \left( {{r}_{a}} \right)}{{R^{*}}^{2}}}}\ d{{V}_{q}}d{{V}_{a}}$

Запаздывающий момент рассчитывается с помощью выражения

$\mathit{dt} = \frac{\mathit{dz} v - \sqrt{R_{0}^{2} c^{2} - {\left(R_{0}^{2} - \mathit{dz}^{2}\right)} v^{2}}}{c^{2} - v^{2}}$
% $\mathit{dt} = \frac{\mathit{dz} v + \sqrt{R_{0}^{2} c^{2} - {\left(R_{0}^{2} - \mathit{dz}^{2}\right)} v^{2}}}{c^{2} - v^{2}}$

Следовательно

$R = c \frac{  v {\left( {{r}_{a}}\cos \left( {{\theta }_{a}} \right)-{{r}_{q}}\cos \left( {{\theta }_{q}} \right) \right)} - \sqrt{R_{0}^{2} c^{2} - {\left(R_{0}^{2} - {\left( {{r}_{a}}\cos \left( {{\theta }_{a}} \right)-{{r}_{q}}\cos \left( {{\theta }_{q}} \right) \right)}^{2}\right)} v^{2}}}{c^{2} - v^{2}}$

и радиус Лиенара-Вихерта

${{R}^{*}}=R-\frac{v}{c}\left( {{r}_{a}}\cos \left( {{\theta }_{a}} \right)-{{r}_{q}}\cos \left( {{\theta }_{q}} \right) \right)$

%======================================================================== 
\section{Расчёт инертной электромагнитной массы протона и нейтрона}


Представляет интерес расчёт инертной электромагнитной массы протона и нейтрона на основании предложенных в данной работе формул. 

Плотности распределения заряда для протона

${{\rho}_{p}} = \frac{e^{\left(-\frac{r_{q}^{2}}{r_{0}^{2}}\right)}}{\pi^{\frac{3}{2}} r_{0}^{3}}$
, где
$r_{0} = \sqrt{\frac{2}{3}} \left< r_{p} \right>_{rms}$
и
$\left< r_{p} \right>_{rms} = 0.8 \ fm$

и для нейтрона

${{\rho}_{n}} = \frac{2 \, \left< r_{n}^{2} \right> r_{q}^{2} {\left(\frac{2 \, r_{q}^{2}}{r_{1}^{2}} - 5\right)} e^{\left(-\frac{r_{q}^{2}}{r_{1}^{2}}\right)}}{15 \, \pi^{\frac{3}{2}} r_{1}^{7}}$
, где
$\left< r_{n}^{2} \right> = -\ 0.113 \ fm^2$
и
$r_{1} = 0.71 \sqrt{\frac{2}{5}}\ fm$

были взяты из работы \cite{Haddad}.

Аналитический результат расчёта интеграла
$\int\limits_{{{V}_{a}}}{\int\limits_{{{V}_{q}}}{\frac{\rho \left( {{r}_{q}} \right)\rho \left( {{r}_{a}} \right)}{R}}}\ d{{V}_{q}}d{{V}_{a}}$ 
с помощью разложения по сферическим функциям для протона составил
$\frac{1.875 \, \sqrt{2}}{\sqrt{\pi}}$
что равно $1.49603355150537$

Для контроля этот же интеграл был взят численно с помощью функции integrate.quad математического пакета scipy

$(1.4960348943817992, 0.0026474827067254846)$

Кроме того для численного интегрирования была опробована программа Cuba-4.2 двумя методами

VEGAS RESULT:   1.49695439 +- 0.00345851        p = 0.006

SUAVE RESULT:   1.49424975 +- 0.00149158        p = 1.000

Коэффициент преобразования полученного результата в сиситему СИ

$k = \frac{{\mu}_{0}}{ 4 \pi} \frac{e^2}{ 10^{-15}}$, при умножении на который вычисленная электромагнитная инертная масса протона составила $3.84027312853036 \times 10^{-30}$ кг что в $4.21573319817234$ раза больше массы электрона.

Аналитический результат расчёта того же интеграла для нейтрона

$\frac{\left(4.48918680252563 \times 10^{-28}\right) \, \sqrt{5} \sqrt{2} {\left(1824320471 \, \sqrt{10} \sqrt{5} + 14369256122481640385175552 \, \sqrt{2}\right)}}{\sqrt{\pi}}$

что соответствует $0.0162757880193542$

Результат численного интегрирования в программе Cuba-4.2

VEGAS RESULT:   0.00183803 +- 0.00000179        p = 0.000

SUAVE RESULT:   0.00183606 +- 0.00000183        p = 1.000

В системе СИ  электромагнитная инертная масса нейтрона $4.17794582972344 \times 10^{-32}$ кг что составляет $0.0458641985739987$ от массы электрона.
%========================================================================

\begin{thebibliography}{99}

\bibitem{LL2}
\textit{Ландау Л.Д. Лившиц Е.М. Теория поля. М. 1973}

\bibitem{rustot}
\textit{Re: Как запаздывающий Лиенар-Вихерт становится "незапаздывающим". Визуализация}
\\\texttt{http://www.sciteclibrary.ru/cgi-bin/yabb2/YaBB.pl?num=1528093569/330\#330}

\bibitem{tamm}
\textit{И.Е.Тамм. Основы теории электричества. М. 1957}

\bibitem{flugge}
\textit{З.Флюгге Задачи по квантовой механике т.2 М. "Мир" 1974. стр. 296}

\bibitem{misyuchenko}
\textit{В. Ганкин, Ю. Ганкин, О. Куприянова, И. Мисюченко. История электромагнитной массы}
%\\\texttt{http://fphysics.com/d/232484/d/istoriya_em_massy1.pdf}


\bibitem{Haddad}
S. Haddad and S. Suleiman
\textit{NEUTRON CHARGE DISTRIBUTION AND CHARGE DENSITY DISTRIBUTIONS IN LEAD ISOTOPES}
\textit{ACTA PHYSICA POLONICA B, Vol. 30 (1999) No 1}
\\\texttt{http://www.actaphys.uj.edu.pl/fulltext?series=Reg\&vol=30\&page=119}



\end{thebibliography}


\end{document}

