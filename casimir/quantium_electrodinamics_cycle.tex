\documentclass[12pt, letterpaper]{article}
\usepackage[T2A]{fontenc}
\usepackage[utf8]{inputenc}
\usepackage[english, russian]{babel}

\usepackage{graphicx}
\graphicspath{ {./images/} }

\usepackage{wrapfig}
\usepackage[rightcaption]{sidecap}

\begin{document}
\title{Квантово электродинамический аналог цикла Карно на динамическом эффекте Казимира}
\author{А.Ю. Дроздов}
\date{November 2020}

\begin{titlepage}
\maketitle
\end{titlepage}


\begin{abstract}
К вопросу о теоретических основах построения устройств извлечения электромагнитной энергии
\end{abstract}



\section{Идея к обсуждению}

я уже на эту тему немало почитал. но особенно меня поразило вот это
https://youtu.be/fXtxNdvP3E0

из того, что он рассказал получается, что вода, пропущенная через его установку, после ее выключения ведет себя как живая. какое то время она сама генерирует тепло

я читал и размышлял на тему БТГ. на тему основных принципов их построения. знакомство с ртга 37 для меня стало переломным моментом в понимании откуда может получаться прибавка энергии. 

Ю.П. Степановский мне подсказал, что есть такое явление сонолюменисценция

почитав об этом явлении в википедии я узнал о связи сонолюменисценции с эффектом Казимира

а познакомившись с эффектом Казимира, и поразмыслив над ним я получил понимание принципа работы если не всех бтг, то по крайней мере определенного их класса.

дальнейшие размышления и поиск натолкнули меня на англоязычную статью \cite{Chan2001} которую я Вам сегодня выслал. я пока сам еще не изучил. но судя по аннотации она содержит ту же идею, которая мне открылась

мои мысли на эту тему. в эффекте Казимира если расстояние между пластинами 10 нанометров, то пластины приталкиваются внешним электромагнитным давлением равным 1 атмосфере

это означает что в диапазоне длинн волн от $20$ нм до бесконечности сосредоточена энергия нулевых колебаний вакуумной среды $\frac{H^2}{8*pi}$ равная $10^5$ джоулей в кубическом метре плюс пожалуй столько же энергии$\frac{E^2}{8*pi}$

$\displaystyle {F_{c} \over A}={\hbar c\pi ^{2} \over 240d^{4}}$

это формула величины давления приталкивания  пластин в эффекте Казимира из википедии

важно что там в знаменателе $d^4$

если вместе с пластинами имеется механизм который может запасать потенциальную энергию (образно говоря пружинка) то при уменьшении расстояния между пластинами от $10$ нанометров до $5$ нанометров, то пружинка запасет энергию равную интегралу от $1/d^4$ в пределах от $5$ до $10$ нм

при достижении расстояния между пластинами $5$ нанометров давление приталкивания пластин внешним электромагнитным полем составит 16 атмосфер, следовательно в диапазоне длин волн от $10$ до $20$ нм энергия $\frac{H^2}{8*pi}$ составляет $15 * 10^5$ джоулей в кубическом метре. я так понимаю, что поскольку после уменьшения зазора в резонаторе эти колебания уже не могут существовать внутри резонатора, из-за чего собственно пластины станут приталкиваться сильнее внешним полем, то эта энергия должна (или по крайней мере может) уйти из резонатора путем излучения соостветствуюшего диапазона


теперь начинаем следующий цикл и раздвигаем пластины в исходное положение за счет энергии накопленной в пружинке.

Вот собственно и есть мой предположительный механизм выкачки энергии из нулевых колебаний вакуума


Устройство Слободянюка по моему мнению работает на этом принципе, только вместо пластинок там пузырьки

по началу я предполагал следующее

и когда Слободянюк пишет, что он утилизирует энергию гамма фона, то он может быть и недалек от истины в том смысле что он может утилизировать энергию нулевых колебаний вакуума либо гамма спектра либо рентгеновского. Только я предполагаю, не вся излучаемая пузырьками радиация может быть поглощаема водой поэтому было бы интересно прийти к его установке с дозиметром. Его установка сама может быть источником определенного фона и это надо учитывать при эксплуатации. я имею в виду технику безопасности

но пообщавшись с самим Слободянюком я понял что мое понимание больше подходит к кавитационщикам

по его словам устройства кавитационциков фонят в 4 раза больше природного фона

а его устройство наоборот снижает гамма фон

 на сегодняшний день мое предположение. пока что не подтвержденное каким либо прямым опытом

кроме того в первой части текста я тоже допустил достаточно сильную неточность

интеграл от $1/d^4$ в пределах от $d1$ до $d2$ это работа совершаемая внешним электромагнитным полем нулевых колебаний вакуума, который приталкивает пластины

что касается пружинки, то пружинка подчиняется закону Гука. (если в качестве пружинки у нас сжимаемый внутри пузырьков газ, то этот газ подчиняется термодинамическому уравнению адиабатного процесса, или в общем случае политропного процесса. кроме того в случае пузырьков пластины заменяются сферической поверхностью, поэтому сила формула силы Казимира будет несколько иной)

поэтому далеко не факт что вся работа совершаемая внешним полем сможет аккумулироваться в "пружинке", то есть этот вопрос требует дополнительного исследования

в экспериментальной работе иностранных авторов, которую я Вам высылал на английском приводится результат экмперимента очень похожего на мой мысленный эксперимент

там у них сила Казимира действует на растяжение пружинки

они показывают, что можно подобрать режим, при котором будут наблюдаться осциляции, то есть колебания позолоченной пластинки, на которую с одной стороны действует упругая сила пружинки по закону Гука, а с другой стороны сила Казимира со стороны позолоченной сферы

для случая взаимодействия пластинки со сферой авторы приводят формулу силы Казимира у которой расстояние в знаменателе расстояние между пластинкой не в четвертой степени, а в кубе


в своем эксперименте они раскачивали пластинку приложением внешнего поля регулируемой частоты и наблюдали колебания этой пластинки, которые, как они пишут носят нелинейный характер, и кроме того они наблюдают гистерезис

меня очень заинтересовал этот гисетерезис.

С одной стороны авторы ссылаясь на том Механики Ландау и Лифшица пишут, что наблюдаемые смещение резонансной частоты и гистерезис являются закономерным следствием нелинейности полученного осциллятора

то есть осциллятор состоящий только из пружинки и грузика это система линейная

а вот прибавление к системе силы Казимира, действующей на этот грузик, приводит к нелинейности

с другой стороны я хочу понять, а не является ли наблюдаемый авторами гистерезис следствием того, что в процессе колебания и изменения вакуумного промежутка между пластиной и сферой в этом промежутке энергия нулевых колебаний то уменьшается в ходе уменьшения этого промежутка, то увеличивается в ходе его увеличения

в данной работе авторы не обьясняют что происходит в каждом цикле с этой разностью вакуумной энергии промежутка.

откуда она берется в каждом цикле при увеличении промежутка: подкачивается ли она из окружающего данный промежуток соседнего вакуума или она берется полностью из той работы против внешнего поля которую мы затрачиваем раздвигая обкладки

согласно формулам получается, что энергия в промежутке увеличивается как раз на величину работы затраченной против внешних сил давления внешних нулевых колебаний

но надо заметить, что при уменьшении промежутка уже внешние силы нулевых колебаний затрачивают ту же работу, а энергия в промежутке уменьшается

и куда девается энергия казимировкого вакуума внутри промежутка при уменьшении промежутка?

или может быть уходит обратно в соседний вакуум в виде тех же неуловимых нулевых колебаний

или рассеивается в виде излучения, которое можно попытаться собирать

или может быть хотя бы часть этого избытка преобразуется в энергию колебаний пластинки.

посмотрите рисунок 4b

на этом рисунке авторы отобразили результаты следующего теста: зафиксировав частоту прилагаемой внешней силы, колеблющей пластинку, они стали изменять расстояние между пластинкой и сферой

интересно что: когда они приближали сферу к пластинке, они достигли максимума резонансной амплитуды колебаний при расстоянии между пластинкой и сферой порядка 118 нм.

при этом амплитуда крутильных колебаний пластинки составила порядка 200 микрорадиан

но когда они стали отдалять сферу от пластинки, то они достигли максимума резонансной амплитуды на расстоянии порядка 122.5 нм

при этом амплитуда крутильных колебаний пластинки составила уже порядка 300 микрорадиан

казалось бы что происходит: амплитуда вынуждающей колебания силы неизменна

а вот резонансные пики различны

интересно, что бОльший резонансный пик был достигнут в режиме увеличения расстояния между сферой и колеблющейся пластинкой

что бы это значило?

при увеличении расстояния (отодвигая сферу) мы затрачиваем работу против внешних сил нулевых колебаний снаружи
при этом также увеличивается энергия казимировского вакуума в промежутке

и именно в этом режиме получается бОльший резонансный пик

хотя с другой стороны ту же величину пика резонансных колебаний $300$ микрорадиан они получают на той же пластине в линейном режиме. то есть без сферы, влияющей на пластину силой Казимира

хотя стоп, они на этот счет приводят какие то противоречивые данные

сперва рисуют $270$ микрорадиан при наложении $72.5$ микровольта, а потом рисуют пик $300$ микрорадиан при наложении $55.5$ микровольт

кстати в конце работы они пишут что излучение всетаки есть но в их системе оно пренебрежимо мало

получается, что мое предположение, которое Вы выделили красным цветом подтверждается хотя и не понятна доля энергии идущей на излучение, отчего она зависит и как ее увеличивать

они тут еще пишут следующий текст который я не очень понял


In Fig. 4a, we used a constant quality factor Q 7150 to fit all four resonance peaks at different distances. Further improvements in sensitivity could enable us to explore possible changes in Q with distance.

переводчик reverso context переводит так

В Fig. 4a мы использовали коэффициент постоянного качества Q 7150, чтобы уместить все четыре пика резонанса на разных расстояниях. Дальнейшие улучшения чувствительности могут позволить нам исследовать возможные изменения в Q с расстоянием.

% https://en.wikipedia.org/wiki/Q_factor

я в общем понял, что означает эта их фраза: при интерпретации результатов измерений они предположили что добротность осциллятора константа, то есть не зависит от расстояния между сферой и колеблющейся пластинкой

и для того чтобы узнать как на самом зависит добротность от величины Казимировского промежутка им нужны дальнейшие исследования с улучшением чувствительности

но если их предположение о постоянной величине добротности верно, тогда исходя из рисунка 4 а можно предположить что поскольку с уменьшением казимировского промежутка уменьшается резонансная частота и чуть чуть уменьшается величина резонансного пика, то значит уменьшается и рассеиваемая мощность, то есть потери

не означает ли это, что потери компенсируются энергией которая в каждом цикле колебаний, должна куда-то деваться из промежутка при его уменьшении?

если это так, тогда моя гипотеза может оказаться верной и мы можем по аналогии с циклом Карно в термодинамике построить квантово электродинамический цикл на основе динамического эффекта Казимира

и теоретически основываясь на этом цикле проектировать безтопливные генераторы...


\begin{thebibliography}{99}


\bibitem{Chan2001}
\textit{Chan HB, Aksyuk VA, Kleiman RN, Bishop DJ, Capasso F. Nonlinear micromechanical Casimir oscillator. Phys Rev Lett. 2001 Nov 19;87(21):211801. doi: 10.1103/PhysRevLett.87.211801. Epub 2001 Oct 31. PMID: 11736330.}

\bibitem{LLM}
\textit{Ландау Л.Д. Лившиц Е.М. Механика.}

\end{thebibliography}

\end{document}