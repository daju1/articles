\documentclass{article}
\usepackage[T2A]{fontenc}
\usepackage[utf8]{inputenc}
\usepackage[english, russian]{babel}

\begin{document}

\title{Об инерционных свойствах электромагнитной массы}

\author{А.Ю.Дроздов}

\date{Jan 30, 2020}


%========================================================================

\section{Introduction}



Пусть частица с заданным распределением объёмной плотности электрического заряда $\rho \left( r \right)$ приобретает ускорение $\overrightarrow{a}$ . Найти силу, действующую на распределённый в объёме электрический заряд этой частицы со стороны электрического поля самоиндукции. 
Для решения этой задачи электрическое поле самоиндукции $\overrightarrow{E}=-\frac{1}{c}\frac{\partial \overrightarrow{A}}{\partial t}$ 
Выразим исходя из выражения векторного потенциала Лиенара-Вихерта [1] $\overrightarrow{A}=\frac{\overrightarrow{v}}{c}\frac{q}{R-\frac{\overrightarrow{v}\cdot \overrightarrow{R}}{c}}=\frac{\overrightarrow{v}}{c}\frac{q}{{{R}^{*}}}$ дифференцирование которого приводит к выражению [2]
\[\overrightarrow{E}=\frac{dq}{{{R}^{*}}^{2}}\left\{ \frac{\overrightarrow{v}}{c}\left( \frac{R}{{{R}^{*}}}\left( \frac{{{v}^{2}}}{{{c}^{2}}}-\frac{\overrightarrow{a}\cdot \overrightarrow{R}}{{{c}^{2}}}-1 \right)+1 \right)-\frac{\overrightarrow{a}R}{{{c}^{2}}} \right\}\] 
В системе СИ перед этим выражением появляется множитель $\frac{1}{4\pi {{\varepsilon }_{0}}}=\frac{{{\mu }_{0}}{{c}^{2}}}{4\pi }=\frac{{{c}^{2}}}{{{10}^{7}}}$ 
Направляя в сферической системе координат вектор ускорения вдоль оси $z$  запишем
${{R}^{*}}=R-\frac{v}{c}\left( {{z}_{a}}-{{z}_{q}} \right)$  
\[\overrightarrow{a}\cdot \overrightarrow{R}=a\left( {{z}_{a}}-{{z}_{q}} \right)=a\left( {{r}_{a}}\cos \left( {{\theta }_{a}} \right)-{{r}_{q}}\cos \left( {{\theta }_{q}} \right) \right)\]
Где следуя Тамму [3], индексом  обозначены координаты заряда, а индексом  обозначены координаты точки наблюдения
\[\overrightarrow{E}=\iiint\limits_{{{V}_{q}}}{\left\{ \frac{\overrightarrow{v}}{c}\left( \frac{R}{{{R}^{*}}}\left( \frac{{{v}^{2}}}{{{c}^{2}}}-\frac{\overrightarrow{a}\cdot \overrightarrow{R}}{{{c}^{2}}}-1 \right)+1 \right)-\frac{\overrightarrow{a}R}{{{c}^{2}}} \right\}\frac{\rho \left( {{r}_{q}} \right){{r}_{q}}^{2}\sin \left( {{\theta }_{q}} \right)}{{{R}^{*}}^{2}}\ }d{{\theta }_{q}}d{{\varphi }_{q}}d{{r}_{q}}\]
Чтобы учесть запаздывание потенциала следует решить систему уравнений
$s=v\left( t-t' \right)+a{{\left( t-t' \right)}^{2}}$ 
$R=c\left( t-t' \right)$ 
учитывая, что по теореме косинусов
${{R}^{2}}={{R}_{0}}^{2}+{{s}^{2}}-2{{R}_{0}}s\cos \left( \alpha  \right)={{R}_{0}}^{2}+{{s}^{2}}-2{{R}_{0}}s\frac{{{z}_{q'}}-{{z}_{a'}}}{{{R}_{0}}}={{R}_{0}}^{2}+{{s}^{2}}+2s\left( {{z}_{a'}}-{{z}_{q'}} \right)$ 
где ${R}_{0}$ расстояние от точки источника заряда к точке наблюдения без учёта запаздывания
Действующая на заряд со стороны электрического поля самоиндукции инерционная сила равна
$\overrightarrow{{{F}_{z}}}=\iiint\limits_{{{V}_{a}}}{\overrightarrow{{{E}_{z}}}\rho \left( {{r}_{a}} \right){{r}_{a}}^{2}\sin \left( {{\theta }_{a}} \right)}\ d{{\theta }_{a}}d{{\varphi }_{a}}d{{r}_{a}}$ 
Из приведенных формул видно, что сила инерции электромагнитной массы зависит от вида функции распределения плотности заряда в пространстве, а также от скорости и ускорения заряда.
В приближении малых скоростей ${}^{v}/{}_{c}\ll 1$  и малых ускорений $a{{r}_{0}}\ll {{c}^{2}}$ 
\[\overrightarrow{E}=\iiint\limits_{{{V}_{q}}}{\left\{ -\frac{\overrightarrow{a}R}{{{c}^{2}}} \right\}\frac{\rho \left( {{r}_{q}} \right){{r}_{q}}^{2}\sin \left( {{\theta }_{q}} \right)}{{{R}^{*}}^{2}}\ }d{{\theta }_{q}}d{{\varphi }_{q}}d{{r}_{q}}\]
Откуда 
${{F}_{z}}=-\frac{\overrightarrow{a}}{{{c}^{^{2}}}}\int\limits_{{{V}_{a}}}{\int\limits_{{{V}_{q}}}{\frac{\rho \left( {{r}_{q}} \right)\rho \left( {{r}_{a}} \right)}{R}}}\ d{{V}_{q}}d{{V}_{a}}$ 
Сопоставляя с законами Ньютона для электромагнитной массы получаем выражение
$m=\frac{1}{{{c}^{^{2}}}}\int\limits_{{{V}_{a}}}{\int\limits_{{{V}_{q}}}{\frac{\rho \left( {{r}_{q}} \right)\rho \left( {{r}_{a}} \right)}{R}}}\ d{{V}_{q}}d{{V}_{a}}$
Рассчитаем теперь электромагнитную массу равномерно заряженной сферы радиуса ${{r}_{0}}$ 
Расстояние между координатами заряда и точки наблюдения сферической системе координат имеет вид
${{R}_{0}}=\sqrt{\begin{align}
  & {{\left( {{r}_{a}}\sin \left( {{\theta }_{a}} \right)\cos \left( {{\varphi }_{a}} \right)-{{r}_{q}}\sin \left( {{\theta }_{q}} \right)\cos \left( {{\varphi }_{q}} \right) \right)}^{2}}+ \\ 
 & {{\left( {{r}_{a}}\sin \left( {{\theta }_{a}} \right)\sin \left( {{\varphi }_{a}} \right)-{{r}_{q}}\sin \left( {{\theta }_{q}} \right)\sin \left( {{\varphi }_{q}} \right) \right)}^{2}}+ \\ 
 & {{\left( {{r}_{a}}\cos \left( {{\theta }_{a}} \right)-{{r}_{q}}\cos \left( {{\theta }_{q}} \right) \right)}^{2}} \\ 
\end{align}}$ 
Для его упрощения ввиду симметричности задачи поле не зависит от координаты ${{\varphi }_{a}}$поэтому мы можем положив ${{\varphi }_{a}}=0$ упростить формулу а вместо интегрирования по ${{\varphi }_{a}}$ просто применить умножение на $2\pi $ 
$\begin{align}
  & {{R}_{0}}=\sqrt{\begin{align}
  & {{\left( {{r}_{a}}\sin \left( {{\theta }_{a}} \right)-{{r}_{q}}\sin \left( {{\theta }_{q}} \right)\cos \left( {{\varphi }_{q}} \right) \right)}^{2}}+ \\ 
 & {{\left( -{{r}_{q}}\sin \left( {{\theta }_{q}} \right)\sin \left( {{\varphi }_{q}} \right) \right)}^{2}}+ \\ 
 & {{\left( {{r}_{a}}\cos \left( {{\theta }_{a}} \right)-{{r}_{q}}\cos \left( {{\theta }_{q}} \right) \right)}^{2}} \\ 
\end{align}}=\sqrt{\begin{align}
  & {{\left( {{r}_{a}}\sin \left( {{\theta }_{a}} \right) \right)}^{2}}-2\left( {{r}_{a}}\sin \left( {{\theta }_{a}} \right){{r}_{q}}\sin \left( {{\theta }_{q}} \right)\cos \left( {{\varphi }_{q}} \right) \right)+ \\ 
 & {{\left( {{r}_{q}}\sin \left( {{\theta }_{q}} \right)\cos \left( {{\varphi }_{q}} \right) \right)}^{2}}+{{\left( {{r}_{q}}\sin \left( {{\theta }_{q}} \right)\sin \left( {{\varphi }_{q}} \right) \right)}^{2}}+ \\ 
 & {{\left( {{r}_{a}}\cos \left( {{\theta }_{a}} \right)-{{r}_{q}}\cos \left( {{\theta }_{q}} \right) \right)}^{2}} \\ 
\end{align}} \\ 
 & =\sqrt{{{\left( {{r}_{a}}\sin \left( {{\theta }_{a}} \right) \right)}^{2}}-2{{r}_{a}}\sin \left( {{\theta }_{a}} \right){{r}_{q}}\sin \left( {{\theta }_{q}} \right)\cos \left( {{\varphi }_{q}} \right)+{{\left( {{r}_{q}}\sin \left( {{\theta }_{q}} \right) \right)}^{2}}+{{\left( {{r}_{a}}\cos \left( {{\theta }_{a}} \right)-{{r}_{q}}\cos \left( {{\theta }_{q}} \right) \right)}^{2}}} \\ 
\end{align}$

Плотность заряда положим $\rho \left( r \right)=\frac{e}{{}^{4}/{}_{3}\pi {{r}_{0}}^{3}}$ 
В работе [4] приводится 

%========================================================================

\begin{thebibliography}{99}



Литература
\bibitem{Ландау}
1. Ландау Л.Д. Лившиц Е.М. Теория Поля. М. 1973
\bibitem{rustot}
2. Re: Как запаздывающий Лиенар-Вихерт становится "незапаздывающим". Визуализация http://www.sciteclibrary.ru/cgi-bin/yabb2/YaBB.pl?num=1528093569/330#330
\bibitem{Тамм}
3. И.Е.Тамм. Основы теории электричества. М. 1957
\bibitem{Мисюченко}
4. В. Ганкин, Ю. Ганкин, О. Куприянова, И. Мисюченко. История электромагнитной массы http://fphysics.com/d/232484/d/istoriya_em_massy1.pdf 
\bibitem{Haddad}
5. S. Haddad and S. Suleiman NEUTRON CHARGE DISTRIBUTION AND CHARGE DENSITY DISTRIBUTIONS IN LEAD ISOTOPES, ACTA PHYSICA POLONICA B, Vol. 30 (1999) No 1 http://www.actaphys.uj.edu.pl/fulltext?series=Reg&vol=30&page=119
6. 

\end{thebibliography}

\end{document}

