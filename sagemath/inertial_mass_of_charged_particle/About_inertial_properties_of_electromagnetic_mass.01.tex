\documentclass{article}
\usepackage[T2A]{fontenc}
\usepackage[utf8]{inputenc}
\usepackage[english, russian]{babel}

\begin{document}

\title{Об инерционных свойствах электромагнитной массы}

\author{А.Ю.Дроздов}

\date{Jan 30, 2020}


%========================================================================

\section{Introduction}



Пусть частица с заданным распределением объёмной плотности электрического заряда $\rho \left( r \right)$ приобретает ускорение $\overrightarrow{a}$ . Найти силу, действующую на распределённый в объёме электрический заряд этой частицы со стороны электрического поля самоиндукции.

Для решения этой задачи электрическое поле самоиндукции $\overrightarrow{E}=-\frac{1}{c}\frac{\partial \overrightarrow{A}}{\partial t}$
выразим исходя из выражения векторного потенциала Лиенара-Вихерта \cite{LL2} $\overrightarrow{A}=\frac{\overrightarrow{v}}{c}\frac{q}{R-\frac{\overrightarrow{v}\cdot \overrightarrow{R}}{c}}=\frac{\overrightarrow{v}}{c}\frac{q}{{{R}^{*}}}$ дифференцирование которого приводит к выражению \cite{rustot}

\[\overrightarrow{E}=\frac{dq}{{{R}^{*}}^{2}}\left\{ \frac{\overrightarrow{v}}{c}\left( \frac{R}{{{R}^{*}}}\left( \frac{{{v}^{2}}}{{{c}^{2}}}-\frac{\overrightarrow{a}\cdot \overrightarrow{R}}{{{c}^{2}}}-1 \right)+1 \right)-\frac{\overrightarrow{a}R}{{{c}^{2}}} \right\}\]

В системе СИ перед этим выражением появляется множитель $\frac{1}{4\pi {{\varepsilon }_{0}}}=\frac{{{\mu }_{0}}{{c}^{2}}}{4\pi }=\frac{{{c}^{2}}}{{{10}^{7}}}$

Направляя в сферической системе координат вектор ускорения вдоль оси $z$  запишем

${{R}^{*}}=R-\frac{v}{c}\left( {{z}_{a}}-{{z}_{q}} \right)$

\[\overrightarrow{a}\cdot \overrightarrow{R}=a\left( {{z}_{a}}-{{z}_{q}} \right)=a\left( {{r}_{a}}\cos \left( {{\theta }_{a}} \right)-{{r}_{q}}\cos \left( {{\theta }_{q}} \right) \right)\]

Где следуя Тамму \cite{tamm}, индексом $q$ обозначены координаты заряда, а индексом $a$ обозначены координаты точки наблюдения

\[\overrightarrow{E}=\int\limits_{{{r}_{q}}}\int\limits_{{{\varphi}_{q}}}\int\limits_{{{\theta}_{q}}}\\
{\left\{ \frac{\overrightarrow{v}}{c}\left( \frac{R}{{{R}^{*}}}\left( \frac{{{v}^{2}}}{{{c}^{2}}}-\frac{\overrightarrow{a}\cdot \overrightarrow{R}}{{{c}^{2}}}-1 \right)+1 \right)-\frac{\overrightarrow{a}R}{{{c}^{2}}} \right\}\frac{\rho \left( {{r}_{q}} \right){{r}_{q}}^{2}\sin \left( {{\theta }_{q}} \right)}{{{R}^{*}}^{2}}\ }d{{\theta }_{q}}d{{\varphi }_{q}}d{{r}_{q}}\]

Учитывая, что по теореме косинусов
${{R}^{2}}={{R}_{0}}^{2}+{{s}^{2}}-2{{R}_{0}}s\cos \left( \alpha  \right)={{R}_{0}}^{2}+{{s}^{2}}-2{{R}_{0}}s\frac{{{z}_{q'}}-{{z}_{a'}}}{{{R}_{0}}}={{R}_{0}}^{2}+{{s}^{2}}+2s\left( {{z}_{a'}}-{{z}_{q'}} \right)$
где  расстояние от точки источника заряда к точке наблюдения без учёта запаздывания
Действующая на заряд со стороны электрического поля самоиндукции инерционная сила равна

$\overrightarrow{{{F}_{z}}}=\int\limits_{{{r}_{a}}}\int\limits_{{{\varphi}_{a}}}\int\limits_{{{\theta}_{a}}}\\
{\overrightarrow{{{E}_{z}}}\rho \left( {{r}_{a}} \right){{r}_{a}}^{2}\sin \left( {{\theta }_{a}} \right)}\ d{{\theta }_{a}}d{{\varphi }_{a}}d{{r}_{a}}$


Из приведенных формул видно, что сила инерции электромагнитной массы зависит от вида функции распределения плотности заряда в пространстве, а также от скорости и ускорения заряда.
В приближении малых скоростей ${}^{v}/{}_{c}\ll 1$  и малых ускорений $a{{r}_{0}}\ll {{c}^{2}}$
\[\overrightarrow{E}=\int\limits_{{{r}_{q}}}\int\limits_{{{\varphi}_{q}}}\int\limits_{{{\theta}_{q}}}\\
{\left\{ -\frac{\overrightarrow{a}R}{{{c}^{2}}} \right\}\frac{\rho \left( {{r}_{q}} \right){{r}_{q}}^{2}\sin \left( {{\theta }_{q}} \right)}{{{R}^{*}}^{2}}\ }d{{\theta }_{q}}d{{\varphi }_{q}}d{{r}_{q}}\]
Откуда
${{F}_{z}}=-\frac{\overrightarrow{a}}{{{c}^{^{2}}}}\int\limits_{{{V}_{a}}}{\int\limits_{{{V}_{q}}}{\frac{\rho \left( {{r}_{q}} \right)\rho \left( {{r}_{a}} \right)}{R}}}\ d{{V}_{q}}d{{V}_{a}}$
Сопоставляя с законами Ньютона для электромагнитной массы получаем выражение
$m=\frac{1}{{{c}^{^{2}}}}\int\limits_{{{V}_{a}}}{\int\limits_{{{V}_{q}}}{\frac{\rho \left( {{r}_{q}} \right)\rho \left( {{r}_{a}} \right)}{R}}}\ d{{V}_{q}}d{{V}_{a}}$ (в системе сгс) и соответственно $m=\frac{{{\mu }_{0}}}{4\pi }\int\limits_{{{V}_{a}}}{\int\limits_{{{V}_{q}}}{\frac{\rho \left( {{r}_{q}} \right)\rho \left( {{r}_{a}} \right)}{R}}}\ d{{V}_{q}}d{{V}_{a}}$ в системе СИ
Рассчитаем теперь электромагнитную массу равномерно заряженной сферы радиуса ${{r}_{0}}$

Поскольку расстояние между координатами заряда и точки наблюдения ${{R}_{0}}=\left|\overrightarrow{r_{q}} - \overrightarrow{r_{a}}\right|$ находится в знаменателе, то в сферической системе координат можно применить разложение по сферическим гармоникам следующего вида  \cite{flugge} если $\left( {{r}_{q}}<{{r}_{a}} \right)$ то

$\frac{1}{\left| \overrightarrow{{{r}_{q}}}-\overrightarrow{{{r}_{a}}} \right|}=\frac{1}{{{r}_{a}}}\sum\limits_{l=0}^{\infty }{{{\left( \frac{{{r}_{q}}}{{{r}_{a}}} \right)}^{l}}{{P}_{l}} \cos \left( \gamma  \right)}$

и если $\left( {{r}_{a}}<{{r}_{q}} \right)$ то

$\frac{1}{\left| \overrightarrow{{{r}_{q}}}-\overrightarrow{{{r}_{a}}} \right|}=\frac{1}{{{r}_{q}}}\sum\limits_{l=0}^{\infty }{{{\left( \frac{{{r}_{a}}}{{{r}_{q}}} \right)}^{l}}{{P}_{l}} \cos \left( \gamma  \right)}$

В данной формуле ${{P}_{l}} \cos \left( \gamma  \right)$ это полиномы Лежандра аргумент которых $\gamma$ есть угол между векторами ${{r}_{q}}$  и ${{r}_{a}}$. Применяя формулу, известную как теорему сложения

${{P}_{l}}\cos \left( \gamma  \right)=\frac{4\pi }{2l+1}\sum\limits_{m=-l}^{l}{Y_{l,m}^{*}\left( {{\theta }_{a}},{{\varphi }_{a}} \right)}\ {{Y}_{l,m}}\left( {{\theta }_{q}},{{\varphi }_{q}} \right)$

получаем способ аналитического вычисления интеграла инертной электромагнитной массы.

Пусть заряд представляет собой сферу, равномерно заряженную по всему объёму тогда плотность заряда составит $\rho \left( r \right)=\frac{e}{{}^{4}/{}_{3}\pi {{r}_{0}}^{3}}$

Производя вычисления для интеграла электромагнитной массы получено значение 

$m =\frac{1}{{{c}^{^{2}}}}\frac{6}{5}\frac{e^2}{{{r}_{0}}}$ (сгс) и
$m =\frac{{{\mu }_{0}}}{4\pi }\frac{6}{5}\frac{e^2}{{{r}_{0}}}$ (СИ)

Пусть заряд представляет собой сферическую поверхность с равномерным поверхностным распределением заряда по поверхности сферы, равным $\sigma=\frac{e}{4\pi {{r}_{0}}^{2}})$ тогда для вычисления инертной электромагнитной массы потребуется формула
$m=\frac{1}{{{c}^{^{2}}}}\int\limits_{{{S}_{a}}}{\int\limits_{{{S}_{q}}}{\frac{\sigma \left( {{r}_{q}} \right)\sigma \left( {{r}_{a}} \right)}{R}}}\ d{{S}_{q}}d{{S}_{a}}$

Вычисления по которой дают следующий результат

$m =\frac{1}{{{c}^{^{2}}}}\frac{e^2}{{{r}_{0}}}$ (сгс) и
$m =\frac{{{\mu }_{0}}}{4\pi }\frac{e^2}{{{r}_{0}}}$ (СИ)

В работе \cite{misyuchenko} приводится способ вычисления инертной электромагнитной массы электрона исходя из коэффициента самоиндукции сферы и производной тока сферы по времени.
Однако приведенная авторами формула коэффициента самоиндукции сферы $L =\frac{{{\mu }_{0}}{r}_{0}}{2\pi }$ (СИ) на мой взгляд занижена в 2 раза. 
В ходе данной работы мною было произведено вычисление коэффициента самоиндукции сферической поверхности с равномерно распределённым поверхностным зарядом. Полученная мною формула $L =\frac{{{\mu }_{0}}{r}_{0}}{\pi }$ (СИ) .
Результат вычислений электромагнитной массы электрона авторы работы \cite{misyuchenko} приводят следующий $m =\frac{{{\mu }_{0}}}{8\pi }\frac{e^2}{{{r}_{0}}}$ (СИ) то есть в два раза меньший, чем полученный в данной работе. 
Далее авторы пишут, "Отметим тот важный факт, что выведенная из закона самоиндукции масса полностью совпадает с Эйнштейновской массой $m=\frac{U}{c^2}$ , если под полной энергией электрона U понимать собственную энергию его электрического поля."

Однако на мой взгляд в данном вопросе имеется путаница, связанная с коэффициентом одна вторая. 

Русская википедия в статье "Классический радиус электрона"  на момент написания данной работы даёт следующее определение:

Классический радиус электрона равен радиусу полой сферы, на которой равномерно распределён заряд, если этот заряд равен заряду электрона, а потенциальная энергия электростатического поля ${U}_{0}$  полностью эквивалентна половине массы электрона (без учета квантовых эффектов):

${\displaystyle U_{0}={\frac {1}{2}}{\frac {1}{4\pi \varepsilon _{0}}}\cdot {\frac {e^{2}}{r_{0}}}={\frac {1}{2}}m_{0}c^{2}}$.

Возникает закономерный вопрос: а чему соответствует вторая половина массы электрона?

Далее, Тамм \cite{tamm} для собственной электрической энергии заряженного шара радиуса $a$ находит $W=\frac{e^2}{2a}$ если заряд распределён на поверхности шара и $W=\frac{3e^2}{5a}$ если заряд распределён по всему объёму шара. Появление коэффициента $\frac{1}{2}$  в формуле энергии системы зарядов Тамм объясняет тем, что "в сумму энергия каждой пары зарядов входит дважды, так, например, в ней встретится как член ${e_{1}}{e_{1}}/{R_{12}}$ так и равный ему член ${e_{2}}{e_{1}}/{R_{21}}$".

Однако в задаче, рассмотренной в самом начале данной работы при нахождении силы, действующей на распределённый в объёме электрический заряд этой частицы со стороны электрического поля самоиндукции, подобные рассуждения неприменимы. Поскольку хотя и каждая пара зарядов $d{e}_{1}$ и $d{e}_{2}$ при совместном поступательном движении создаёт две одинаковые как по виду формулы, так и по значинию силы самоиндукции, они обе должны быть включены в общую электромагнитную инерцию, поскольку одна из них - это сила самоиндукции действующая на заряд $d{e}_{1}$  со стороны поля заряда $d{e}_{2}$ , а вторая это сила самоиндукции действующая на заряд $d{e}_{2}$  со стороны поля заряда $d{e}_{1}$.

%======================================================================== 

\begin{thebibliography}{99}

\bibitem{LL2}
\textit{Ландау Л.Д. Лившиц Е.М. Теория поля. М. 1973}

\bibitem{rustot}
\textit{Re: Как запаздывающий Лиенар-Вихерт становится "незапаздывающим". Визуализация}
\\\texttt{http://www.sciteclibrary.ru/cgi-bin/yabb2/YaBB.pl?num=1528093569/330\#330}

\bibitem{tamm}
\textit{И.Е.Тамм. Основы теории электричества. М. 1957}

\bibitem{flugge}
\textit{З.Флюгге Задачи по квантовой механике т.2 М. "Мир" 1974. стр. 296}

\bibitem{misyuchenko}
\textit{В. Ганкин, Ю. Ганкин, О. Куприянова, И. Мисюченко. История электромагнитной массы}
%\\\texttt{http://fphysics.com/d/232484/d/istoriya_em_massy1.pdf}


\bibitem{Haddad}
S. Haddad and S. Suleiman
\textit{NEUTRON CHARGE DISTRIBUTION AND CHARGE DENSITY DISTRIBUTIONS IN LEAD ISOTOPES}
\textit{ACTA PHYSICA POLONICA B, Vol. 30 (1999) No 1}
\\\texttt{http://www.actaphys.uj.edu.pl/fulltext?series=Reg\&vol=30\&page=119}



\end{thebibliography}


\end{document}

