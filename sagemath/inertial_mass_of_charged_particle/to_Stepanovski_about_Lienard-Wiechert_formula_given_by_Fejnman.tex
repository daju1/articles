\documentclass{article}
\usepackage[T2A]{fontenc}
\usepackage[utf8]{inputenc}
\usepackage[english, russian]{babel}

\begin{document}

\title{Юрию Петровичу Степановскому о выводе потенциалов Лиенара Вихерта в лекциях Фейнмана}

\author{А.Ю.Дроздов}

%\date{Jan 30, 2020}


\begin{titlepage}
\maketitle
\end{titlepage}

%========================================================================

\section{Постановка задачи}

Добрый день Юрий Петрович

Вы пишете:  пожалуйста, разберитесь сами с Ферми и Фейнманом, а потом мне расскажете.

Хорошо, 

\section{ Предыстория: проблема 4/3}

 началось все с того, что я начал разбирать решение задачу номер 689 из Батыгина Топтыгина (издание 1962 года)


Найти силу $\vec{F}$, с которой заряженная симметричная частица действует сама на себя (сила самодействия) при ускоренном поступательном движении с малой скоростью $v << c$. Запаздывание и лоренцево сокращение не учитывать.
Указание. Вычислить равнодействующую сил, приложенных к малым элементам $de$ заряда частицы, воспользовавшись лиенар-вихертовым выражением напряжённости поля (XII. 25)  

Под этим выражением Батыгин Топтыгин приводят выражение

$\vec{E}\left(\vec{r},t\right) = e \frac{\left(1-\beta^2\right) \left(\vec{n}-\vec{v}/c \right)}{\left(1-\vec{n}\cdot\vec{v}/c\right)^3\,R^2} + e \frac{ \vec{n} \times \left[ \left( \vec{n}-\vec{v}/c\right) \times \vec{a}\right]}{c^2 \left(1-\vec{n} \cdot \vec{v}/c \right)^3 \,R }$

$\vec{n}=\frac{\vec{R}}{R}$ $\beta=\frac{v}{c}$

на странице 433 Батыгин Топтыгин интегрируют

$d\vec{F} = - d e_2 d\vec{E} = \frac{de_1 de_2}{c^2 r} \left[ \vec{a} - \vec{r}/r\left( \vec{r}/r \cdot \vec{a}\right) \right]$

и приводят результат 

$m =\frac{1}{{{c}^{^{2}}}}\frac{4}{3}\int\frac{de_1\,de_2}{r}$ (сгс)

то что электродинамический подход решения задачи об электромагнитной массе, основанный на применении потенциалов Лиенара-Вихерта приводит к противоречию, побудило меня обратиться к выводу потенциалов Лиенара-Вихерта 

\section{Критика вывода Фейнмана потенциалов Лиенара Вихерта со стороны Меандра}

надеюсь, параграф 5 главы 21 Фейнмана \cite{Feinman} у Вас есть.

Здесь я считаю нужным привести критику этого параграфа данную Меандром (Сергей Анатольевич Цикра из Донецка) \cite{meandr}
%[quote author=3F37333C3620520 link=1607492586/164#164 date=1610039346][quote]

Фейнман:
Толщина каждого $\Delta V_i$   $\omega_i$, а объем  $\omega a^2$.
Поэтому каждый элемент объема, накладывающийся на распределение заряда, содержит в себе заряд $\omega a^2\rho$, где $\rho$ - плотность заряда внутри кубика (мы считаем ее однородной).


%[/quote] 
Цикра:
Это верно, когда речь идет о РЕАЛЬНОМ распределении заряда, а не о видимом с запаздыванием. Для запаздывающих слоев такой же остается только площадь поперечного сечения $a^2$, но видимая ширина слоев и плотность изменяются (обратно пропорционально, в результате чего количество заряда в каждом слое не изменяется).

%[quote]
Фейнман:
Когда расстояние от заряда до точки (1) велико, то можно все $r_i$ в знаменателях положить равным некоторому среднему значению, скажем, взятому с с учетом запаздывания положению r' центра кубика.


%[/quote]
Цикра:
Это тоже пропущу без принципиальных возражений, хотя более правильным было бы учитывать запаздывающие расстояния до центра каждого слоя. Математически в этом нет никакой проблемы, если известен закон преобразования реального расстояния в запаздывающее, который подразумевается для запаздывающего расстояния до центра всего кубика.

%[quote]
Фейнман:
Сумма (21.30) превращается в 
$$\sum\limits_{i=1}^N \frac{\rho \omega a^2}{r'}$$
$\Delta V_N$ - тот последний элемент $\Delta V_i$, который еще накладывается на распределение зарядов (см.фиг.21.7,д)
Сумма тем самым равна
$$N\frac{\rho \omega a^2}{r'}=\frac{\rho a^3}{r'}\left (\frac{N\omega}{a}\right )$$.

%[/quote]
Цикра:
Это тоже принимаю, с ОБЯЗАТЕЛЬНЫМ замечанием, что дробь в скобках тождественно равна =1.
Число слоев N исходно определялось из длины $а$ и толщины слоя $\omega$ (или наоборот - толщина слоя из длины а и количества слоев N), поэтому в ЭТИХ скобках ПРИНЦИПИАЛЬНО не может получиться ничего кроме 1.

%[quote]

Фейнман:
Но $\rho a^3$ - просто общий заряд, а  $N\omega$ - длина b, показанная на фиг.21.7,д.
Получается
$$\varphi\frac{q}{4\pi\varepsilon_0 r'}\left( \frac{b}{a}\right )$$  .   (21.31)

% [/quote]
Цикра:
НЕТ, НЕ ПОЛУЧАЕТСЯ !
Во-первых, потому что $N\omega$ это реальная длина грани кубика, а видимо-запаздывающая длина b получится с видимо-запаздывающей шириной слоев (при одинаковом их количестве)

 $N\omega'=b$.

Если же кто-то настаивает на одинаковой ширине слоев (например измеряемой по неподвижным маркерам), то должно быть ДРУГОЕ их количество N', с тем же соотношением

$N'\omega=b$.

Во-вторых, применение этой видимо-запаздывающей длины b  ТРЕБУЕТ вернуться  назад и применить во всем предыдущем суммо-интегрировании и видимо-запаздывающую ширину слоев и видимо-запаздывающую плотность, то есть выражения
$$\sum\limits_{i=1}^N \frac{\rho' \omega' a^2}{r'}$$,

$$N\frac{\rho' \omega' a^2}{r'}=\frac{\rho' a^3}{r'}\left (\frac{N\omega'}{a}\right )$$.

Именно теперь выражение в скобках равно коэффициенту пропорциональности (b/a) преобразования реального объема куба в видимо-запаздывающий объем, который будучи помноженным на видимо-запаздывающую плотность (определяемую в обратной пропорции), дает реальную величину заряда, в итоге получается вот ЭТО, без каких-то лиенар-вихертоватостей:

$$\varphi\frac{q}{4\pi\varepsilon_0 r'}$$  .   (21.31')

Это доказывает, что в "выводе" Фейнманом выражения потенциалов Лиенара-Вихерта содержится показанная мной ошибка (в самом конце), и еще раз подтверждает, что знаменитый знаменатель в выражении потенциалов Лиенара-Вихерта не имеет физического смысла (в рамках теорий, использующих принцип независимости скорости возмущения (света) от движения источника).

%[quote author=30383C33392F5D0 link=1607492586/171#171 date=1610118144]
Цикра:
Для того, чтобы покончить с разбором полетов творческой мысли Фейнмана, обращу внимание и на последующий его вираж:

%[quote]
Фейнман:
Наконец, поскольку "размер" а заряда не вошел в окончательный итог,то тот результат получится, если результат стянется до любых размеров, вплоть до точки.

% [/quote]
Цикра:
Фейнман здесь еще раз мягко говоря слукавил, потому что размер "а" ДОЛЖЕН БЫЛ войти в окончательный итог - в неявном виде при интегрировании плотности $\rho$ в общую величину заряда (а в случае применения видимого размера b интегрировать нужно было видимую плотность $\rho'$).
Фокусы с плотностью Фейнман мог показывать ТОЛЬКО с распределенным зарядом, а на точечный заряд они НИКАК не влияют, потому что у него НЕТ размеров, объема и плотности - НЕЧЕМУ преобразовываться, кроме запаздывающего расстояния.

%[/quote]
\section{Что я об этом думаю}

Прежде всего Фейнман исходит из уравнения, 

$$\varphi(1,t) = \frac{1}{4\pi\epsilon_0}\int\frac{\rho(2,t-r_{12}/c)}{r_{12}}dV_{2}$$

которое в свою очередь является решением волнового уравнения Даламбера.

Изобретение Меадра под названием "видимая плотность" заряда сюда не клеится. Плотность не видимая, а настоящая но только лишь взятая  в запаздывающий момент времени.

Думаю, что и Фейнман бы не согласился с утверждением Меандра о том, что "видимая плотность" заряда более рыхлая чем настоящая плотность заряда. Почему я так думаю, потому что Фейнман пишет:

Появился поправочный множитель. Он появился потому что в то время как наш интеграл "проносится над зарядом", сам заряд движется. Когда заряд движется к точке (1), его вклад в интеграл увеличивается в $b/a$ раз.

То есть, поскольку вклад заряда в интеграл увеличивается благодаря его движению, "видимая плотность заряда" по мысли Фейнмана не уменьшается, а остаётся прежней. 

Я со своей стороны тоже позволю себе не согласиться с предположением Меандра о уменьшении видимой плотности заряда. 

То что увеличился видимый объём заряда не означает уменьшения видимой плотности заряда. Плотность заряда не изменяется и объём заряда не изменяется (мы пока не рассматриваем лоренцево сокращение). То что Фейнман пытался показать увеличением видимой длины заряда на мой взгляд связано с спрессовываением поля перед зарядом: не плотность заряда уменьшается вследствие его движения, а плотность поля скалярного потенциала увеличивается перед зарядом (и уменьшается за ним)

%(Более того, если учесть Лоренцево сокращение размеров заряда то плотность заряда даже увеличится, но пока что мы эту модель не рассматриваем.)


\section{Как интеграл "проносится над зарядом"}
%\section{Но я все же попробовал Фейнмана подкорректировать}

Фейнман:
Когда расстояние от заряда до точки (1) велико, то можно все $r_i$ в знаменателях положить равным некоторому среднему значению, скажем, взятому с с учетом запаздывания положению r' центра кубика. Сумма (21.30) превращается в 
$$\sum\limits_{i=1}^N \frac{\rho \omega a^2}{r'}$$
$\Delta V_N$ - тот последний элемент $\Delta V_i$, который еще накладывается на распределение зарядов (см.фиг.21.7,д)
Сумма тем самым равна
$$N\frac{\rho \omega a^2}{r'}=\frac{\rho a^3}{r'}\left (\frac{N\omega}{a}\right )$$.

Мне все же интересно появится ли значимая поправка к интегралу, если всё же не ограничиваться случаем, "Когда расстояние от заряда до точки (1) велико". Интерес этот связан с тем, что в задаче о вычисления силы самодействия заряда на самого себя традиционно используются потенциалы ЛВ, но ведь в этой задаче нельзя сказать, что расстояние от заряда то точки наблюдения велико. Здесь точка наблюдения находится внутри самого заряда.

Поэтому я перепишу сумму в следующем виде

$$\sum\limits_{i=-M}^M \frac{\rho \omega' a^2}{r'_{i}(t'_{i})}=\rho  a^2\sum\limits_{i=-M}^M \frac{\omega'(t'_{i})}{r'_{i}(t'_{i})}= \frac{q}{a}\sum\limits_{i=-M}^M \frac{\omega'(t'_{i})}{r'_{i}(t'_{i})}$$

где $N=2M+1$, таким образом $i=0$ соответствует центральной части заряда. 

По началу я хотел позаимствовать у Меандра формулу взаимосвязи видимой ширины полоски заряда  $\omega'$ с действительной шириной этой полоски  $\omega$

$$\omega' =  \omega \frac{b}{a}$$

Но потом понял, что видимая ширина полоски в том случае если заряд движется с ускорением (а именно этот случай важен в связи с решением задачи о самодействии заряда самого на себя) видимая ширина полоски будет зависеть от скорости заряда в запаздывающий момент, ведь заряд-то ускоряется. 

Фейнман: А чему же равно $b$? Это длина куба зарядов, увеличенное на расстояние, пройденное зарядом от $t_1 = (t-r_1/c)$ до $t_N = (t-r_N/c)$. Это расстояние пройденное зарядом за время 

$$\Delta t = t_N - t_1 = \frac{r_1-r_N}{c}= \frac{b}{c}$$.

А поскольку скорость заряда равна $v$, то пройденное расстояние равно $v \Delta t = v b / c$. Но длина $b$ - само это расстояние плюс $a$:

$$b = a + \frac{v}{c}b$$.

Отсюда 

$$b = \frac{a}{1- (v/c)}$$.

Чуть перефразирую Фейнмана: А чему же равно $\omega'$? Это длина полоски в кубе зарядов, увеличенная на расстояние, пройденное зарядом от $t_{i} = (t-r_{i}/c)$ до $t_{i+1} = (t-r_{i+1}/c)$. Это расстояние пройденное зарядом за время 

$$\Delta t = t_{i+1} - t_{i} = \frac{r_{i+1}-r_{i}}{c}= \frac{\omega'}{c}$$.

А поскольку скорость заряда равна $v$, то пройденное расстояние равно $v \Delta t = v \omega' / c$. Но длина $\omega'$ - само это расстояние плюс $\omega$:

$$\omega' = \omega + \frac{v}{c}\omega'$$.

Отсюда 

$$\omega' = \frac{\omega}{1- (v/c)}$$.

Отмечу здесь, что под скоростью заряда необходимо подразумевать скорость полоски в запаздывающий момент времени

Поэтому

$$ \frac{q}{a}\sum\limits_{i=-M}^M \frac{\omega'(t'_{i})}{r'_{i}(t'_{i})}= \frac{q}{a}\sum\limits_{i=-M}^M \frac{1}{r'_{i}(t'_{i})} \frac{\omega}{1- (v'_{i}(t'_{i})/c)}$$

Далее запаздывающее расстояние от точки наблюдения к $i$-той части заряда

$r'_{i}\left(t'_{i}\right)=x_a-x_i\left(t'_{i}\right)=x_a-\left(x_q\left(t'_{i}\right)+i \omega\right)=r_q\left(t'_{i}\right)-i \omega=c\left(t - t'_{i}\right)$

здесь $x_q\left(t'_{i}\right)$ коондината центральной части заряда в момент $t'_{i}$.


Теперь сумму

$$\frac{q}{a}\sum\limits_{i=-M}^M \frac{\omega }{r'_{q}(t'_{i})-i \omega}\frac{1}{1- (v'_{i}(t'_{i})/c)}$$

% (При желании уточнить данное решение учётом лоренцева сокращения величину $\omega$ нужно будет снабдить фактором Лоренца)

можно переписать в виде интеграла, который собственно и "проносится над зарядом"

$$\frac{q}{a}\int_{-a/2}^{a/2}\frac{1}{1- (v'_{q}(t'(l))/c)} \frac{dl }{r'_{q}(t'(l))-l}$$

При написании этой формулы я неявно подразумевал, что в процессе ускорения заряд не деформируется, поэтому скорость полоски в запаздывающий момент времени $v'_{i}(t'_{i})$ равна скорости заряда в тот же самый запаздывающий момент  $v'_{q}(t'(l))$.

В приближении Фейнмана, когда расстояние между зарядом наблюдателем велико по сравнению с размерами заряда

$$\int_{-a/2}^{a/2}\frac{1}{1- (v'_{q}(t'(l))/c)} \frac{dl }{r'_{q}(t'(l))-l} = \frac{1}{1- (v'_{q}(t')/c)}\frac{a}{r'_{q}(t')}$$

и тогда получается формула Лиенара Вихерта, но я хочу посмотреть изменится ли формула, если расстояние от заряда к наблюдателю сравнимо с размерами самого заряда, что собственно и происходит при решении задачи о самодействии поля ускоряемого заряда на самого себя.

Поэтому мне потребуется решить уравнение
$$r'_{q}(t'(l))-l= c t - c t'(l)$$

относительно $t'(l)$. А именно мне нужно найти зависимость запаздывающего момента от линейной координаты части заряда $l$ над которой "проносится" в запаздывающий момент  $t'_{i}$ интеграл. $l$ равно нулю в центре заряда и изменяется от $-a/2$ до ${a/2}$

$$x_a-x_q\left(t'(l)\right)-l= c t - c t'(l)$$

здесь $x_q\left(t'(l)\right)$ это координата центра заряда в запаздывающий момент $t'(l)$, зависящий от точки заряда, над которой проносится интеграл.

$$ c t'_{i}-x_q\left(t'(l)\right)= c t - x_a+l$$

раскрываю зависимость координаты центра заряда от времени, задавая начальные координату и скорость центра заряда в момент 0 и его ускорение

$$ c t'(l)-x_{q}(0) - v_q t'(l) - a_q {t'(l)}^{2} / 2 = c t - x_a+l$$

$$ c t'(l) - v_q t'(l) - a_q {t'(l)}^{2} / 2 = c t - \left\{x_a - (x_{q}(0) + l)\right\}$$

Пусть скорость заряда в момент ноль $v_q$ равна нулю. Пусть также в момент ноль центр заряда $x_{q}(0)$ будет совмещён с началом координат.

% Пусть также расчёт поля мы будем вести для момента текущего момента $t = 0$

$$ c t'(l) - a_q {t'(l)}^{2} / 2 = c t - \left\{x_a -  l\right\}$$

составляем квадратное уравнение

$$ a_q {t'}^{2} / 2 -c t'  + c t - \left\{x_a -  l\right\} = 0$$

%дискриминант

%$$ D = c^2 + 2  a_q \left\{x_a -  l\right\}$$

решение


$$t' = \frac{c - \sqrt{-2 \, a_{q} c t + c^{2} - 2 \, a_{q} l + 2 \, a_{q} x_{a}}}{a_{q}}$$

подставляя полученное решение в интегранд, имеем

$$\frac{2}{{\left(\frac{{\left(c - \sqrt{-2 \, a_{q} c t + c^{2} - 2 \, a_{q} l + 2 \, a_{q} x_{a}}\right)}^{2}}{a_{q}} + 2 \, l - 2 \, x_{a}\right)} {\left(\frac{c - \sqrt{-2 \, a_{q} c t + c^{2} - 2 \, a_{q} l + 2 \, a_{q} x_{a}}}{c} - 1\right)}}$$

это выражение можно упростить

$$-\frac{a_{q}}{2 \, a_{q} c t - c^{2} + 2 \, a_{q} l - 2 \, a_{q} x_{a} - \sqrt{-2 \, a_{q} c t + c^{2} - 2 \, a_{q} l + 2 \, a_{q} x_{a}} {\left(a_{q} t - c\right)}}$$

неопределённый интеграл имеет вид

$$-\log\left(a_{q} t - c + \sqrt{-2 \, a_{q} c t + c^{2} - 2 \, a_{q} l + 2 \, a_{q} x_{a}}\right)$$

%$$\frac{q \log\left(\frac{a_{q} t - c + \sqrt{-2 \, a_{q} c t + a a_{q} + c^{2} + 2 \, a_{q} x_{a}}}{a_{q} t - c + \sqrt{-2 \, a_{q} c t - a a_{q} + c^{2} + 2 \, a_{q} x_{a}}}\right)}{a}$$


$$\frac{q}{a} \log\left(\frac{a_{q} t - c + \sqrt{-2 \, a_{q} c t + a a_{q} + c^{2} + 2 \, a_{q} x_{a}}}{a_{q} t - c + \sqrt{-2 \, a_{q} c t - a a_{q} + c^{2} + 2 \, a_{q} x_{a}}}\right)$$

\section{Моя попытка вывести потенциал Лиенара Вихерта}

Изоповерхность запаздывающего момента $t'$ есть геометрическое место точек в которых электрический потенциал обусловлен положением заряда в момент $t'$. 


Lля покоящегося заряда густота изолиний запаздывающего момента 

$R'=c(t-t')$

$$\Delta R' = c(t'_{0} -t'_{1})$$

собственно густота изолиний

$$\frac{\Delta t'}{\Delta R'} = \frac{1}{c}$$

но какой в этом физический смысл?

поскольку электрический заряд как известно является источником потока электрического смещения $\vec D$ мы можем посчитать поток электрического смещения через сегмент телесного угла $d \Omega$

плотность потока электрического смещения через сферическую поверхность радиуса $R$ равна $\frac{q}{4\pi R^2}$

Выражаем в сферической системе координат сегмент телесного угла $d \Omega = d \theta  d \phi$

площадь сегмента на сферической поверхности радиуса $R$ равна $d S = R^2 d \phi d \theta$

поток  $\vec D$ через телесный угол $d \Omega$ равен

$$P_{\vec D} = \frac{q}{4\pi R^2} R^2 cos \theta d \phi d \theta = \frac{q}{4\pi} cos \theta d \phi d\theta= \frac{q d\Omega}{4 \pi}$$

плотность потока $\vec D$ равна

$$p_{\vec D} = \frac{q}{4\pi}$$

Давайте здесь воспользуемся гидродинамической аналогией: если имеется плотность потока электрического смещения через некоторую поверхность то умножая ее на $dS dt$  получим выражение для "массы" прошедшей через элемент поверхности $dS$ за время $dt$ потока элетрического смещения

Возможно применительно к потоку электрического смещения такая гидродинамическая аналогия покажется кому-то смешной и совершенно неприменимой, но я могу показать что такая аналогия имеет вполне определенный с точки зрения современной науки физический смысл

Механистическая гидродинамическая аналогия заключается в том что заряд каждую единицу времени как бы испускает электрическое смещение, которое распространяется во всех направлениях со скоростью света, независимо от скорости самого заряда. Покоящийся заряд делает то же самое: потенциал покоящегося заряда запаздывает в той же мере как и потенциал движущегося заряда.

Вообразив, что поток электрического смещения это поток некоторой условной сжимаемой "жидкости ", "газа " или иной "среды ", истекаемой из заряда можно вычислить "массу ", "обьем " или иную меру "количества " этой "среды " электрического смещения находящегося в обьеме пространства, ограниченном изоповерхностями того или иного запаздывающего момента и сектором телесного угла ограниченным линиями перпендикулярными изоповерхностям запаздывающего момента

Вычислим теперь "количество " "среды " электрического смещения для покоящегося заряда в обьеме пространства, ограниченном изоповерхностями запаздывающих моментов  $t'_{1}$ и $t'_{2}$ в секторе телесного угла $d\Omega$.

$$m_{\vec D} = \int\limits_{t'_{1}}^{t'_{2}}\frac{q}{4\pi}d\Omega d t'$$

обьем пространства, ограниченный изоповерхностями запаздывающих моментов  $t'_{1}$ и $t'_{2}$ в секторе телесного угла $d\Omega$.

$$V_{\vec D} = \int\limits_{R'_{1}}^{R'_{2}}{R'}^2 d\Omega d R' $$

%$$R' = c (t-t')$$

$$d R' = c d t'$$

$$V_{\vec D} = \int\limits_{t'_{2}}^{t'_{1}}{R'}^2 d \Omega c d t'$$


"плотность" потока электрического смещения в случае покоящегося заряда

$$\rho_{\vec D} = \frac{q}{4\pi {R'}^2 c} =  \frac{q}{4\pi {R'}^2}\frac{\Delta t'}{\Delta R'} = E\frac{\Delta t'}{\Delta R'}$$

пропорциональна электрическому полю.

Теперь интересен вопрос как вычислить плотность потока электрического смещения для движущегося заряда

Расчёт объёма пространства между изоповерхностями запаздывающего момента.

При вычислении обьема между изоповерхностями запаздывающего момента возникает вопрос: из какой запаздываюшей координаты проводить телесный угол, внутри которого вычислять искомый обьем? При исследовании этого вопроса становится ясно, что использование прямого телесного угла приведет к ошибкам, независимо от того из какой точки этот телесный угол, проводить, ибо заряд находится в движении. 

Решение заключается в том, чтобы ограничить телесный угол искривленными поверхностями такими, чтобы поверхность телесного обьема была нормальна изоповерхности текущего запаздывающего момента

Для построения оных прибегнем к следующим соображениям 

Отметим на изоповерхности запаздывающего момента $t'=0$ в сферической системе отсчета, связанной с запаздывающей координатой некую точку с координатами $(\theta, \phi)$.

При изменении запаздывающего момента на малую величину запаздывания координата смещается на величину $v dt'$

На ту же величину смещается центр сферической системы координат, связанный уже с новой запаздывающей координатой.

Радиус изоповерхности изменяется от $c(t-t'_{0})$ до $c(t-(t'_{0} + d t'))$

Отметим на смещенной вместе с запаздывающей координатой сферической системе координат точку с теми же координатами $(\theta, \phi)$

При последовательном изменении запаздывающего момента от $t'=0$ до $t'=t$ точка с координатами $\theta, \phi$ на сферической изоповерхности запаздывающего момента опишет некоторую кривую, нормальную изоповерхностям запаздывающего момента



С уважением, Алексей

\begin{thebibliography}{99}

%\bibitem{LL2}
%\textit{Ландау Л.Д. Лившиц Е.М. Теория поля. М. 1973}

%\bibitem{rustot}
%\textit{Re: Как запаздывающий Лиенар-Вихерт становится "незапаздывающим". Визуализация}
%\\\texttt{http://www.sciteclibrary.ru/cgi-bin/yabb2/YaBB.pl?num=1528093569/330\#330}

\bibitem{Feinman}
\textit{Фейнмановские лекции по физике т.6. Электродинамика М. 1966, стр. 158}

\bibitem{meandr}
\textit{Re: Интервал запаздывания и запаздывающее время.}
\\\texttt{http://www.sciteclibrary.ru/cgi-bin/yabb2/YaBB.pl?num=1607492586/164\#164}

%\bibitem{tamm}
%\textit{И.Е.Тамм. Основы теории электричества. М. 1957}

%\bibitem{flugge}
%\textit{З.Флюгге Задачи по квантовой механике т.2 М. "Мир" 1974. стр. 296}

%\bibitem{misyuchenko}
%\textit{В. Ганкин, Ю. Ганкин, О. Куприянова, И. Мисюченко. История электромагнитной массы}
%\\\texttt{http://fphysics.com/d/232484/d/istoriya_em_massy1.pdf}


%\bibitem{Haddad}
%S. Haddad and S. Suleiman
%\textit{NEUTRON CHARGE DISTRIBUTION AND CHARGE DENSITY DISTRIBUTIONS IN LEAD ISOTOPES}
%\textit{ACTA PHYSICA POLONICA B, Vol. 30 (1999) No 1}
%\\\texttt{http://www.actaphys.uj.edu.pl/fulltext?series=Reg\&vol=30\&page=119}



\end{thebibliography}



\end{document}

