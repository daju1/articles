\documentclass{article}
\usepackage[T2A]{fontenc}
\usepackage[utf8]{inputenc}
\usepackage[english, russian]{babel}

\begin{document}

\title{Юрию Петровичу Степановскому о проблеме 4/3}

\author{А.Ю.Дроздов}

%\date{Jan 30, 2020}


\begin{titlepage}
\maketitle
\end{titlepage}

%========================================================================

\section{Юрию Петровичу Степановскому о проблеме 4/3}

Добрый день Юрий Петрович

Вы пишете: Вы почему-то все время находите авторов, которые решают давно решенные, или несуществующие проблемы. Проблема 4/3 давно решена, причем решил известный физик-теоретик, а не люди, которые читают, в основном, одну Википедию



Да. Эти авторы Фейнман и Батыгин, Топтыгин. Если Вы мне дадите ссылку на физика-теоретика, решившего проблему 4/3, то мне это будет очень интересно.

Я потому и обращаюсь к Вам потому что Вы знаете больше, чем Википедия.

А началось все с того, что я начал разбирать решение задачу номер 689 из Батыгина Топтыгина (издание 1962 года)


Найти силу $\vec{F}$, с которой заряженная симметричная частица действует сама на себя (сила самодействия) при ускоренном поступательном движении с малой скоростью $v << c$. Запаздывание и лоренцево сокращение не учитывать.
Указание. Вычислить равнодействующую сил, приложенных к малым элементам $de$ заряда частицы, воспользовавшись лиенар-вихертовым выражением напряжённости поля (XII. 25)  

Под этим выражением Батыгин Топтыгин приводят выражение

$\vec{E}\left(\vec{r},t\right) = e \frac{\left(1-\beta^2\right) \left(\vec{n}-\vec{v}/c \right)}{\left(1-\vec{n}\cdot\vec{v}/c\right)^3\,R^2} + e \frac{ \vec{n} \times \left[ \left( \vec{n}-\vec{v}/c\right) \times \vec{a}\right]}{c^2 \left(1-\vec{n} \cdot \vec{v}/c \right)^3 \,R }$

$\vec{n}=\frac{\vec{R}}{R}$ $\beta=\frac{v}{c}$

на странице 433 Батыгин Топтыгин интегрируют

$d\vec{F} = - d e_2 d\vec{E} = \frac{de_1 de_2}{c^2 r} \left[ \vec{a} - \vec{r}/r\left( \vec{r}/r \cdot \vec{a}\right) \right]$

и приводят результат 

$m =\frac{1}{{{c}^{^{2}}}}\frac{4}{3}\int\frac{de_1\,de_2}{r}$ (сгс)

Сначала я попытался сам решить эту задачу, пришёл к тому же интегралу, что и Батыгин Топтыгын, но когда я своими силами попытался его проинтегрировать то 4/3 я не получил. Мой ответ, если я конечно не ошибся : 14/25 но никак не 4/3. 

За разъяснением ситуации я обратился к Константину Алексеевичу Чишко. Он ответил, что скорее всего Батыгин и Топтыгин эту задачу сами не решали, а решение у кого-то списали. 


Тогда я обратился к одному из ортодокальных теоретиков форума 
http://www.sciteclibrary.ru 
под ником rustot. Дело в том что он на форуме приводил вывод электрического поля исходя из потенциалов Лиенара - Вихерта. 

Вот выражение Рустота из темы Дробышева, как запаздывающий Лиенар-Вихерт становится незапаздывающим \cite{rustot}.

$\vec{E} = \frac{q}{k^2}((-\frac{\vec{r}}{c k})(-c - \frac{\vec{r}\vec{a}}{c} + \frac{v^2}{c})-\frac{\vec{v}}{c}+\frac{\vec{v}}{c^2}(\frac{r}{k}(-c - \frac{\vec{r}\vec{a}}{c} + \frac{v^2}{c}) + c) - \frac{\vec{a}r}{c^2})$

$\vec{E} = \frac{q}{k^3}((\vec{r}-\frac{r}{c}\vec{v})(1 + \frac{\vec{r}\vec{a}}{c^2} - \frac{v^2}{c^2}) - \vec{a}\frac{kr}{c^2})$

где $k= r-\frac{\overrightarrow{v}\cdot \overrightarrow{r}}{c} $ - радиус Лиенара Вихерта.

И вот я задал ему Вопрос 

можете ли Вы показать, каким методом Батыгин и Топтыгин вычисляют этот интеграл, или если они уже готовое решение переписали, то откуда?

Кто этот интеграл вычислял.

Его ответ был, я этого не решал, не знаю. готовые решения можно попытаться найти по "электромагнитная масса"


Ну вот С ходу нашлась вот эта работа, http://nfp-team.narod.ru/EM43.pdf

В ней Мисюченко утверждает, что он решил проблему 4/3, но это другая задача, Это интеграл от вектора Пойтинга движущегося заряда в пространстве вовне, она есть не только у Фейнмана , но и у Батыгина-Топтыгина, в моём издании номер 687. 

Этот интеграл проще тем что он тройной (при учете симметрии по углу фи двойной)

А в задаче про самодействии частицы, интеграл шестикратный и по объёму частицы  (при учете симметрии по углу фи четырёхкратный).

Я попытался сам проинтегрировать но у меня никак 4/3 не получается, возможно я не правильно действую, я воспользовался разложением обратного радиус вектора по сферическим гармоникам который прифодится во втором томе Флюге на стр. 300

${{R}_{0}}=\left|\overrightarrow{r_{q}} - \overrightarrow{r_{a}}\right|$ находится в знаменателе, то в сферической системе координат можно применить разложение по сферическим гармоникам следующего вида  \cite{flugge} если $\left( {{r}_{q}}<{{r}_{a}} \right)$ то

$\frac{1}{\left| \overrightarrow{{{r}_{q}}}-\overrightarrow{{{r}_{a}}} \right|}=\frac{1}{{{r}_{a}}}\sum\limits_{l=0}^{\infty }{{{\left( \frac{{{r}_{q}}}{{{r}_{a}}} \right)}^{l}}{{P}_{l}} \cos \left( \gamma  \right)}$

и если $\left( {{r}_{a}}<{{r}_{q}} \right)$ то

$\frac{1}{\left| \overrightarrow{{{r}_{q}}}-\overrightarrow{{{r}_{a}}} \right|}=\frac{1}{{{r}_{q}}}\sum\limits_{l=0}^{\infty }{{{\left( \frac{{{r}_{a}}}{{{r}_{q}}} \right)}^{l}}{{P}_{l}} \cos \left( \gamma  \right)}$

В данной формуле ${{P}_{l}} \cos \left( \gamma  \right)$ это полиномы Лежандра аргумент которых $\gamma$ есть угол между векторами ${{r}_{q}}$  и ${{r}_{a}}$. Применяя формулу, известную как теорему сложения

${{P}_{l}}\cos \left( \gamma  \right)=\frac{4\pi }{2l+1}\sum\limits_{m=-l}^{l}{Y_{l,m}^{*}\left( {{\theta }_{a}},{{\varphi }_{a}} \right)}\ {{Y}_{l,m}}\left( {{\theta }_{q}},{{\varphi }_{q}} \right)$


вот такое разложение. Но проблема в том, что мне нужно в третью степень обратный радиус в знаменателе возвести.

Может быть Вы знаете, как разложить $1/R^3$
?

От Рустота я ответа до сих пор не получил, теперь надежда на то что Вы, Юрий Петрович, дадите мне ссылку на физика-теоретика. 

А то у меня закрадывается сомнение, что ответ 4/3 в задаче про самодействие частицы просто подогнали под решение задачи Фейнмана об интеграле вектора Пойтинга

С уважением, Алексей

\begin{thebibliography}{99}

%\bibitem{LL2}
%\textit{Ландау Л.Д. Лившиц Е.М. Теория поля. М. 1973}

\bibitem{rustot}
\textit{Re: Как запаздывающий Лиенар-Вихерт становится "незапаздывающим". Визуализация}
\\\texttt{http://www.sciteclibrary.ru/cgi-bin/yabb2/YaBB.pl?num=1528093569/330\#330}

%\bibitem{tamm}
%\textit{И.Е.Тамм. Основы теории электричества. М. 1957}

\bibitem{flugge}
\textit{З.Флюгге Задачи по квантовой механике т.2 М. "Мир" 1974. стр. 296}

%\bibitem{misyuchenko}
%\textit{В. Ганкин, Ю. Ганкин, О. Куприянова, И. Мисюченко. История электромагнитной массы}
%\\\texttt{http://fphysics.com/d/232484/d/istoriya_em_massy1.pdf}


%\bibitem{Haddad}
%S. Haddad and S. Suleiman
%\textit{NEUTRON CHARGE DISTRIBUTION AND CHARGE DENSITY DISTRIBUTIONS IN LEAD ISOTOPES}
%\textit{ACTA PHYSICA POLONICA B, Vol. 30 (1999) No 1}
%\\\texttt{http://www.actaphys.uj.edu.pl/fulltext?series=Reg\&vol=30\&page=119}



\end{thebibliography}



\end{document}

