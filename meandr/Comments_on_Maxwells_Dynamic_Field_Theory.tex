\documentclass{article}
\usepackage[T2A]{fontenc}
\usepackage[utf8]{inputenc}
\usepackage[english, russian]{babel}

\begin{document}

\title{Комментарии к Динамической теории поля Максвелла в изложении Цикры Сергея  Анатольевича}

\author{А.Ю.Дроздов}

\date{Mar 16, 2021}


\begin{titlepage}
\maketitle
\end{titlepage}

%========================================================================

\section{Интерпретация А как удельного импульса самого ПОЛЯ (скалярного потенциала) как движущейся среды}

Объяснение явления намагниченности магнитным полем вращающегося заряда
С.А.Цикра:

Автор данной статьи придерживается оригинальной Динамической Теории Электромагнитного Поля Максвелла (далее ДТПМ) [2,3], в соответствии с которой  в.м.п. А является импульсом электрического поля:

$\vec{A} = \varphi\,\vec{v}/c^2$

А.Ю.Дроздов : Чем Ваша(или Максвелла)  трактовка отличается от официальной теории
https://ru.wikipedia.org/wiki/Импульс

Обобщённый импульс в электромагнитном поле
В электромагнитном поле обобщённый импульс частицы равен:

${\mathbf  {p}}={\frac  {m{\mathbf  {v}}}{{\sqrt  {1-v^{2}/c^{2}}}}}+q{\mathbf  A}$




С.А.Цикра: 1. Там А соответствует импульсу ЧАСТИЦЫ в ЭМ поле, а у меня А соответствует удельному импульсу самого ПОЛЯ (скалярного потенциала) как движущейся среды
%http://www.sciteclibrary.ru/cgi-bin/yabb2/YaBB.pl?num=1416716685/74#74
В общепринятой теории ЭМ поле не может двигаться (перемещаться), может только изменяться
%http://www.sciteclibrary.ru/cgi-bin/yabb2/YaBB.pl?num=1534518253/92#92
Импульс ЭМ поля в общепринятой теории - это НЕ вмп А, а по факту вектор Пойтинга.
С моей стороны вмп А это импульс удельный не по объему пространства, а по величине пробного заряда в этом поле.



\section{Re: Динамическая теория Максвелла Ответ 74  Движение поля}
%[quote author=5F57535C5640320 link=1416716686/74#74 date=1417153357][quote author=7742555348514E444F73270 link=1416716685/0#0 date=1416716685]Уже много раз мы возвращаемся к этой теме, но вопросы все еще остаются. [/quote]
Уже много раз  на вопросы, которые тут в очередной раз "всплывают", я давал  ответы именно с позиций ДТПМ - Динамической теории поля Максвелла.
Не удивительно, что эти ответы тонут в темной серо-коричневой массе присутствующих - там тонет все, отличающееся по цвету, запаху и качеству.
Но Петрович не из этой массы, а из небольшой когорты особо извращенных ортодоксальных садистов-иллюзионистов, способных незаметно извратить все до неузнаваемости, сделать из конструктивной рациональной идеи неконструктивную иррациональную.

Вот последний пример этому:
%[quote author=152037312A332C262D11450 link=1416716685/70#70 date=1417143177]
Теория Максвелла имела бы завершенный вид, если бы Сергей действительно рассмотрел гипотетический случай
[b]V=A/m[/b].
Существование такого поля скорости следует, например, из теории электронов Лоренца, которую Максвелл еще не знал.  Но Сергей знает и может изучать такого рода поля. %[/quote]
Теория Максвелла имела бы завершенный вид, если бы свое определения векторного магнитного потенциала как ИМПУЛЬСА ПОЛЯ, неоднократно  однозначно используемое в т.2 Трактата, Максвелл вырази бы формулой
[b]A[/b]=[b]v[/b]*fi/c[sup]2[/sup]
которая действительно появилась значительно позже у Лоренца, но мимоходом, как побочный продукт в ходе его преобразований, когда импульс ЭМ поля уже принято было считать через вектор Пойтинга.
В этом уравнении ясно видно, что А является импульсом поля скалярного потенциал фи, переводимого к от энергии к "массе" коэффициентом c[sup]2[/sup]. 

Определение вмп А как импульса ПОЛЯ - СРЕДЫ, ответственной за все электромагнитные явления, Максвелл дал еще за 9 лет до Трактата в работе, которая так и назвалась - Динамическая теория электромагнитного поля.
Но Максвелл исподволь шел к этому понимания вмп как импульса движущейся электрической среды еще раньше - в двух предшествующих работах о силовых линиях, в которых серые массы "последователей" сейчас не видят ничего, кроме упоминаний эфира с "шестеренками". Еще тогда он писал о вмп А как о кинетической характеристике "электротонического" состояния электрической среды.
А на роль  этой среды не подходит ничего, кроме давно известного кулоновского скалярного потенциала зарядов.

Серые массы этого у упор не понимают, а извращенный Петрович постоянно "забывает", чтобы пропихивать свои довольно тривиальные извращения-перевертыши:
[quote]Существование такого поля скорости следует, например, из теории электронов Лоренца, которую Максвелл еще не знал. [/quote]
Прежде всего нужно говорить не о поле скорости, а о СКОРОСТИ ПОЛЯ - объективной материальной движущейся среды, ответственной за электромагнитные явления.
Если не будет такой физической среды, то "поле скорости" не к чему прицепить.
А на роль  этой среды не подходит ничего, кроме давно известного кулоновского скалярного потенциала зарядов.
И я в таком понимании лишь последнее звено в той цепи, начало которой положил Максвелл, а Хевисайд с Лоренцом ее разорвали.

Но это еще не все. Нужно, чтобы поле как среда не просто двигалось, а двигалось в неразрывной связи с его источниками.
В теории Хевисайда-Лоренца это обеспечивается ВОЛНАМИ, расходящимися в ОБЩЕЙ среде от тел-источников ВОЗМУЩЕНИЯ поля. В этой трактовке "скорость поля" есть просто скорость движения волн ЭМ возмущений - скорость света, ни больше ни меньше - ЕДИНСТВЕННАЯ возможная в этой трактовке скорость.

А в ДТПМ каждый источник имеет СВОЕ ПОЛЕ, которое движется вместе с ним  с ТОЙ же скоростью, если движение равномерное, инерционное, НЕвозмущенное.
И только в случае ускоренного движения источника в его поле возникают волновые возмущения-деформации. Но и в этом случае скорость поля может быть конкретно определена.

То есть теория Максвелла изначально дает средства для описания и квазистатических процессов, и процессов динамически ускоренных, тем не менее НЕ сводя их к "общему" случаю "произвольного движения", как это делает теория Хевисайда-Лоренца-Эйнштейна, подразумевая именно ускоренное движение с волновыми деформациями поля.

Это тоже ПРИНЦИПИАЛЬНАЯ разница двух теорий,о которой я уже устал писать.

%[quote author=152037312A332C262D11450 link=1416716685/71#71 date=1417145120]
Но в современной физике ищут релятивистские эффекты, поэтому изучают участки цепей с незамкнутыми токами, когда влияние неопределенного в стандартной модели слагаемого $\varphi=\psi+\bf v\bf A$ максимально велико. %[/quote] 
Именно Хевисайд с ЛОренцом ответственны за то, что теперь в стандартной модели потенциалы НЕопределены не только до постоянной константы, но до любой функции во времени, а конвективный квазипотенуиал [b]vA[/b] который явно был у Максвелла, не входит в явном виде в уравнения электродинамики, а вводится лишь как скрытый элемент Лоренцева преобразования полей или как одно составляющее 4-х потенциала. 

 
 % [/quote]

\section{Вопросы по поводу импульса поля в динамической теории Максвелла"}
А.Ю.Дроздов : 

 Если А соответствует удельному импульсу самого ПОЛЯ (скалярного потенциала) как движущейся среды

$\vec{A} = \varphi\,\vec{v}/c^2$

тогда разделив на скорость мы получим удельную "массу" поля

$m = \varphi/c^2$ 

Смотрите я уже решал задачу вычисления инертной электромагнитной массы заряженной частицы. Формулировка моей задачи идентична задаче номер 689 из Сборника задач по электродинамике Батыгина Топтыгина издания 1962 года

Я решал задачу в следующей постановке

Пусть частица с заданным распределением объёмной плотности электрического заряда $\rho \left( r \right)$ приобретает ускорение $\overrightarrow{a}$ . Найти силу, действующую на распределённый в объёме электрический заряд этой частицы со стороны электрического поля самоиндукции.

А Батыгин и Топтыгин в постановке

Найти силу $\vec{F}$, с которой заряженная симметричная частица действует сама на себя (сила самодействия) при ускоренном поступательном движении с малой скоростью $v << c$. Запаздывание и лоренцево сокращение не учитывать.
Указание. Вычислить равнодействующую сил, приложенных к малым элементам $de$ заряда частицы, воспользовавшись лиенар-вихертовым выражением напряжённости поля (XII. 25)  

Под этим выражением Батыгин Топтыгин приводят выражение

$\vec{E}\left(\vec{r},t\right) = e \frac{\left(1-\beta^2\right) \left(\vec{n}-\vec{v}/c \right)}{\left(1-\vec{n}\cdot\vec{v}/c\right)^3\,R^2} + e \frac{ \vec{n} \times \left[ \left( \vec{n}-\vec{v}/c\right) \times \vec{a}\right]}{c^2 \left(1-\vec{n} \cdot \vec{v}/c \right)^3 \,R }$

$\vec{n}=\frac{\vec{R}}{R}$ $\beta=\frac{v}{c}$

которое судя по всему  просто переписано из ЛЛ2, то которое "не представляет труда вычислить".





Для двойного векторного произведения справедлива формула Лагранжа  («бац минус цаб»):

${\vec {a}}\times \big({\vec {b}}\times {\vec {c}}\big ) = {\vec {b}} \big( {\vec {a}} \cdot {\vec {c}}\big) - {\vec {c}} \big( {\vec {a}} \cdot {\vec {b}}\big)$

$a = \vec{n}$ $b =  \left( \vec{n}-\vec{v}/c\right)$ $c =  \vec{a}$


$\vec{E}\left(\vec{r},t\right) = e \frac{\left(1-\beta^2\right) \left(\vec{n}-\vec{v}/c \right)}{\left(1-\vec{n}\cdot\vec{v}/c\right)^3\,R^2} +  e \frac{ \left( \vec{n}-\vec{v}/c\right) \left( \vec{n} \cdot \vec{a} \right) - \vec{a} \left( \vec{n} \cdot \left( \vec{n}-\vec{v}/c\right)\right)}{c^2 \left(1-\vec{n} \cdot \vec{v}/c \right)^3 \,R }$


Я же использовал выражение Рустота из темы Дробышева, как запаздывающий Лиенар-Вихерт становится незапаздывающим.

$\vec{E} = \frac{q}{k^2}((-\frac{\vec{r}}{c k})(-c - \frac{\vec{r}\vec{a}}{c} + \frac{v^2}{c})-\frac{\vec{v}}{c}+\frac{\vec{v}}{c^2}(\frac{r}{k}(-c - \frac{\vec{r}\vec{a}}{c} + \frac{v^2}{c}) + c) - \frac{\vec{a}r}{c^2})$

$\vec{E} = \frac{q}{k^3}((\vec{r}-\frac{r}{c}\vec{v})(1 + \frac{\vec{r}\vec{a}}{c^2} - \frac{v^2}{c^2}) - \vec{a}\frac{kr}{c^2})$


Батыгин Топтыгин интегрируют

$d\vec{F} = - d e_2 d\vec{E} = \frac{de_1 de_2}{c^2 r} \left[ \vec{a} - \vec{r}/r\left( \vec{r}/r \cdot \vec{a}\right) \right]$


 приводят результат 



$m =\frac{1}{{{c}^{^{2}}}}\frac{4}{3}\int\frac{de_1\,de_2}{r}$ (сгс)

Я интегрирую

Градиентное электрическое поле ($z$ компонента)
	\[{E}_{1}=\\
\int\limits_{{{r}_{q}}}\int\limits_{{{\varphi}_{q}}}\int\limits_{{{\theta}_{q}}}\\
\left\{ \frac{z_a-z_q}{{{R}_{0}}}\left( 1+\frac{a\left( {{z}_{a}}-{{z}_{q}} \right)}{c^2} \right) \\
 \right\}\\
\frac{\rho \left( {{r}_{q}} \right){{r}_{q}}^{2}\sin \left( {{\theta }_{q}} \right)}{{{R}_{0}}^{2}}\\
d{{\theta }_{q}}d{{\varphi }_{q}}d{{r}_{q}}\] 	

Электрическое поле самоиндукции ($z$ компонента)
\[{E}_{2}=\\
\int\limits_{{{r}_{q}}}\int\limits_{{{\varphi}_{q}}}\int\limits_{{{\theta}_{q}}}\\
{\left\{ -\frac{{a_z}R_{0}}{{{c}^{2}}} \right\}\\
\frac{\rho \left( {{r}_{q}} \right){{r}_{q}}^{2}\sin \left( {{\theta }_{q}} \right)}{{{R}_{0}}^{2}}\ }d{{\theta }_{q}}d{{\varphi }_{q}}d{{r}_{q}}\]
где ${R}_{0}$ расстояние от точки источника заряда к точке наблюдения без учёта запаздывания.


${{F}_{1}}=\frac{\overrightarrow{a}}{{{c}^{^{2}}}}\int\limits_{{{V}_{a}}}{\int\limits_{{{V}_{q}}}{\left( {{z}_{a}}-{{z}_{q}} \right)^2\frac{\rho \left( {{r}_{q}} \right)\rho \left( {{r}_{a}} \right)}{R_{0}^3}}}\ d{{V}_{q}}d{{V}_{a}}$


${{F}_{2}}=-\frac{\overrightarrow{a}}{{{c}^{^{2}}}}\int\limits_{{{V}_{a}}}{\int\limits_{{{V}_{q}}}{\frac{\rho \left( {{r}_{q}} \right)\rho \left( {{r}_{a}} \right)}{R_{0}}}}\ d{{V}_{q}}d{{V}_{a}}$


Мой результат следующий

для электромагнитной массы я получил выражение
$m=\frac{1}{{{c}^{^{2}}}}\int\limits_{{{V}_{a}}}{\int\limits_{{{V}_{q}}}{\frac{\rho \left( {{r}_{q}} \right)\rho \left( {{r}_{a}} \right)}{R_{0}}}}\ d{{V}_{q}}d{{V}_{a}}$ (в системе сгс) и соответственно $m=\frac{{{\mu }_{0}}}{4\pi }\int\limits_{{{V}_{a}}}{\int\limits_{{{V}_{q}}}{\frac{\rho \left( {{r}_{q}} \right)\rho \left( {{r}_{a}} \right)}{R_{0}}}}\ d{{V}_{q}}d{{V}_{a}}$ в системе СИ

Тут опять мне нужно ещё разобраться почему я не получил известную проблему $\frac{4}{3}$ а Батыгин и Топтыгин получили (и кстати, Фейнман тоже получил $\frac{4}{3}$).

Но сейчас я хочу обратить внимание на размерность. 

Ваша размерность массы поля $m = \varphi/c^2 =\frac{1}{{{c}^{^{2}}}}\,\int\frac{de}{r}$ (сгс)

Вы правда пишете, что 

"С моей стороны вмп А это импульс удельный не по объему пространства, а по величине пробного заряда в этом поле." 

Наверное это Ваше замечание как раз и вызвано проблемами размерности.

Но тогда:

1. чем все же отличается Ваше определение А как удельного импульса самого ПОЛЯ от официального Обобщённого импульса ЧАСТИЦЫ в электромагнитном поле? - Вы ведь фразой "удельный по величине пробного заряда в этом поле" фактически умножаете на заряд пробной частицы, и по сути у Вас тогда формула та же, что и официальный обобщённый импульс

2. как все же в Вашей теории посчитать импульс поля движущегося заряженного шара? Ведь в официальной теории через вектор Пойтинга импульс поля движущегося заряженного шара считается без привлечения пробного заряда (Батыгин Топтыгин, задача 687, а также Фейнман)

А в Вашей теории я не вижу как без привлечения заряда пробной частицы (или заряда всего пространства - сейчас это я уже выдумал бог весть что в попытке удовлетворить требованиям размерности)  посчитать импульс поля движущегося заряда?

Подышал свежим воздухом и понял, что есть два пути, как посчитать импульс поля заряда, придерживаясь концепции ДТПМ: 

Первый путь. В качестве пробного заряда использовать сам заряд источник поля. Тогда выражаясь языком официальной теории это будет обобщённый импульс частицы в поле собственного векторного потенциала.

А выражаясь Вышим языком это будет интегрирование  векторного магнитного потенциала А как удельного импульса  не только  по объему пространства, но по величине самого заряда источника поля в своём собственном поле. 

С таким подходом мы получим для электромагнитной массы формулы идентичные моим формулам.

Второй путь. Я слушал Ваш семинар 2014 года и понял Вашу мысль. Вы говорили о том что, поскольку

Цитирую Вас: Потенциал не мой, и даже не Максвелла - теория потенциала была развита еще до него в теории дальнодействия. Максвелл лишь придал потенциалу физический смысл в теории ближнедействующих полей. Еще в статье "О физических силовых линиях" потенциал фи появился в уравнении (77) и Максвелл дал ему первое определение:
Цитировать
Физическая интерпретация фи заключается в том, что эта величина определяет электрическое натяжение в каждой точке пространства


то, поле скалярного потенциала обладает плотностью потенциальной энергии натяжения, и поскольку Вы из релятивизма принимаете формулу $W = m\,c^2$, постольку поле скалярного потенциала обладает как бы аналогом массы по формуле $m = \frac{W}{c^2}$, где W - это плотность энергии электрического наряжения поля.
А поскольку в Вашей теории потенциал движется тогда векторный потенциал $A = \frac{\varphi\,\vec{v}}{c^2}$, по Вашему суть плотность импульса.

Но я с этим не согласен, потому что в этих Ваших рассуждениях скрыта следующая неточность: 
плотность потенциальной энергии натяжения нужно выражать через тензор натяжения Максвелла, а тензор натяжения выразится через заряд источник поля во второй степени а не через заряд в первой степени.
В простейшем случае покоящегося сферического заряда тензор натяжения выразится как $\sigma = \frac{E^2}{8\,\pi} =\frac{\left(grad\,\varphi\right)^2}{8\,\pi}=\frac{q^2}{8\,\pi\,r^4}$ (сгс)

Следовательно векторный потенциал не может быть плотностью импульса поля в буквальном смысле этого слова, потому что векторный потенциал выражается через заряд в первой степени, а импульс поля - через заряд во второй степени.


\section{Вопросы по работе "Объяснение явления намагниченности магнитным полем вращающегося заряда"}
А.Ю.Дроздов : 
как будет выглядеть поле векторного потенциала вращающегося заряженного шара (или шара заряд которого расположен на его поверхности, если посчитать его по ортодоксальной формуле для векторного потенциала из магнитостатики, а также если посчитать его по Лиенару Вихерту. Если знаете где найти готовое решение хорошо, если нет — я бы сам это с удовольствием посчитал

Исходя из Вашей и Максвелла трактовки, что векторный потенциал суть импульс поля можете показать как посчитать импульс поля движущегося заряженного шара

Тоже как правильно в Вашей теории считать момент количества движения поля вращающегося заряженного шара

Я сегодня (уже вчера) прочитал Фейнмановские лекции том 6 параграф 2 механическая и электрическая энергия, в которой Фейнман разбирал движение рамки с током в магнитном поле. Фейнман приходит к выводу, что механическая работа перемещения рамки с током в магнитном поле в точности равна электрической работе, производимой над источником тока токовой петли. Заменяем в этих рассуждениях токовую петлю вращающимся заряженным шаром (классическим), и помещаем этот вращающийся заряженный шар в неоднородное магнитное поле. Поскольку есть градиент скалярного произведения магнитного момента вращающегося классического заряда на индукцию внешнего магнитного поля, то при перемещении этого шара в неоднородном магнитном поле будет совершаться механическая работа. Но одновременно также будет изменяться скорость вращения этого шара за счёт изменения электрической энергии движущегося заряженного шара в неоднородном магнитном поле. Ну это понятно до тех пор, пока заряженный шар классический. А если это спин? Спин же такая штука строго квантованная. Не будет значит спин ускоряться или замедляться (в том смысле что не должен меняться спиновый момент количества движения). А что будет происходить  при движении спина в таком магнитном поле у которого направление поля и его градиент взаимно перпендикулярны, как во втором параграфе Фейнмана?

[ 15 марта 2021 г. 19:58 ] Только что за ужином пришла идея. Я вообще придерживаюсь представления о том, что частицы суть солитонные волны в среде физического вакуума, которые получаются благодаря тому что настоящие уравнения все же нелинейны на микромасштабах. Квантование спина это проявление волновых свойств частиц. Иными словами мировая среда, или физический вакуум не умеет колебаться по другому, не квантуясь, всегда где-то возникнет или число длин волн или какое нибудь ещё квантовое число. Теперь рассмотрим опыт, который как бы немножко похож на опыт Штерна Герлаха, но отличается конфигурацией магнитов. В опыте Штерна Герлаха направление градиента магнитного поля и направление самого магнитного поля одинаковы. Я предлагаю такую модификацию этого опыта, чтобы направление градиента магнитного поля было перпендикулярно направлению магнитного поля. Практически это можно осуществить намагнитив цилиндр поперёк оси цилиндра. Тогда поле между двумя такими цилиндрами будет отвечать требованию перпендикулярности поля и его градиента. Согласно известному свойству спина ориентироваться вдоль или поперёк внешнего магнитного поля часть атомов будут втягиваться в область максимального поля, а часть будут выталкиваться. Согласно Фейнману электрическая энергия должна либо поглощаться либо выделяться в источнике тока для поддершания тока в токовой петле постоянным (при движении токовой петли из области большего поля в область меньшего или наоборот). Тогда возникает вопрос, а что есть тогда такой "источник тока" для спина, поддерживающий значение спинового момента количества движения постоянным? Получается что сама среда физического вакуума и есть этот "источник тока", уж коли эта среда не умеет по другому колебаться как только лишь квантуясь. Получается что одни атомы будут отбирать недостающую для поддержания постоянного спинового момента энергию из среды физического вакуума, а другие, наоборот возвращать в эту среду свой излишек. Ну вот такое мое объяснение. Что Вы думаете?



\section{Re: Специальная Теория "meandr". Ответ 59 - 20.03.16 :: 11:57:57}

2. У меня есть одно существенное отличие от тех уравнений, которые Максвелл записал в Трактате, в частности  в наиболее важном п. 601 в уравнении напряженности Е :
у меня коэффициент 1/2 при конвективном потенциале vA как положено при кинетической энергии как интеграле импульса A=v*ф по скорости (в этом случае потенциалы НЕ калибруются). Я пришел к этому в 2016 г.
%http://www.sciteclibrary.ru/cgi-bin/yabb2/YaBB.pl?num=1450988497/59#59
У Максвелла этого коэффициента нет (вернее он равен 1), поэтому его уравнения калибруются по потенциалам и приводятся к общепринятому ныне виду - но все это правильно работает ТОЛЬКО в релятивистском 4х пространстве, а не в классическом, как у меня.
Именно из-за этого отличия я теперь называю свою теорию не ДТПМ (ТДПМ), а зорб-теория.

			Попытка №5

На основе идей классической теории Максвелла, применившего принципы обобщенной механики Лагранжа-Гамильтона и теорию потенциалов, определяю силы, действующие между любыми двумя зарядами, движущимися относительно друг друга равномерно и прямолинейно (инерционно), выражая их только с применением относительных скоростей зарядов и полей скалярного потенциала, инвариантных в любой ИСО.

Для i-го заряда в окружении j-х зарядов Сила, действующая на заряд, выражается соотношением 

$\vec F_i=-q_i\left [\sum_j\nabla\varphi_j+\sum_j\nabla((\vec v_j-\vec v_i)^2\varphi_j/2c^2))+\frac{1}{c^2}\sum_j\frac{d_i}{dt}((\vec v_j-\vec v_i)\varphi_j)\right ]$

$\vec F_i=-q_i\left [\sum_j\nabla\varphi_j+\sum_j\nabla((\vec v_j-\vec v_i)^2\varphi_j/2c^2))+\frac{1}{c^2}\sum_j\frac{d_i}{dt}((\vec v_j-\vec v_i)\varphi_j)\right ]$

$\varphi$
$\varphi$

 - скалярный потенциал заряда, инвариантный во всех системах отсчета (при инерционном движении заряда - недеформированный кулоновский потенциал, при ускоренном движении - деформированный);

$\frac{d_i}{dt}$

$\frac{d_i}{dt}$

- ПОЛНАЯ субстанциональная производная, соответствующая заряду i (равна частной производной в его собственной системе); c - кинетический коэффициент (не обязательно отождествление его со скоростью света).

Рассмотрим это на примере двух зарядов q1 и q2, совершающих инерционно движение с произвольными скоростями v1 и v2 в разных направлениях в некоторой произвольной системе наблюдения.

Обозначаю относительную скорость зарядов

$\vec v_2-\vec v_1=\vec v_{12}=-(\vec v_1-\vec v_2)=-\vec v_{21}$

$\vec v_2-\vec v_1=\vec v_{12}=-(\vec v_1-\vec v_2)=-\vec v_{21}$

Сила воздействия второго заряда на первый

$\vec F_{12}=q_1\left [-\nabla\varphi_2-\nabla(v_{12}^2\varphi_2/2c^2)-\frac{1}{c^2}\vec v_{12} \frac{d_1\varphi_2}{dt}\right ]$

$\vec F_{12}=q_1\left [-\nabla\varphi_2-\nabla(v_{12}^2\varphi_2/2c^2)-\frac{1}{c^2}\vec v_{12} \frac{d_1\varphi_2}{dt}\right ]$

(2)

Сила воздействия первого заряда на второй

$\vec F_{21}=q_2\left [-\nabla\varphi_1-\nabla(v_{21}^2\varphi_1/2c^2)-\frac{1}{c^2}\vec v_{21} \frac{d_2\varphi_1}{dt}\right ]$

$\vec F_{21}=q_2\left [-\nabla\varphi_1-\nabla(v_{21}^2\varphi_1/2c^2)-\frac{1}{c^2}\vec v_{21} \frac{d_2\varphi_1}{dt}\right ]$

(3)

Первое потенциально-градиентное слагаемое в обоих случаях одинаковое с точностью до направления - это сила кулоновского взаимодействия


$|q_1\nabla\varphi_2|=|q_2\nabla\varphi_1|=q_1q_2/(4\pi\varepsilon_0 r_{12}^2)$

$|q_1\nabla\varphi_2|=|q_2\nabla\varphi_1|=q_1q_2/(4\pi\varepsilon_0 r_{12}^2)$

(4)

Второе слагаемое есть градиент кинетической энергии поля потенциала второго заряда относительно первого:

$$q_1\frac{v_{12}^2}{2c^2}\nabla\varphi_2$$

$$q_1\frac{v_{12}^2}{2c^2}\nabla\varphi_2$$

(5)

Для взаимного движения пары инерционных зарядов получим одинаковые по величине и противоположные по направлению градиентные составляющие сил.


Третье слагаемое - полная субстанциональная производная относительного импульса поля скалярного потенциала, что соответствует представлениям Максвелла о векторном потенциале как об "электрокинетическом импульсе поля":

$$-\frac{1}{c^2}\frac{d_1}{dt}(\vec v_{12}\varphi_2)=-\frac{d\vec A_{12}}{dt}$$

$$-\frac{1}{c^2}\frac{d_1}{dt}(\vec v_{12}\varphi_2)=-\frac{d\vec A_{12}}{dt}$$

(6)

Тогда все уравнение (2) можно записать в виде

$$\vec F_{12}=q_1 [-(1+ v_{12}^2/2c^2)\nabla\varphi_2-\frac{d\vec A_{12}}{dt}]$$

$$\vec F_{12}=q_1 [-(1+ v_{12}^2/2c^2)\nabla\varphi_2-\frac{d\vec A_{12}}{dt}]$$

(7)

Образно это выражение можно понимать так, что "распухшее" при движении градиентное поле заряда "сжимается" в продольном направлении производной его импульса по времени - но ниже увидим, что это не единственно возможная интерпретация.

Формально полная производная импульса поля (вмп А) при относительном движении инерционных зарядов равна частной производной в системе одного заряда, для которого определяется сила взаимодействия, и выполняется соотношение:

$$-\frac{d\vec A_{12}}{dt}=(\vec v_{12}\nabla)\vec A_{12}=\nabla(\vec v_{12}\vec A_{12})-[\vec v_{12}\times rot\vec A_{12}]$$

$$-\frac{d\vec A_{12}}{dt}=(\vec v_{12}\nabla)\vec A_{12}=\nabla(\vec v_{12}\vec A_{12})-[\vec v_{12}\times rot\vec A_{12}]$$

(8)

Однако, следует помнить, что вмп А не есть самостоятельно существующее материальное движущееся поле (как это предположил Максвелл, применив к нему "теорию движущихся тел неизменной формы), а всего лишь импульс, кинетическая характеристика движения одного заряда относительно поля скалярного потенциала другого заряда, движущегося вместе с этим своим зарядом-источником. Соответственно, в системе любого заряда импульса его собственного поля нет - есть только импульсы полей зарядов-партнеров.

Появившийся градиент конвективного кинетического потенциала vA также присутствовал у Максвелла в Трактате в п.600, хотя не была учтена кинетическая энергия движущегося поля (по крайней мере в явном виде).<br><br>Для рассматриваемой пары инерционных зарядов выполняется соотношение

$\nabla(\vec v_{12}\vec A_{12})=\nabla\varphi_2 (v_{12}^2/c^2)$

$\nabla(\vec v_{12}\vec A_{12})=\nabla\varphi_2 (v_{12}^2/c^2)$

(9)

Тогда уравнение силы взаимодействия примет вид

$\vec F_{12}=q_1 [-(1- v_{12}^2/2c^2)\nabla\varphi_2-[\vec v_{12}\times rot\vec A_{12}]]$

$\vec F_{12}=q_1 [-(1- v_{12}^2/2c^2)\nabla\varphi_2-[\vec v_{12}\times rot\vec A_{12}]]$

(10)

Теперь образно эта сила выглядит так, будто "сжавшееся" при движении градиентное поле дополнительно "растягивается" в поперечном направлении составляющей, перпендикулярной скорости и импульсу этого поля.

Для второго заряда получим аналогичное уравнение, определяющую силу, равную по величине и противоположную по направлению.

Отмечу, что эти две силы НЕ являются центральными (в том смысле, что НЕ направлены вдоль прямой, проходящей через эти два заряда) - ведь кроме градиентной составляющей, действующей центрально, есть составляющая, определяемая производной относительного импульса поля (вмп А) в общем случае НЕ центральная.

В механике вещественных сред такие силы, не центральные или перпендикулярные основному импульсу среды, обычно обусловлены непотенциальным течением среды и диссипативным действием вязкости. Здесь же мы тоже имеем проявление непотенциального характера (ротор импульса=ротор вмп А=индукция В), но нет диссипации энергии.

ПРимечание.

Можно показать, что для пары инерционных зарядов выполняются соотношения

$(\vec v\nabla)(\vec v\varphi)=\vec v(\vec v\nabla\varphi)=\vec v(\nabla(\vec v\varphi))$

$(\vec v\nabla)(\vec v\varphi)=\vec v(\vec v\nabla\varphi)=\vec v(\nabla(\vec v\varphi))$

(11)

$[\vec v\times rot(\vec v\varphi)]=[\vec v\times[\vec v\times\nabla\varphi]]=\vec v(\vec v\nabla\varphi)-\vec v^2\nabla\varphi$

$[\vec v\times rot(\vec v\varphi)]=[\vec v\times[\vec v\times\nabla\varphi]]=\vec v(\vec v\nabla\varphi)-\vec v^2\nabla\varphi$

(12)

Эти соотношения могут быть полезны в дальнейшем. 


%========================================================================

\begin{thebibliography}{99}

\bibitem{LL2}
\textit{Ландау Л.Д. Лившиц Е.М. Теория поля. М. 1973}

\bibitem{rustot}
\textit{Re: Как запаздывающий Лиенар-Вихерт становится "незапаздывающим". Визуализация}
\\\texttt{http://www.sciteclibrary.ru/cgi-bin/yabb2/YaBB.pl?num=1528093569/330\#330}

\bibitem{tamm}
\textit{И.Е.Тамм. Основы теории электричества. М. 1957}

\bibitem{flugge}
\textit{З.Флюгге Задачи по квантовой механике т.2 М. "Мир" 1974. стр. 296}

\bibitem{misyuchenko}
\textit{В. Ганкин, Ю. Ганкин, О. Куприянова, И. Мисюченко. История электромагнитной массы}
%\\\texttt{http://fphysics.com/d/232484/d/istoriya_em_massy1.pdf}


\bibitem{Haddad}
S. Haddad and S. Suleiman
\textit{NEUTRON CHARGE DISTRIBUTION AND CHARGE DENSITY DISTRIBUTIONS IN LEAD ISOTOPES}
\textit{ACTA PHYSICA POLONICA B, Vol. 30 (1999) No 1}
\\\texttt{http://www.actaphys.uj.edu.pl/fulltext?series=Reg\&vol=30\&page=119}



\end{thebibliography}


\end{document}

