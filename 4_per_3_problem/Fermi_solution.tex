\documentclass[11pt]{article}

    \usepackage[T2A]{fontenc}
\usepackage[utf8]{inputenc}
\usepackage[english, russian]{babel}

\usepackage[breakable]{tcolorbox}
    \usepackage{parskip} % Stop auto-indenting (to mimic markdown behaviour)
    
    \usepackage{iftex}
    \ifPDFTeX
    	\usepackage[T1]{fontenc}
    	\usepackage{mathpazo}
    \else
    	\usepackage{fontspec}
    \fi

    % Basic figure setup, for now with no caption control since it's done
    % automatically by Pandoc (which extracts ![](path) syntax from Markdown).
    \usepackage{graphicx}
    % Maintain compatibility with old templates. Remove in nbconvert 6.0
    \let\Oldincludegraphics\includegraphics
    % Ensure that by default, figures have no caption (until we provide a
    % proper Figure object with a Caption API and a way to capture that
    % in the conversion process - todo).
    \usepackage{caption}
    \DeclareCaptionFormat{nocaption}{}
    \captionsetup{format=nocaption,aboveskip=0pt,belowskip=0pt}

    \usepackage{float}
    \floatplacement{figure}{H} % forces figures to be placed at the correct location
    \usepackage{xcolor} % Allow colors to be defined
    \usepackage{enumerate} % Needed for markdown enumerations to work
    \usepackage{geometry} % Used to adjust the document margins
    \usepackage{amsmath} % Equations
    \usepackage{amssymb} % Equations
    \usepackage{textcomp} % defines textquotesingle
    % Hack from http://tex.stackexchange.com/a/47451/13684:
    \AtBeginDocument{%
        \def\PYZsq{\textquotesingle}% Upright quotes in Pygmentized code
    }
    \usepackage{upquote} % Upright quotes for verbatim code
    \usepackage{eurosym} % defines \euro
    \usepackage[mathletters]{ucs} % Extended unicode (utf-8) support
    \usepackage{fancyvrb} % verbatim replacement that allows latex
    \usepackage{grffile} % extends the file name processing of package graphics 
                         % to support a larger range
    \makeatletter % fix for old versions of grffile with XeLaTeX
    \@ifpackagelater{grffile}{2019/11/01}
    {
      % Do nothing on new versions
    }
    {
      \def\Gread@@xetex#1{%
        \IfFileExists{"\Gin@base".bb}%
        {\Gread@eps{\Gin@base.bb}}%
        {\Gread@@xetex@aux#1}%
      }
    }
    \makeatother
    \usepackage[Export]{adjustbox} % Used to constrain images to a maximum size
    \adjustboxset{max size={0.9\linewidth}{0.9\paperheight}}

    % The hyperref package gives us a pdf with properly built
    % internal navigation ('pdf bookmarks' for the table of contents,
    % internal cross-reference links, web links for URLs, etc.)
    \usepackage{hyperref}
    % The default LaTeX title has an obnoxious amount of whitespace. By default,
    % titling removes some of it. It also provides customization options.
    \usepackage{titling}
    \usepackage{longtable} % longtable support required by pandoc >1.10
    \usepackage{booktabs}  % table support for pandoc > 1.12.2
    \usepackage[inline]{enumitem} % IRkernel/repr support (it uses the enumerate* environment)
    \usepackage[normalem]{ulem} % ulem is needed to support strikethroughs (\sout)
                                % normalem makes italics be italics, not underlines
    \usepackage{mathrsfs}
    
\usepackage{wrapfig}
\usepackage[rightcaption]{sidecap}
\providecommand{\keywords}[1]{\textbf{\textit{Keywords:}} #1}

\author{A. Yu. Drozdov}

    
    % Colors for the hyperref package
    \definecolor{urlcolor}{rgb}{0,.145,.698}
    \definecolor{linkcolor}{rgb}{.71,0.21,0.01}
    \definecolor{citecolor}{rgb}{.12,.54,.11}

    % ANSI colors
    \definecolor{ansi-black}{HTML}{3E424D}
    \definecolor{ansi-black-intense}{HTML}{282C36}
    \definecolor{ansi-red}{HTML}{E75C58}
    \definecolor{ansi-red-intense}{HTML}{B22B31}
    \definecolor{ansi-green}{HTML}{00A250}
    \definecolor{ansi-green-intense}{HTML}{007427}
    \definecolor{ansi-yellow}{HTML}{DDB62B}
    \definecolor{ansi-yellow-intense}{HTML}{B27D12}
    \definecolor{ansi-blue}{HTML}{208FFB}
    \definecolor{ansi-blue-intense}{HTML}{0065CA}
    \definecolor{ansi-magenta}{HTML}{D160C4}
    \definecolor{ansi-magenta-intense}{HTML}{A03196}
    \definecolor{ansi-cyan}{HTML}{60C6C8}
    \definecolor{ansi-cyan-intense}{HTML}{258F8F}
    \definecolor{ansi-white}{HTML}{C5C1B4}
    \definecolor{ansi-white-intense}{HTML}{A1A6B2}
    \definecolor{ansi-default-inverse-fg}{HTML}{FFFFFF}
    \definecolor{ansi-default-inverse-bg}{HTML}{000000}

    % common color for the border for error outputs.
    \definecolor{outerrorbackground}{HTML}{FFDFDF}

    % commands and environments needed by pandoc snippets
    % extracted from the output of `pandoc -s`
    \providecommand{\tightlist}{%
      \setlength{\itemsep}{0pt}\setlength{\parskip}{0pt}}
    \DefineVerbatimEnvironment{Highlighting}{Verbatim}{commandchars=\\\{\}}
    % Add ',fontsize=\small' for more characters per line
    \newenvironment{Shaded}{}{}
    \newcommand{\KeywordTok}[1]{\textcolor[rgb]{0.00,0.44,0.13}{\textbf{{#1}}}}
    \newcommand{\DataTypeTok}[1]{\textcolor[rgb]{0.56,0.13,0.00}{{#1}}}
    \newcommand{\DecValTok}[1]{\textcolor[rgb]{0.25,0.63,0.44}{{#1}}}
    \newcommand{\BaseNTok}[1]{\textcolor[rgb]{0.25,0.63,0.44}{{#1}}}
    \newcommand{\FloatTok}[1]{\textcolor[rgb]{0.25,0.63,0.44}{{#1}}}
    \newcommand{\CharTok}[1]{\textcolor[rgb]{0.25,0.44,0.63}{{#1}}}
    \newcommand{\StringTok}[1]{\textcolor[rgb]{0.25,0.44,0.63}{{#1}}}
    \newcommand{\CommentTok}[1]{\textcolor[rgb]{0.38,0.63,0.69}{\textit{{#1}}}}
    \newcommand{\OtherTok}[1]{\textcolor[rgb]{0.00,0.44,0.13}{{#1}}}
    \newcommand{\AlertTok}[1]{\textcolor[rgb]{1.00,0.00,0.00}{\textbf{{#1}}}}
    \newcommand{\FunctionTok}[1]{\textcolor[rgb]{0.02,0.16,0.49}{{#1}}}
    \newcommand{\RegionMarkerTok}[1]{{#1}}
    \newcommand{\ErrorTok}[1]{\textcolor[rgb]{1.00,0.00,0.00}{\textbf{{#1}}}}
    \newcommand{\NormalTok}[1]{{#1}}
    
    % Additional commands for more recent versions of Pandoc
    \newcommand{\ConstantTok}[1]{\textcolor[rgb]{0.53,0.00,0.00}{{#1}}}
    \newcommand{\SpecialCharTok}[1]{\textcolor[rgb]{0.25,0.44,0.63}{{#1}}}
    \newcommand{\VerbatimStringTok}[1]{\textcolor[rgb]{0.25,0.44,0.63}{{#1}}}
    \newcommand{\SpecialStringTok}[1]{\textcolor[rgb]{0.73,0.40,0.53}{{#1}}}
    \newcommand{\ImportTok}[1]{{#1}}
    \newcommand{\DocumentationTok}[1]{\textcolor[rgb]{0.73,0.13,0.13}{\textit{{#1}}}}
    \newcommand{\AnnotationTok}[1]{\textcolor[rgb]{0.38,0.63,0.69}{\textbf{\textit{{#1}}}}}
    \newcommand{\CommentVarTok}[1]{\textcolor[rgb]{0.38,0.63,0.69}{\textbf{\textit{{#1}}}}}
    \newcommand{\VariableTok}[1]{\textcolor[rgb]{0.10,0.09,0.49}{{#1}}}
    \newcommand{\ControlFlowTok}[1]{\textcolor[rgb]{0.00,0.44,0.13}{\textbf{{#1}}}}
    \newcommand{\OperatorTok}[1]{\textcolor[rgb]{0.40,0.40,0.40}{{#1}}}
    \newcommand{\BuiltInTok}[1]{{#1}}
    \newcommand{\ExtensionTok}[1]{{#1}}
    \newcommand{\PreprocessorTok}[1]{\textcolor[rgb]{0.74,0.48,0.00}{{#1}}}
    \newcommand{\AttributeTok}[1]{\textcolor[rgb]{0.49,0.56,0.16}{{#1}}}
    \newcommand{\InformationTok}[1]{\textcolor[rgb]{0.38,0.63,0.69}{\textbf{\textit{{#1}}}}}
    \newcommand{\WarningTok}[1]{\textcolor[rgb]{0.38,0.63,0.69}{\textbf{\textit{{#1}}}}}
    
    
    % Define a nice break command that doesn't care if a line doesn't already
    % exist.
    \def\br{\hspace*{\fill} \\* }
    % Math Jax compatibility definitions
    \def\gt{>}
    \def\lt{<}
    \let\Oldtex\TeX
    \let\Oldlatex\LaTeX
    \renewcommand{\TeX}{\textrm{\Oldtex}}
    \renewcommand{\LaTeX}{\textrm{\Oldlatex}}
    % Document parameters
    % Document title
    \title{Fermi\_solution}
    
    
    
    
    
% Pygments definitions
\makeatletter
\def\PY@reset{\let\PY@it=\relax \let\PY@bf=\relax%
    \let\PY@ul=\relax \let\PY@tc=\relax%
    \let\PY@bc=\relax \let\PY@ff=\relax}
\def\PY@tok#1{\csname PY@tok@#1\endcsname}
\def\PY@toks#1+{\ifx\relax#1\empty\else%
    \PY@tok{#1}\expandafter\PY@toks\fi}
\def\PY@do#1{\PY@bc{\PY@tc{\PY@ul{%
    \PY@it{\PY@bf{\PY@ff{#1}}}}}}}
\def\PY#1#2{\PY@reset\PY@toks#1+\relax+\PY@do{#2}}

\@namedef{PY@tok@w}{\def\PY@tc##1{\textcolor[rgb]{0.73,0.73,0.73}{##1}}}
\@namedef{PY@tok@c}{\let\PY@it=\textit\def\PY@tc##1{\textcolor[rgb]{0.25,0.50,0.50}{##1}}}
\@namedef{PY@tok@cp}{\def\PY@tc##1{\textcolor[rgb]{0.74,0.48,0.00}{##1}}}
\@namedef{PY@tok@k}{\let\PY@bf=\textbf\def\PY@tc##1{\textcolor[rgb]{0.00,0.50,0.00}{##1}}}
\@namedef{PY@tok@kp}{\def\PY@tc##1{\textcolor[rgb]{0.00,0.50,0.00}{##1}}}
\@namedef{PY@tok@kt}{\def\PY@tc##1{\textcolor[rgb]{0.69,0.00,0.25}{##1}}}
\@namedef{PY@tok@o}{\def\PY@tc##1{\textcolor[rgb]{0.40,0.40,0.40}{##1}}}
\@namedef{PY@tok@ow}{\let\PY@bf=\textbf\def\PY@tc##1{\textcolor[rgb]{0.67,0.13,1.00}{##1}}}
\@namedef{PY@tok@nb}{\def\PY@tc##1{\textcolor[rgb]{0.00,0.50,0.00}{##1}}}
\@namedef{PY@tok@nf}{\def\PY@tc##1{\textcolor[rgb]{0.00,0.00,1.00}{##1}}}
\@namedef{PY@tok@nc}{\let\PY@bf=\textbf\def\PY@tc##1{\textcolor[rgb]{0.00,0.00,1.00}{##1}}}
\@namedef{PY@tok@nn}{\let\PY@bf=\textbf\def\PY@tc##1{\textcolor[rgb]{0.00,0.00,1.00}{##1}}}
\@namedef{PY@tok@ne}{\let\PY@bf=\textbf\def\PY@tc##1{\textcolor[rgb]{0.82,0.25,0.23}{##1}}}
\@namedef{PY@tok@nv}{\def\PY@tc##1{\textcolor[rgb]{0.10,0.09,0.49}{##1}}}
\@namedef{PY@tok@no}{\def\PY@tc##1{\textcolor[rgb]{0.53,0.00,0.00}{##1}}}
\@namedef{PY@tok@nl}{\def\PY@tc##1{\textcolor[rgb]{0.63,0.63,0.00}{##1}}}
\@namedef{PY@tok@ni}{\let\PY@bf=\textbf\def\PY@tc##1{\textcolor[rgb]{0.60,0.60,0.60}{##1}}}
\@namedef{PY@tok@na}{\def\PY@tc##1{\textcolor[rgb]{0.49,0.56,0.16}{##1}}}
\@namedef{PY@tok@nt}{\let\PY@bf=\textbf\def\PY@tc##1{\textcolor[rgb]{0.00,0.50,0.00}{##1}}}
\@namedef{PY@tok@nd}{\def\PY@tc##1{\textcolor[rgb]{0.67,0.13,1.00}{##1}}}
\@namedef{PY@tok@s}{\def\PY@tc##1{\textcolor[rgb]{0.73,0.13,0.13}{##1}}}
\@namedef{PY@tok@sd}{\let\PY@it=\textit\def\PY@tc##1{\textcolor[rgb]{0.73,0.13,0.13}{##1}}}
\@namedef{PY@tok@si}{\let\PY@bf=\textbf\def\PY@tc##1{\textcolor[rgb]{0.73,0.40,0.53}{##1}}}
\@namedef{PY@tok@se}{\let\PY@bf=\textbf\def\PY@tc##1{\textcolor[rgb]{0.73,0.40,0.13}{##1}}}
\@namedef{PY@tok@sr}{\def\PY@tc##1{\textcolor[rgb]{0.73,0.40,0.53}{##1}}}
\@namedef{PY@tok@ss}{\def\PY@tc##1{\textcolor[rgb]{0.10,0.09,0.49}{##1}}}
\@namedef{PY@tok@sx}{\def\PY@tc##1{\textcolor[rgb]{0.00,0.50,0.00}{##1}}}
\@namedef{PY@tok@m}{\def\PY@tc##1{\textcolor[rgb]{0.40,0.40,0.40}{##1}}}
\@namedef{PY@tok@gh}{\let\PY@bf=\textbf\def\PY@tc##1{\textcolor[rgb]{0.00,0.00,0.50}{##1}}}
\@namedef{PY@tok@gu}{\let\PY@bf=\textbf\def\PY@tc##1{\textcolor[rgb]{0.50,0.00,0.50}{##1}}}
\@namedef{PY@tok@gd}{\def\PY@tc##1{\textcolor[rgb]{0.63,0.00,0.00}{##1}}}
\@namedef{PY@tok@gi}{\def\PY@tc##1{\textcolor[rgb]{0.00,0.63,0.00}{##1}}}
\@namedef{PY@tok@gr}{\def\PY@tc##1{\textcolor[rgb]{1.00,0.00,0.00}{##1}}}
\@namedef{PY@tok@ge}{\let\PY@it=\textit}
\@namedef{PY@tok@gs}{\let\PY@bf=\textbf}
\@namedef{PY@tok@gp}{\let\PY@bf=\textbf\def\PY@tc##1{\textcolor[rgb]{0.00,0.00,0.50}{##1}}}
\@namedef{PY@tok@go}{\def\PY@tc##1{\textcolor[rgb]{0.53,0.53,0.53}{##1}}}
\@namedef{PY@tok@gt}{\def\PY@tc##1{\textcolor[rgb]{0.00,0.27,0.87}{##1}}}
\@namedef{PY@tok@err}{\def\PY@bc##1{{\setlength{\fboxsep}{\string -\fboxrule}\fcolorbox[rgb]{1.00,0.00,0.00}{1,1,1}{\strut ##1}}}}
\@namedef{PY@tok@kc}{\let\PY@bf=\textbf\def\PY@tc##1{\textcolor[rgb]{0.00,0.50,0.00}{##1}}}
\@namedef{PY@tok@kd}{\let\PY@bf=\textbf\def\PY@tc##1{\textcolor[rgb]{0.00,0.50,0.00}{##1}}}
\@namedef{PY@tok@kn}{\let\PY@bf=\textbf\def\PY@tc##1{\textcolor[rgb]{0.00,0.50,0.00}{##1}}}
\@namedef{PY@tok@kr}{\let\PY@bf=\textbf\def\PY@tc##1{\textcolor[rgb]{0.00,0.50,0.00}{##1}}}
\@namedef{PY@tok@bp}{\def\PY@tc##1{\textcolor[rgb]{0.00,0.50,0.00}{##1}}}
\@namedef{PY@tok@fm}{\def\PY@tc##1{\textcolor[rgb]{0.00,0.00,1.00}{##1}}}
\@namedef{PY@tok@vc}{\def\PY@tc##1{\textcolor[rgb]{0.10,0.09,0.49}{##1}}}
\@namedef{PY@tok@vg}{\def\PY@tc##1{\textcolor[rgb]{0.10,0.09,0.49}{##1}}}
\@namedef{PY@tok@vi}{\def\PY@tc##1{\textcolor[rgb]{0.10,0.09,0.49}{##1}}}
\@namedef{PY@tok@vm}{\def\PY@tc##1{\textcolor[rgb]{0.10,0.09,0.49}{##1}}}
\@namedef{PY@tok@sa}{\def\PY@tc##1{\textcolor[rgb]{0.73,0.13,0.13}{##1}}}
\@namedef{PY@tok@sb}{\def\PY@tc##1{\textcolor[rgb]{0.73,0.13,0.13}{##1}}}
\@namedef{PY@tok@sc}{\def\PY@tc##1{\textcolor[rgb]{0.73,0.13,0.13}{##1}}}
\@namedef{PY@tok@dl}{\def\PY@tc##1{\textcolor[rgb]{0.73,0.13,0.13}{##1}}}
\@namedef{PY@tok@s2}{\def\PY@tc##1{\textcolor[rgb]{0.73,0.13,0.13}{##1}}}
\@namedef{PY@tok@sh}{\def\PY@tc##1{\textcolor[rgb]{0.73,0.13,0.13}{##1}}}
\@namedef{PY@tok@s1}{\def\PY@tc##1{\textcolor[rgb]{0.73,0.13,0.13}{##1}}}
\@namedef{PY@tok@mb}{\def\PY@tc##1{\textcolor[rgb]{0.40,0.40,0.40}{##1}}}
\@namedef{PY@tok@mf}{\def\PY@tc##1{\textcolor[rgb]{0.40,0.40,0.40}{##1}}}
\@namedef{PY@tok@mh}{\def\PY@tc##1{\textcolor[rgb]{0.40,0.40,0.40}{##1}}}
\@namedef{PY@tok@mi}{\def\PY@tc##1{\textcolor[rgb]{0.40,0.40,0.40}{##1}}}
\@namedef{PY@tok@il}{\def\PY@tc##1{\textcolor[rgb]{0.40,0.40,0.40}{##1}}}
\@namedef{PY@tok@mo}{\def\PY@tc##1{\textcolor[rgb]{0.40,0.40,0.40}{##1}}}
\@namedef{PY@tok@ch}{\let\PY@it=\textit\def\PY@tc##1{\textcolor[rgb]{0.25,0.50,0.50}{##1}}}
\@namedef{PY@tok@cm}{\let\PY@it=\textit\def\PY@tc##1{\textcolor[rgb]{0.25,0.50,0.50}{##1}}}
\@namedef{PY@tok@cpf}{\let\PY@it=\textit\def\PY@tc##1{\textcolor[rgb]{0.25,0.50,0.50}{##1}}}
\@namedef{PY@tok@c1}{\let\PY@it=\textit\def\PY@tc##1{\textcolor[rgb]{0.25,0.50,0.50}{##1}}}
\@namedef{PY@tok@cs}{\let\PY@it=\textit\def\PY@tc##1{\textcolor[rgb]{0.25,0.50,0.50}{##1}}}

\def\PYZbs{\char`\\}
\def\PYZus{\char`\_}
\def\PYZob{\char`\{}
\def\PYZcb{\char`\}}
\def\PYZca{\char`\^}
\def\PYZam{\char`\&}
\def\PYZlt{\char`\<}
\def\PYZgt{\char`\>}
\def\PYZsh{\char`\#}
\def\PYZpc{\char`\%}
\def\PYZdl{\char`\$}
\def\PYZhy{\char`\-}
\def\PYZsq{\char`\'}
\def\PYZdq{\char`\"}
\def\PYZti{\char`\~}
% for compatibility with earlier versions
\def\PYZat{@}
\def\PYZlb{[}
\def\PYZrb{]}
\makeatother


    % For linebreaks inside Verbatim environment from package fancyvrb. 
    \makeatletter
        \newbox\Wrappedcontinuationbox 
        \newbox\Wrappedvisiblespacebox 
        \newcommand*\Wrappedvisiblespace {\textcolor{red}{\textvisiblespace}} 
        \newcommand*\Wrappedcontinuationsymbol {\textcolor{red}{\llap{\tiny$\m@th\hookrightarrow$}}} 
        \newcommand*\Wrappedcontinuationindent {3ex } 
        \newcommand*\Wrappedafterbreak {\kern\Wrappedcontinuationindent\copy\Wrappedcontinuationbox} 
        % Take advantage of the already applied Pygments mark-up to insert 
        % potential linebreaks for TeX processing. 
        %        {, <, #, %, $, ' and ": go to next line. 
        %        _, }, ^, &, >, - and ~: stay at end of broken line. 
        % Use of \textquotesingle for straight quote. 
        \newcommand*\Wrappedbreaksatspecials {% 
            \def\PYGZus{\discretionary{\char`\_}{\Wrappedafterbreak}{\char`\_}}% 
            \def\PYGZob{\discretionary{}{\Wrappedafterbreak\char`\{}{\char`\{}}% 
            \def\PYGZcb{\discretionary{\char`\}}{\Wrappedafterbreak}{\char`\}}}% 
            \def\PYGZca{\discretionary{\char`\^}{\Wrappedafterbreak}{\char`\^}}% 
            \def\PYGZam{\discretionary{\char`\&}{\Wrappedafterbreak}{\char`\&}}% 
            \def\PYGZlt{\discretionary{}{\Wrappedafterbreak\char`\<}{\char`\<}}% 
            \def\PYGZgt{\discretionary{\char`\>}{\Wrappedafterbreak}{\char`\>}}% 
            \def\PYGZsh{\discretionary{}{\Wrappedafterbreak\char`\#}{\char`\#}}% 
            \def\PYGZpc{\discretionary{}{\Wrappedafterbreak\char`\%}{\char`\%}}% 
            \def\PYGZdl{\discretionary{}{\Wrappedafterbreak\char`\$}{\char`\$}}% 
            \def\PYGZhy{\discretionary{\char`\-}{\Wrappedafterbreak}{\char`\-}}% 
            \def\PYGZsq{\discretionary{}{\Wrappedafterbreak\textquotesingle}{\textquotesingle}}% 
            \def\PYGZdq{\discretionary{}{\Wrappedafterbreak\char`\"}{\char`\"}}% 
            \def\PYGZti{\discretionary{\char`\~}{\Wrappedafterbreak}{\char`\~}}% 
        } 
        % Some characters . , ; ? ! / are not pygmentized. 
        % This macro makes them "active" and they will insert potential linebreaks 
        \newcommand*\Wrappedbreaksatpunct {% 
            \lccode`\~`\.\lowercase{\def~}{\discretionary{\hbox{\char`\.}}{\Wrappedafterbreak}{\hbox{\char`\.}}}% 
            \lccode`\~`\,\lowercase{\def~}{\discretionary{\hbox{\char`\,}}{\Wrappedafterbreak}{\hbox{\char`\,}}}% 
            \lccode`\~`\;\lowercase{\def~}{\discretionary{\hbox{\char`\;}}{\Wrappedafterbreak}{\hbox{\char`\;}}}% 
            \lccode`\~`\:\lowercase{\def~}{\discretionary{\hbox{\char`\:}}{\Wrappedafterbreak}{\hbox{\char`\:}}}% 
            \lccode`\~`\?\lowercase{\def~}{\discretionary{\hbox{\char`\?}}{\Wrappedafterbreak}{\hbox{\char`\?}}}% 
            \lccode`\~`\!\lowercase{\def~}{\discretionary{\hbox{\char`\!}}{\Wrappedafterbreak}{\hbox{\char`\!}}}% 
            \lccode`\~`\/\lowercase{\def~}{\discretionary{\hbox{\char`\/}}{\Wrappedafterbreak}{\hbox{\char`\/}}}% 
            \catcode`\.\active
            \catcode`\,\active 
            \catcode`\;\active
            \catcode`\:\active
            \catcode`\?\active
            \catcode`\!\active
            \catcode`\/\active 
            \lccode`\~`\~ 	
        }
    \makeatother

    \let\OriginalVerbatim=\Verbatim
    \makeatletter
    \renewcommand{\Verbatim}[1][1]{%
        %\parskip\z@skip
        \sbox\Wrappedcontinuationbox {\Wrappedcontinuationsymbol}%
        \sbox\Wrappedvisiblespacebox {\FV@SetupFont\Wrappedvisiblespace}%
        \def\FancyVerbFormatLine ##1{\hsize\linewidth
            \vtop{\raggedright\hyphenpenalty\z@\exhyphenpenalty\z@
                \doublehyphendemerits\z@\finalhyphendemerits\z@
                \strut ##1\strut}%
        }%
        % If the linebreak is at a space, the latter will be displayed as visible
        % space at end of first line, and a continuation symbol starts next line.
        % Stretch/shrink are however usually zero for typewriter font.
        \def\FV@Space {%
            \nobreak\hskip\z@ plus\fontdimen3\font minus\fontdimen4\font
            \discretionary{\copy\Wrappedvisiblespacebox}{\Wrappedafterbreak}
            {\kern\fontdimen2\font}%
        }%
        
        % Allow breaks at special characters using \PYG... macros.
        \Wrappedbreaksatspecials
        % Breaks at punctuation characters . , ; ? ! and / need catcode=\active 	
        \OriginalVerbatim[#1,codes*=\Wrappedbreaksatpunct]%
    }
    \makeatother

    % Exact colors from NB
    \definecolor{incolor}{HTML}{303F9F}
    \definecolor{outcolor}{HTML}{D84315}
    \definecolor{cellborder}{HTML}{CFCFCF}
    \definecolor{cellbackground}{HTML}{F7F7F7}
    
    % prompt
    \makeatletter
    \newcommand{\boxspacing}{\kern\kvtcb@left@rule\kern\kvtcb@boxsep}
    \makeatother
    \newcommand{\prompt}[4]{
        {\ttfamily\llap{{\color{#2}[#3]:\hspace{3pt}#4}}\vspace{-\baselineskip}}
    }
    

    
    % Prevent overflowing lines due to hard-to-break entities
    \sloppy 
    % Setup hyperref package
    \hypersetup{
      breaklinks=true,  % so long urls are correctly broken across lines
      colorlinks=true,
      urlcolor=urlcolor,
      linkcolor=linkcolor,
      citecolor=citecolor,
      }
    % Slightly bigger margins than the latex defaults
    
    \geometry{verbose,tmargin=1in,bmargin=1in,lmargin=1in,rmargin=1in}
    
    

\begin{document}
    
    \maketitle
    
    

    
    \section{Рассмотрим решение задачи номер 689 из Батыгина Топтыгина
(издание 1962
года)}\label{ux440ux430ux441ux441ux43cux43eux442ux440ux438ux43c-ux440ux435ux448ux435ux43dux438ux435-ux437ux430ux434ux430ux447ux438-ux43dux43eux43cux435ux440-689-ux438ux437-ux431ux430ux442ux44bux433ux438ux43dux430-ux442ux43eux43fux442ux44bux433ux438ux43dux430-ux438ux437ux434ux430ux43dux438ux435-1962-ux433ux43eux434ux430}

Найти силу \(\vec{F}\), с которой заряженная симметричная частица
действует сама на себя (сила самодействия) при ускоренном поступательном
движении с малой скоростью \(v << c\). Запаздывание и лоренцево
сокращение не учитывать. Указание. Вычислить равнодействующую сил,
приложенных к малым элементам \(de\) заряда частицы, воспользовавшись
лиенар-вихертовым выражением напряжённости поля (XII. 25)

Под этим выражением Батыгин Топтыгин приводят выражение

\(\vec{E}\left(\vec{r},t\right) = e \frac{\left(1-\beta^2\right) \left(\vec{n}-\vec{v}/c \right)}{\left(1-\vec{n}\cdot\vec{v}/c\right)^3\,R^2} + e \frac{ \vec{n} \times \left[ \left( \vec{n}-\vec{v}/c\right) \times \vec{a}\right]}{c^2 \left(1-\vec{n} \cdot \vec{v}/c \right)^3 \,R }\)

\(\vec{n}=\frac{\vec{R}}{R}\) \(\beta=\frac{v}{c}\)

на странице 433 Батыгин Топтыгин интегрируют

\(d\vec{F} = - d e_2 d\vec{E} = \frac{de_1 de_2}{c^2 r} \left[ \vec{a} - \vec{r}/r\left( \vec{r}/r \cdot \vec{a}\right) \right]\)

и приводят результат

\(m =\frac{1}{{{c}^{^{2}}}}\frac{4}{3}\int\frac{de_1\,de_2}{r}\) (сгс)

    \section{Постановка
задачи}\label{ux43fux43eux441ux442ux430ux43dux43eux432ux43aux430-ux437ux430ux434ux430ux447ux438}

Пусть частица с заданным распределением объёмной плотности
электрического заряда \(\rho \left( r \right)\) приобретает ускорение
\(\overrightarrow{a}\) . Найти силу, действующую на распределённый в
объёме электрический заряд этой частицы со стороны электрического поля
самоиндукции.

Для решения этой задачи выразим электрическое поле исходя из выражения
потенциалов Лиенара Вихерта \cite{LL2}, как известно, дифференцированием
скалярного потенциала ЛВ по координатам точки наблюдения
\(\overrightarrow{E_1} = - \nabla\varphi\) и дифференцированием
векторного потенциала по времени
\(\overrightarrow{E_2}=-\frac{1}{c}\frac{\partial \overrightarrow{A}}{\partial t}\),
где \(\varphi=\frac{q}{{{R}^{*}}}\) и
\(\overrightarrow{A}=\frac{\overrightarrow{v}}{c}\frac{q}{{{R}^{*}}}\)
где
\({{R}^{*}}=\left( R-\frac{\overrightarrow{v}\cdot \overrightarrow{R}}{c} \right)\)
- радиус Лиенара Вихерта.

    Назовём первую компоненту \(\overrightarrow{E_1}\) электрического поля
градиентным полем, а вторую компоненту \(\overrightarrow{E_1}\)
электрическим полем индукции или самоиндукции

Опуская громоздкие промежуточные выкладки \cite{rustot}, для градиента
по координатам наблюдения скалярного потенциала ЛВ можно привести
\[\nabla \varphi =\nabla \frac{dq}{{{R}^{*}}}=-\frac{dq}{{{R}^{*}}^{2}}\left\{ \frac{\overrightarrow{R}}{{{R}^{*}}}\left( 1+\frac{\overrightarrow{a}\cdot \overrightarrow{R}}{{{c}^{2}}}-\frac{\overrightarrow{v}\cdot \overrightarrow{v}}{{{c}^{2}}} \right)-\frac{\overrightarrow{v}}{c} \right\}\]

А для производной по времени наблюдения векторного потенциала
\[\frac{1}{c}\frac{\partial }{\partial t}\overrightarrow{A}=\frac{1}{c}\frac{\partial }{\partial t}\frac{dq\overrightarrow{v}}{c{{R}^{*}}}=\frac{dq}{{{R}^{*}}^{2}}\left\{ \frac{\overrightarrow{a}R}{{{c}^{2}}}-\frac{\overrightarrow{v}}{c}\left( \frac{R}{{{R}^{*}}}\left( -1-\frac{\overrightarrow{a}\cdot \overrightarrow{R}}{{{c}^{2}}}+\frac{\overrightarrow{v}\cdot \overrightarrow{v}}{{{c}^{2}}} \right)+1 \right) \right\}\]

    Суммарное электрическое поле \%
\(\vec{E} = \frac{q}{k^3}((\vec{r}-\frac{r}{c}\vec{v})(1 + \frac{\vec{r}\vec{a}}{c^2} - \frac{v^2}{c^2}) - \vec{a}\frac{kr}{c^2})\)
\%
http://www.sciteclibrary.ru/cgi-bin/yabb2/YaBB.pl?num=1528093569/415\#415

\[\vec{dE} = \frac{dq}{{{R}^{*}}^{3}}\left( \left(\vec{R}-\frac{R}{c}\vec{v} \right) \left(1 + \frac{\vec{R}\vec{a}}{c^2} - \frac{v^2}{c^2} \right) - \vec{a}\frac{{R}^{*}R}{c^2} \right)\]

Градиентное электрическое поле
\[\overrightarrow{dE}_{1}=\frac{dq}{{{R}^{*}}^{2}}\left\{ \frac{\overrightarrow{R}}{{{R}^{*}}}\left( 1+\frac{\overrightarrow{a}\cdot \overrightarrow{R}}{{{c}^{2}}}-\frac{\overrightarrow{v}\cdot \overrightarrow{v}}{{{c}^{2}}} \right)-\frac{\overrightarrow{v}}{c} \right\}\]

    Электрическое поле самоиндукции

\[\overrightarrow{dE}_{2}=\frac{dq}{{{R}^{*}}^{2}}\left\{ \frac{\overrightarrow{v}}{c}\left( \frac{R}{{{R}^{*}}}\left( \frac{{{v}^{2}}}{{{c}^{2}}}-\frac{\overrightarrow{a}\cdot \overrightarrow{R}}{{{c}^{2}}}-1 \right)+1 \right)-\frac{\overrightarrow{a}R}{{{c}^{2}}} \right\}\]

В системе СИ перед этим выражением появляется множитель
\(\frac{1}{4\pi {{\varepsilon }_{0}}}=\frac{{{\mu }_{0}}{{c}^{2}}}{4\pi }=\frac{{{c}^{2}}}{{{10}^{7}}}\)

    Направляя в сферической системе координат вектор ускорения вдоль оси
\(z\) запишем

\({{R}^{*}}=R-\frac{v}{c}\left( {{z}_{a}}-{{z}_{q}} \right)=R-\frac{v}{c}\left( {{r}_{a}}\cos \left( {{\theta }_{a}} \right)-{{r}_{q}}\cos \left( {{\theta }_{q}} \right) \right)\)

\[\overrightarrow{a}\cdot \overrightarrow{R}=a\left( {{z}_{a}}-{{z}_{q}} \right)=a\left( {{r}_{a}}\cos \left( {{\theta }_{a}} \right)-{{r}_{q}}\cos \left( {{\theta }_{q}} \right) \right)\]

Где следуя Тамму \cite{tamm}, индексом \(q\) обозначены координаты
заряда, а индексом \(a\) обозначены координаты точки наблюдения, находим
поле в точке наблюдения путём интегрирования по объёму зарядов -
источников

\[\overrightarrow{E}_{a}=\int\limits_{{{V}_{q}}}
{\left(\frac{\overrightarrow{dE}_{1}}{dq}+\frac{\overrightarrow{dE}_{2}}{dq}\right)\rho \left( {{r}_{q}} \right) }d{{V }_{q}}\]

    Действующая на заряд со стороны электрических полей (градиентного и
самоиндукции) инерционная сила равна интегралу по от проекции поля на
ось \(z\) по тому же объёму того же заряда, но который теперь уже можно
навать пробным

\[{F}_{z}=\int\limits_{{{r}_{a}}}\int\limits_{{{\varphi}_{a}}}\int\limits_{{{\theta}_{a}}}{{E}_{z}\rho \left( {{r}_{a}} \right){{r}_{a}}^{2}\sin \left( {{\theta }_{a}} \right)}\ d{{\theta }_{a}}d{{\varphi }_{a}}d{{r}_{a}}\]

Из приведенных формул видно, что сила инерции электромагнитной массы
зависит от вида функции распределения плотности заряда в пространстве, а
также от скорости и ускорения заряда.

    \section{Приближение малых скоростей без учёта
запаздывания}\label{ux43fux440ux438ux431ux43bux438ux436ux435ux43dux438ux435-ux43cux430ux43bux44bux445-ux441ux43aux43eux440ux43eux441ux442ux435ux439-ux431ux435ux437-ux443ux447ux451ux442ux430-ux437ux430ux43fux430ux437ux434ux44bux432ux430ux43dux438ux44f}

В приближении малых скоростей \({}^{v}/{}_{c}\ll 1\) и малых ускорений
\(a{{r}_{0}}\ll {{c}^{2}}\) и при игнорировании запаздывания

Градиентное электрическое поле (\(z\) компонента) \[{E}_{1}=
\int\limits_{{{r}_{q}}}\int\limits_{{{\varphi}_{q}}}\int\limits_{{{\theta}_{q}}}
\left\{ \frac{z_a-z_q}{{{R}_{0}}}\left( 1+\frac{a\left( {{z}_{a}}-{{z}_{q}} \right)}{c^2} \right)
 \right\}
\frac{\rho \left( {{r}_{q}} \right){{r}_{q}}^{2}\sin \left( {{\theta }_{q}} \right)}{{{R}_{0}}^{2}}
d{{\theta }_{q}}d{{\varphi }_{q}}d{{r}_{q}}\]

Электрическое поле самоиндукции (\(z\) компонента) \[{E}_{2}=
\int\limits_{{{r}_{q}}}\int\limits_{{{\varphi}_{q}}}\int\limits_{{{\theta}_{q}}}
{\left\{ -\frac{{a_z}R_{0}}{{{c}^{2}}} \right\}
\frac{\rho \left( {{r}_{q}} \right){{r}_{q}}^{2}\sin \left( {{\theta }_{q}} \right)}{{{R}_{0}}^{2}}\ }d{{\theta }_{q}}d{{\varphi }_{q}}d{{r}_{q}}\]
где \({R}_{0}\) расстояние от точки источника заряда к точке наблюдения
без учёта запаздывания.

    Градиентное электрическое поле \({E}_{1}\) можно разложить на чисто
кулоновское слагаемое \({E}_{1q}\) и слагаемое зависящее от ускорения
\({E}_{1a}\)

\({E}_{1} = {E}_{1q} + {E}_{1a}\)

\[{E}_{1q}=
\int\limits_{{{r}_{q}}}\int\limits_{{{\varphi}_{q}}}\int\limits_{{{\theta}_{q}}}
\left\{ \frac{z_a-z_q}{{{R}_{0}}}
 \right\}
\frac{\rho \left( {{r}_{q}} \right){{r}_{q}}^{2}\sin \left( {{\theta }_{q}} \right)}{{{R}_{0}}^{2}}
d{{\theta }_{q}}d{{\varphi }_{q}}d{{r}_{q}}\]

\[{E}_{1a}=
\int\limits_{{{r}_{q}}}\int\limits_{{{\varphi}_{q}}}\int\limits_{{{\theta}_{q}}}
\left( \frac{a\left( {{z}_{a}}-{{z}_{q}} \right)^2}{c^2} \right)
\frac{\rho \left( {{r}_{q}} \right){{r}_{q}}^{2}\sin \left( {{\theta }_{q}} \right)}{{{R}_{0}}^{3}}
d{{\theta }_{q}}d{{\varphi }_{q}}d{{r}_{q}}\]

Откуда

\({{F}_{1}}=\frac{\overrightarrow{a}}{{{c}^{^{2}}}}\int\limits_{{{V}_{a}}}{\int\limits_{{{V}_{q}}}{\left( {{z}_{a}}-{{z}_{q}} \right)^2\frac{\rho \left( {{r}_{q}} \right)\rho \left( {{r}_{a}} \right)}{R_{0}^3}}}\ d{{V}_{q}}d{{V}_{a}}\)

\({{F}_{2}}=-\frac{\overrightarrow{a}}{{{c}^{^{2}}}}\int\limits_{{{V}_{a}}}{\int\limits_{{{V}_{q}}}{\frac{\rho \left( {{r}_{q}} \right)\rho \left( {{r}_{a}} \right)}{R_{0}}}}\ d{{V}_{q}}d{{V}_{a}}\)

    Сопоставляя с законами Ньютона для электромагнитной массы получаем
выражение

\(m_1= - \frac{1}{{{c}^{^{2}}}}\int\limits_{{{V}_{a}}}{\int\limits_{{{V}_{q}}}{\left( {{r}_{a}}\cos \left( {{\theta }_{a}} \right)-{{r}_{q}}\cos \left( {{\theta }_{q}} \right) \right)^2\frac{\rho \left( {{r}_{q}} \right)\rho \left( {{r}_{a}} \right)}{R_{0}^3}}}\ d{{V}_{q}}d{{V}_{a}}\)

\(m_2=\frac{1}{{{c}^{^{2}}}}\int\limits_{{{V}_{a}}}{\int\limits_{{{V}_{q}}}{\frac{\rho \left( {{r}_{q}} \right)\rho \left( {{r}_{a}} \right)}{R_{0}}}}\ d{{V}_{q}}d{{V}_{a}}\)

(в системе сгс) и соответственно

\(m_1= - \frac{{{\mu }_{0}}}{4\pi } \int\limits_{{{V}_{a}}}{\int\limits_{{{V}_{q}}}{\left( {{r}_{a}}\cos \left( {{\theta }_{a}} \right)-{{r}_{q}}\cos \left( {{\theta }_{q}} \right) \right)^2\frac{\rho \left( {{r}_{q}} \right)\rho \left( {{r}_{a}} \right)}{R_{0}^3}}}\ d{{V}_{q}}d{{V}_{a}}\)

\(m_2=\frac{{{\mu }_{0}}}{4\pi }\int\limits_{{{V}_{a}}}{\int\limits_{{{V}_{q}}}{\frac{\rho \left( {{r}_{q}} \right)\rho \left( {{r}_{a}} \right)}{R_{0}}}}\ d{{V}_{q}}d{{V}_{a}}\)
в системе СИ

    Рассчитаем теперь электромагнитную массу равномерно заряженной сферы
радиуса \({{r}_{0}}\)

Поскольку расстояние между координатами заряда и точки наблюдения
\({{R}_{0}}=\left|\overrightarrow{r_{q}} - \overrightarrow{r_{a}}\right|\)
находится в знаменателе, то в сферической системе координат можно
применить разложение по сферическим гармоникам следующего вида
\cite{flugge} если \(\left( {{r}_{q}}<{{r}_{a}} \right)\) то

\(\frac{1}{\left| \overrightarrow{{{r}_{q}}}-\overrightarrow{{{r}_{a}}} \right|}=\frac{1}{{{r}_{a}}}\sum\limits_{l=0}^{\infty }{{{\left( \frac{{{r}_{q}}}{{{r}_{a}}} \right)}^{l}}{{P}_{l}} \cos \left( \gamma \right)}\)

и если \(\left( {{r}_{a}}<{{r}_{q}} \right)\) то

\(\frac{1}{\left| \overrightarrow{{{r}_{q}}}-\overrightarrow{{{r}_{a}}} \right|}=\frac{1}{{{r}_{q}}}\sum\limits_{l=0}^{\infty }{{{\left( \frac{{{r}_{a}}}{{{r}_{q}}} \right)}^{l}}{{P}_{l}} \cos \left( \gamma \right)}\)

В данной формуле \({{P}_{l}} \cos \left( \gamma \right)\) это полиномы
Лежандра аргумент которых \(\gamma\) есть угол между векторами
\({{r}_{q}}\) и \({{r}_{a}}\). Применяя формулу, известную как теорему
сложения

\({{P}_{l}}\cos \left( \gamma \right)=\frac{4\pi }{2l+1}\sum\limits_{m=-l}^{l}{Y_{l,m}^{*}\left( {{\theta }_{a}},{{\varphi }_{a}} \right)}\ {{Y}_{l,m}}\left( {{\theta }_{q}},{{\varphi }_{q}} \right)\)

получаем способ аналитического вычисления интеграла индукционной
компоненты инертной электромагнитной массы.

    Пусть заряд представляет собой сферу, равномерно заряженную по всему
объёму тогда плотность заряда составит
\(\rho \left( r \right)=\frac{e}{{}^{4}/{}_{3}\pi {{r}_{0}}^{3}}\)

Производя вычисления для интеграла градиентной компоненты
электромагнитной массы получено значение

\(m_1 =\frac{1}{{{c}^{^{2}}}}\frac{2}{5}\frac{e^2}{{{r}_{0}}}\) (сгс) и
\(m_1 =\frac{{{\mu }_{0}}}{4\pi }\frac{2}{5}\frac{e^2}{{{r}_{0}}}\) (СИ)

Производя вычисления для интеграла индукционной компоненты
электромагнитной массы получено значение

\(m_2 =\frac{1}{{{c}^{^{2}}}}\frac{6}{5}\frac{e^2}{{{r}_{0}}}\) (сгс) и
\(m_2 =\frac{{{\mu }_{0}}}{4\pi }\frac{6}{5}\frac{e^2}{{{r}_{0}}}\) (СИ)

    Пусть заряд представляет собой сферическую поверхность с равномерным
поверхностным распределением заряда по поверхности сферы, равным
\(\sigma=\frac{e}{4\pi {{r}_{0}}^{2}}\) тогда для вычисления
индукционной компоненты инертной электромагнитной массы потребуется
формула
\(m_2=\frac{1}{{{c}^{^{2}}}}\int\limits_{{{S}_{a}}}{\int\limits_{{{S}_{q}}}{\frac{\sigma \left( {{r}_{q}} \right)\sigma \left( {{r}_{a}} \right)}{R}}}\ d{{S}_{q}}d{{S}_{a}}\)

Вычисления по которой дают следующий результат

\(m_2 =\frac{1}{{{c}^{^{2}}}}\frac{e^2}{{{r}_{0}}}\) (сгс) и
\(m_2 =\frac{{{\mu }_{0}}}{4\pi }\frac{e^2}{{{r}_{0}}}\) (СИ)

    Тамм \cite{tamm} для собственной электрической энергии заряженного шара
радиуса \(a\) находит \(W=\frac{e^2}{2a}\) если заряд распределён на
поверхности шара и \(W=\frac{3e^2}{5a}\) если заряд распределён по всему
объёму шара. Появление коэффициента \(\frac{1}{2}\) в формуле энергии
системы зарядов Тамм объясняет тем, что "в сумму энергия каждой пары
зарядов входит дважды, так, например, в ней встретится как член
\({e_{1}}{e_{1}}/{R_{12}}\) так и равный ему член
\({e_{2}}{e_{1}}/{R_{21}}\)".

Однако в задаче, рассмотренной в самом начале данной работы при
нахождении силы, действующей на распределённый в объёме электрический
заряд этой частицы со стороны электрического поля самоиндукции, подобные
рассуждения неприменимы. Поскольку хотя и каждая пара зарядов
\(d{e}_{1}\) и \(d{e}_{2}\) при совместном поступательном движении
создаёт две одинаковые как по виду формулы, так и по значинию силы
самоиндукции, они обе должны быть включены в общую электромагнитную
инерцию, поскольку одна из них - это сила самоиндукции действующая на
заряд \(d{e}_{1}\) со стороны поля заряда \(d{e}_{2}\), а вторая это
сила самоиндукции действующая на заряд \(d{e}_{2}\) со стороны поля
заряда \(d{e}_{1}\).

Таким образом потенциальная энергия электростатического поля модели
электрона в виде полой сферы
\({U}_{0} =\frac{1}{2}\frac{e^2}{{{r}_{0}}}\) (сгс) при том что
индукционной компонента инертной массы электрона этой модели равна
\(m_2 =\frac{1}{{{c}^{^{2}}}}\frac{e^2}{{{r}_{0}}}\) (сгс).

В любой другой модели потенциальная энергия электростатического поля
заряженной частицы равна
\({U}_{0}=\frac{1}{2}\int\limits_{{{V}_{a}}}{\int\limits_{{{V}_{q}}}{\frac{\rho \left( {{r}_{q}} \right)\rho \left( {{r}_{a}} \right)}{R_{0}}}}\ d{{V}_{q}}d{{V}_{a}}\)
(сгс) тогда как индукционной компонента инертной массы
\(m=\frac{1}{{{c}^{^{2}}}}\int\limits_{{{V}_{a}}}{\int\limits_{{{V}_{q}}}{\frac{\rho \left( {{r}_{q}} \right)\rho \left( {{r}_{a}} \right)}{R_{0}}}}\ d{{V}_{q}}d{{V}_{a}}\)
(сгс).

    Электромагнитная масса равномерно заряженной сферы с общим зарадом \(e\)
и радиусом \(r_0\) равна \(\frac{8}{5} \frac{e^2}{r_0 c^2}\); если же
заметить, что электростатическая энергия
\(u = \frac{3}{5} \frac{e^2}{r_0}\), то для массы находим как раз
\(\left(\frac{4}{3}\right) 2 \frac{u}{c^2}\).

Русская википедия в статье "Классический радиус электрона" на момент
написания данной работы даёт следующее определение:

Классический радиус электрона равен радиусу полой сферы, на которой
равномерно распределён заряд, если этот заряд равен заряду электрона, а
потенциальная энергия электростатического поля \({U}_{0}\) полностью
эквивалентна половине массы электрона (без учета квантовых эффектов):

\({\displaystyle U_{0}={\frac {1}{2}}{\frac {1}{4\pi \varepsilon _{0}}}\cdot {\frac {e^{2}}{r_{0}}}={\frac {1}{2}}m_{0}c^{2}}\).

    \section{Решение проблемы
4/3}\label{ux440ux435ux448ux435ux43dux438ux435-ux43fux440ux43eux431ux43bux435ux43cux44b-43}

Энрико Ферми в статье \cite{Fermi1923} 1923 года,
https://github.com/daju1/articles/blob/master/4\_per\_3\_problem/references/\%D0\%A4\%D0\%B5\%D1\%80\%D0\%BC\%D0\%B823.pdf

https://github.com/daju1/articles/blob/master/4\_per\_3\_problem/Fermi1923.ipynb

предлагает метод решения проблемы 4/3.

    Применяя Гамильтонов метод с использованием аппарата теории
относительности следующим образом: для вычисления действия нужно
использовать переменные границы интегрирования для времени \(t_1\) и
\(t_2\) Ферми приходит к соотношению (третья без номера формуле сверху
на странице 80)

\[\int \vec E^{(i)} de + \int \vec E^{(i)}\frac{\vec a \vec R}{c^2} de + e \vec E^{(e)} + \int \vec E^{(e)}\frac{\vec a \vec R}{c^2} de = 0\]

    Первое, он находит возможным пренебречь последним интегралом.

Второе. Он пишет, что \(E^{(i)}\) это сумма сил Кулона и члена,
содержащего ускорение. Действительно, чтобы получить для
электромагнитной массы значение 4/3 нужно как например Батыгин и
Топтыгин вычисляют инертную массу (в задаче 689 их задачника 1962 года
издания), при вычислении первого интеграла в качестве \(E^{(i)}\) нужно
взять поле по Лиенару Вихерту (при том что Батыгин и Топтыгин в этой
задаче пренебрегают как запаздываение так и лоренцевым сокращением).

Если при вычислении второго интеграла в качестве \(E^{(i)}\) взять
просто лишь кулоновское поле, как это сделал Ферми, вычисляя интеграл в
соотношении (4), то он это сделал только лишь потому что он пренебрёг
слагаемыми интеграла которые содержат квадрат ускорения.

Давайте посмотрим насколько оправдано это пренебрежение.

В приближении малых скоростей \({}^{v} \big / {}_{c}\ll 1\) и малых
ускорений \(a{{r}_{0}}\ll {{c}^{2}}\) и при игнорировании запаздывания

Градиентное электрическое поле (\(z\) компонента) \[{E}_{1}=
\int\limits_{{{r}_{q}}}\int\limits_{{{\varphi}_{q}}}\int\limits_{{{\theta}_{q}}}
\left\{ \frac{z_a-z_q}{{{R}_{0}}^3}\left( 1+\frac{a_z\left( {{z}_{a}}-{{z}_{q}} \right)}{c^2} \right)
 \right\}
{\rho \left( {{r}_{q}} \right){{r}_{q}}^{2}\sin \left( {{\theta }_{q}} \right)}
d{{\theta }_{q}}d{{\varphi }_{q}}d{{r}_{q}}\]

Электрическое поле самоиндукции (\(z\) компонента) \[{E}_{2}=
\int\limits_{{{r}_{q}}}\int\limits_{{{\varphi}_{q}}}\int\limits_{{{\theta}_{q}}}
{\left\{ -\frac{{a_z}}{{{c}^{2}}{{{R}_{0}}}} \right\}
{\rho \left( {{r}_{q}} \right){{r}_{q}}^{2}\sin \left( {{\theta }_{q}} \right)}\ }d{{\theta }_{q}}d{{\varphi }_{q}}d{{r}_{q}}\]
где \({R}_{0}\) расстояние от точки источника заряда к точке наблюдения
без учёта запаздывания.

Записывая в обозначениях близких к работе Ферми, обозначая как
\({E}^{(i)}\) поле, обусловленное самой системой

\[{E}^{(i)}={E}_{1}+{E}_{2}=
\int\left\{ \frac{z_a-z_q}{{{R}_{0}}^3}+\frac{a_z\left( {{z}_{a}}-{{z}_{q}} \right)^2}{c^2{{{R}_{0}}^3}} - \frac{{a_z}}{{{c}^{2}}{{{R}_{0}}}}
\right\} de'\]

    Вычисляя инертную массу электрона с помощью потенциалов ЛВ, получаем
вслед за Батыгиным-Топтыгиным 4/3
\%https://github.com/daju1/articles/blob/master/sagemath/inertial\_mass\_of\_charged\_particle/About\_inertial\_properties\_of\_electromagnetic\_mass.01.pdf

\[\int {E}^{(i)} de = \int\int \left\{ \frac{z_a-z_q}{{{R}_{0}}^3}+\frac{a_z\left( {{z}_{a}}-{{z}_{q}} \right)^2}{c^2{{{R}_{0}}^3}}  -\frac{{a_z}}{{{c}^{2}}{{{R}_{0}}}}
 \right\}
de de'\]

для равномерно по всему объёму заряженной сферы

\[\int {E}^{(i)} de = \left\{ 0 - \frac{2}{5}  - \frac{6}{5} \right\}\frac{a_z e^2}{ c^2{{{R}}}}\]

Учитывая что для равномерно заряженной по всему объему сферы

\[u = \frac{3}{5}\frac{e^2}{R}\]

находим

\[\int {E}^{(i)} de = -\frac{4}{3} \cdot 2\frac{u}{c^2}\,a_z.\]

    Соотношение (4) из работы Ферми, если записать его в современных
обозначениях будет выглядеть так:

\[-\frac{4}{3}\cdot 2 \,\frac{u}{c^2} \, a_z + \int \vec E^{(i)}\frac{\vec a \vec R}{c^2} de + \vec F = 0\]

    Подставляя выражение для \({E}^{(i)}\)

\[-\frac{4}{3}\cdot 2 \,\frac{u}{c^2} \, a_z + \int \int
\left\{ \frac{z_a-z_q}{{{R}_{0}}^3}+\frac{a_z\left( {{z}_{a}}-{{z}_{q}} \right)^2}{c^2{{{R}_{0}}^3}}  -\frac{{a_z}}{{{c}^{2}}{{{R}_{0}}}}
 \right\}
de'\frac{\vec a \vec R}{c^2} de + \vec F = 0\]

    и учитывая, что ускорение имеет только лишь \(z\) компоненту

\[-\frac{4}{3}\cdot 2 \,\frac{u}{c^2} \, a_z + \int \int
\left\{ \frac{z_a-z_q}{{{R}_{0}}^3}+\frac{a_z\left( {{z}_{a}}-{{z}_{q}} \right)^2}{c^2{{{R}_{0}}^3}}  -\frac{{a_z}}{{{c}^{2}}{{{R}_{0}}}}
 \right\}
de'\frac{a_z \left(z_a-z_q\right)}{c^2} de + \vec F = 0\]

\[-\frac{4}{3}\cdot 2 \,\frac{u}{c^2} \, a_z + \int \int
\left\{ \frac{a_z\left(z_a-z_q\right)^2}{{c}^{2}{{R}_{0}}^3}+\frac{a_z^2\left( {{z}_{a}}-{{z}_{q}} \right)^3}{c^4{{{R}_{0}}^3}}  -\frac{{a_z^2}\left(z_a-z_q\right)}{{{c}^{4}}{{{R}_{0}}}}
 \right\}
de' de + \vec F = 0\]

    Учитывая численное значение первого интеграла для равномерно заряженной
по объёму сферы

\[-\frac{4}{3}\cdot 2 \,\frac{u}{c^2} \, a_z + \frac{2}{5} \frac{a_z e^2}{c^2{R}} + \int \int \left\{
\frac{a_z^2\left( {{z}_{a}}-{{z}_{q}} \right)^3}{c^4{{{R}_{0}}^3}}  -\frac{{a_z^2}\left(z_a-z_q\right)}{{{c}^{4}}{{{R}_{0}}}}
 \right\}
de' de + \vec F = 0\]

И подставляя значение электростатической энергии

\[-\frac{4}{3}\cdot 2 \,\frac{u}{c^2} \, a_z + \frac{1}{3} \cdot 2 \, \frac{u}{c^2} a_z + \int \int \left\{
\frac{a_z^2\left( {{z}_{a}}-{{z}_{q}} \right)^3}{c^4{{{R}_{0}}^3}}  -\frac{{a_z^2}\left(z_a-z_q\right)}{{{c}^{4}}{{{R}_{0}}}}
 \right\}
de' de + \vec F = 0\]

    Приходим к выражению

\[\vec F =  2 \,\frac{u}{c^2} \, a_z - \int \int \left\{
\frac{a_z^2\left( {{z}_{a}}-{{z}_{q}} \right)^3}{c^4{{{R}_{0}}^3}}  -\frac{{a_z^2}\left(z_a-z_q\right)}{{{c}^{4}}{{{R}_{0}}}}
 \right\} de' de.\]

    В этом выражении множитель 2 перед \({u}/{c^2}\) соответствует
определению классического радиуса электрона из Википедии, который "равен
радиусу заряженной сферы (на которой равномерно распределён заряд), если
этот заряд равен заряду электрона, а потенциальная энергия
электростатического поля \(\textbf{u}\) полностью эквивалентна половине
массы электрона".

    Ферми показал дополнительный интеграл (равный -1/3), имеющий
отрицательный знак, который нужно использовать чтобы решить проблему
4/3.

Исходя из формализма ЛВ этот дополнительный интеграл Ферми как раз равен
половине градиентного интеграла который возникает при решении задачи об
инертной массе электрона методом ЛВ. А градиентный интеграл это интеграл
поля \(E_1\) в обозначениях текущей работы. Знак компоненты поля
\(E_1\), полученный в данной работе, отрицательный. Если увеличить,
вслед за Ферми, величину градиентного интеграла \(E_1\) (Ферми его
увеличил в полтора раза), то можно получить отрицательный знак
электрического импульса центрально симметричного взрыва плазмы, как это
показано в публикации Менде.

И таким образом все сходится. По крайней мере получается качественное
сходство теории и практики.

Но теперь возникает вопрос: как на практике для точного решения
практических задач использовать потенциалы Лиенара Вихерта с учётом
\({}^{4}/{}_{3}\) поправки Энрико Ферми?

\% Каждый раз вводить поправку увеличивая градиентную компоненту поля?
Тогда как эта поправка должна выглядеть....

Суммарное электрическое поле по Лиенару Вихерту

\[\vec{E} = \frac{q}{{{R}^{*}}^{3}}\left( \left(\vec{R}-\frac{R}{c}\vec{v} \right) \left(1 + \frac{\vec{R}\vec{a}}{c^2} - \frac{v^2}{c^2} \right) - \vec{a}\,\frac{{R}^{*}R}{c^2} \right)\]

получено в электродинамике (см. например ЛЛ2, \("\)Теория поля\("\),
параграф 63, формула (63,8)) исходя из вариации действия типа А,
противоречащей принципу относительности. Но для того, чтобы получить
поле, соответствующее вариации типа В, необходимо поле Лиенара Вихерта
дополнить множителем Ферми
\(\left(1 + \frac{\vec{R}\vec{a}}{c^2}\right)\), возникшим благодаря
искривлению мировой трубки в 4-пространстве при движении заряда с
ускорением.

Таким образом, новая формула поля, назовём ее поле
Лиенара-Вихерта-Ферми, имеет вид:

\[\vec{E} = \frac{q}{{{R}^{*}}^{3}}\left( \left(\vec{R}-\frac{R}{c}\vec{v} \right) \left(1 + \frac{\vec{R}\vec{a}}{c^2} - \frac{v^2}{c^2} \right) - \vec{a}\,\frac{{R}^{*}R}{c^2} \right) \left(1 + \frac{\vec{R}\vec{a}}{c^2}\right).\]

Суммарное магнитное поле по Лиенару Вихерту:

\[\vec{B} = -\frac{q}{{{R}^{*}}^3}\left(\vec{R}\times\frac{\vec{v}}{c}\left(1 + \frac{\vec{R}\vec{a}}{c^2} - \frac{v^2}{c^2}\right) + \vec{R}\times\frac{\vec{a}{{R}^{*}}}{c^2}\right).\]

И, соответственно, новая формула магнитного поля Лиенара-Вихерта-Ферми,
имеет вид:

\[\vec{B} = -\frac{q}{{{R}^{*}}^3}\left(\vec{R}\times\frac{\vec{v}}{c}\left(1 + \frac{\vec{R}\vec{a}}{c^2} - \frac{v^2}{c^2}\right) + \vec{R}\times\frac{\vec{a}{{R}^{*}}}{c^2}\right) \left(1 + \frac{\vec{R}\vec{a}}{c^2}\right).\]

    Здесь нужно отметить, что множитель Ферми
\(\left( 1 + \frac{\vec a \vec r}{c^2}\right)\) был получен для случая
когда скорость равна нулю. В общем случае ненулевой скорости множитель
Ферми будет выглядеть

\[\left( 1 + \frac{g_i r^i}{c^2}\right)\]

где \(g_i\) - 4-ускорение, которое, как известно, является вектором
кривизны мировой линии, \(r^i\) - 4-радиус-вектор из точки источника
поля (заряда) в точку наблючения поля.

    \section{Импульс поля движущегося
заряда}\label{ux438ux43cux43fux443ux43bux44cux441-ux43fux43eux43bux44f-ux434ux432ux438ux436ux443ux449ux435ux433ux43eux441ux44f-ux437ux430ux440ux44fux434ux430}

    Импульс поля движущейся частицы

\[\vec G = \int \vec g \, dV\]

где \(\vec g = \frac{1}{4 \pi c} \vec E \times \vec B\).

Здесь электрическое поле по Лиенару Вихерту

\[\vec{E} = \frac{q}{{{R}^{*}}^{3}}\left( \left(\vec{R}-\frac{R}{c}\vec{v} \right) \left(1 + \frac{\vec{R}\vec{a}}{c^2} - \frac{v^2}{c^2} \right) - \vec{a}\,\frac{{R}^{*}R}{c^2} \right),\]

    или покомпонентно

    \[{E_x} = \frac{q}{{{R}^{*}}^{3}}\left( \left({R_x}-\frac{R}{c}{v_x} \right) \left(1 + \frac{\vec{R}\vec{a}}{c^2} - \frac{v^2}{c^2} \right) - {a_x}\,\frac{{R}^{*}R}{c^2} \right),\]

    \[{E_y} = \frac{q}{{{R}^{*}}^{3}}\left( \left({R_y}-\frac{R}{c}{v_y} \right) \left(1 + \frac{\vec{R}\vec{a}}{c^2} - \frac{v^2}{c^2} \right) - {a_y}\,\frac{{R}^{*}R}{c^2} \right),\]

    \[{E_z} = \frac{q}{{{R}^{*}}^{3}}\left( \left({R_z}-\frac{R}{c}{v_z} \right) \left(1 + \frac{\vec{R}\vec{a}}{c^2} - \frac{v^2}{c^2} \right) - {a_z}\,\frac{{R}^{*}R}{c^2} \right).\]

    магнитное поле по Лиенару Вихерту

    \[\vec{B} = -\frac{q}{{{R}^{*}}^3}\left(\left(\vec{R}\times\frac{\vec{v}}{c}\right)\left(1 + \frac{\vec{R}\vec{a}}{c^2} - \frac{v^2}{c^2}\right) + \left(\vec{R}\times\vec{a}\right)\frac{{{R}^{*}}}{c^2}\right).\]

    или покомпонентно

    \[{B_x} = -\frac{q}{{{R}^{*}}^3}\left(\left(R_y\,\frac{v_z}{c} -  R_z\,\frac{v_y}{c}\right)\left(1 + \frac{\vec{R}\vec{a}}{c^2} - \frac{v^2}{c^2}\right) + \left(R_y\,a_z -  R_z\,a_y\right)\frac{{{R}^{*}}}{c^2}\right),\]

    \[{B_y} = -\frac{q}{{{R}^{*}}^3}\left(\left(R_z\,\frac{v_x}{c} -  R_x\,\frac{v_z}{c}\right)\left(1 + \frac{\vec{R}\vec{a}}{c^2} - \frac{v^2}{c^2}\right) + \left(R_z\,a_x -  R_x\,a_z\right)\frac{{{R}^{*}}}{c^2}\right),\]

    \[{B_z} = -\frac{q}{{{R}^{*}}^3}\left(\left(R_x\,\frac{v_y}{c} -  R_y\,\frac{v_x}{c}\right)\left(1 + \frac{\vec{R}\vec{a}}{c^2} - \frac{v^2}{c^2}\right) + \left(R_x\,a_y -  R_y\,a_x\right)\frac{{{R}^{*}}}{c^2}\right).\]

    Если ускорение равно нулю

    \[\vec{E} = \frac{q}{{{R}^{*}}^{3}}\left( \left(\vec{R}-\frac{R}{c}\vec{v} \right) \left(1 - \frac{v^2}{c^2} \right) \right),\]

    \[\vec{B} = -\frac{q}{{{R}^{*}}^3}\left(\left(\vec{R}\times\frac{\vec{v}}{c}\right)\left(1 - \frac{v^2}{c^2}\right) \right).\]

    \[{B_x} = -\frac{q}{{{R}^{*}}^3}\left(\left(R_y\,\frac{v_z}{c} -  R_z\,\frac{v_y}{c}\right)\left(1 - \frac{v^2}{c^2}\right)\right),\]

    \[{B_y} = -\frac{q}{{{R}^{*}}^3}\left(\left(R_z\,\frac{v_x}{c} -  R_x\,\frac{v_z}{c}\right)\left(1 - \frac{v^2}{c^2}\right) \right),\]

    \[{B_z} = -\frac{q}{{{R}^{*}}^3}\left(\left(R_x\,\frac{v_y}{c} -  R_y\,\frac{v_x}{c}\right)\left(1 - \frac{v^2}{c^2}\right) \right).\]

    \[{q_z} = \frac{1}{4 \pi c} \left(E_x\,{B_y} -  E_y\,{B_x}\right).\]

    Если скорость имеет только лишь \(z\) компоненту

    \[{E_x} = \frac{q}{{{R}^{*}}^{3}}\left( {R_x} \left(1 - \frac{v^2}{c^2} \right) \right),\]

    \[{E_y} = \frac{q}{{{R}^{*}}^{3}}\left( {R_y} \left(1 - \frac{v^2}{c^2} \right) \right),\]

    \[{B_x} = -\frac{q}{{{R}^{*}}^3}\left(R_y\,\frac{v_z}{c} \left(1 - \frac{v^2}{c^2}\right)\right),\]

    \[{B_y} = +\frac{q}{{{R}^{*}}^3}\left(R_x\,\frac{v_z}{c}\left(1 - \frac{v^2}{c^2}\right) \right),\]

    \begin{tcolorbox}[breakable, size=fbox, boxrule=1pt, pad at break*=1mm,colback=cellbackground, colframe=cellborder]
\prompt{In}{incolor}{ }{\boxspacing}
\begin{Verbatim}[commandchars=\\\{\}]

\end{Verbatim}
\end{tcolorbox}

    \[{q_z} = \frac{1}{4 \pi c} \left(E_x\,{B_y} -  E_y\,{B_x}\right).\]

    \[{q_z} = \frac{1}{4 \pi c} \left(\frac{q}{{{R}^{*}}^{3}}\left( {R_x} \left(1 - \frac{v^2}{c^2} \right) \right)\,\frac{q}{{{R}^{*}}^3}\left(R_x\,\frac{v_z}{c}\left(1 - \frac{v^2}{c^2}\right) \right) + \frac{q}{{{R}^{*}}^{3}}\left( {R_y} \left(1 - \frac{v^2}{c^2} \right) \right)\,\frac{q}{{{R}^{*}}^3}\left(R_y\,\frac{v_z}{c} \left(1 - \frac{v^2}{c^2}\right)\right)\right).\]

    \[{q_z} = \frac{1}{4 \pi c}\frac{q^2}{{{R}^{*}}^{6}} \left(1 - \frac{v^2}{c^2} \right)^2
\left({R_x}^2 + {R_y}^2 \right)\frac{v_z}{c}.\]

    \[{q_z} = \frac{1}{4 \pi c} \left(E_x^2 + E_y^2\right)\frac{v_z}{c}.\]

    Импульс поля движущейся частицы

\[G_z = \int g_z \, dV = \frac{v_z}{4 \pi c^2} \int \left(E_x^2 + E_y^2\right) = \frac{v_z}{4 \pi c^2} \frac{2}{3}\int \left(E_x^2 + E_y^2 + E_z^2\right) = \frac{v_z}{4 \pi c^2} \frac{2}{3}\int E^2 dV\]

    Что интересно, с импульсом поля не получается используя подход Ферми
решить проблему \(4/3\).

Действительно, в методе Ферми, с использованием вариации типа В, при
постоянной скорости, поправка не возникает. Потому что при постоянной
скорости трубка не искривляется. И поэтому площадь интегрирования для
вариаций обоих типов (хотя эти площадки интегрирования и отличаются по
своей форме), но площадь этих площадок интегрирования одинакова.

    \begin{figure}
\centering
\includegraphics{Variations_A_and_B.jpg}
\caption{Variations_A_and_B.jpg}
\end{figure}


\begin{thebibliography}{99}


    \bibitem{rustot}
\textit{Re: Как запаздывающий Лиенар-Вихерт становится "незапаздывающим". Визуализация}
\textbackslash{}\texttt{http://www.sciteclibrary.ru/cgi-bin/yabb2/YaBB.pl?num=1528093569/330\#330}

\bibitem{Tamm1957}
\textit{Тамм, И. (1957). Основы теории электричества. Москва.}

\bibitem{Fermi1923}
\textit{Ферми, Энрико. (1923). Разрешение существующего противоречия между электродинамической и релятивистской теориями электромагнитной массы. Энрико Ферми, т.1 стр. 73}

\end{thebibliography}
   
    
    
\end{document}
