\documentclass{article}
\usepackage[T2A]{fontenc}
\usepackage[utf8]{inputenc}
\usepackage[english, russian]{babel}

\begin{document}

\title{Об инерционных свойствах электромагнитной массы}

\author{А.Ю.Дроздов}

\date{Jan 30, 2020}


%========================================================================

\section{Introduction}



Пусть частица с заданным распределением объёмной плотности электрического заряда $\rho \left( r \right)$ приобретает ускорение $\overrightarrow{a}$ . Найти силу, действующую на распределённый в объёме электрический заряд этой частицы со стороны электрического поля самоиндукции.

Для решения этой задачи электрическое поле самоиндукции $\overrightarrow{E}=-\frac{1}{c}\frac{\partial \overrightarrow{A}}{\partial t}$
выразим исходя из выражения векторного потенциала Лиенара-Вихерта \cite{LL2} $\overrightarrow{A}=\frac{\overrightarrow{v}}{c}\frac{q}{R-\frac{\overrightarrow{v}\cdot \overrightarrow{R}}{c}}=\frac{\overrightarrow{v}}{c}\frac{q}{{{R}^{*}}}$ дифференцирование которого приводит к выражению \cite{rustot}

\[\overrightarrow{E}=\frac{dq}{{{R}^{*}}^{2}}\left\{ \frac{\overrightarrow{v}}{c}\left( \frac{R}{{{R}^{*}}}\left( \frac{{{v}^{2}}}{{{c}^{2}}}-\frac{\overrightarrow{a}\cdot \overrightarrow{R}}{{{c}^{2}}}-1 \right)+1 \right)-\frac{\overrightarrow{a}R}{{{c}^{2}}} \right\}\]

В системе СИ перед этим выражением появляется множитель $\frac{1}{4\pi {{\varepsilon }_{0}}}=\frac{{{\mu }_{0}}{{c}^{2}}}{4\pi }=\frac{{{c}^{2}}}{{{10}^{7}}}$

Направляя в сферической системе координат вектор ускорения вдоль оси $z$  запишем

${{R}^{*}}=R-\frac{v}{c}\left( {{z}_{a}}-{{z}_{q}} \right)$

\[\overrightarrow{a}\cdot \overrightarrow{R}=a\left( {{z}_{a}}-{{z}_{q}} \right)=a\left( {{r}_{a}}\cos \left( {{\theta }_{a}} \right)-{{r}_{q}}\cos \left( {{\theta }_{q}} \right) \right)\]

Где следуя Тамму \cite{tamm}, индексом $q$ обозначены координаты заряда, а индексом $a$ обозначены координаты точки наблюдения

\[\overrightarrow{E}=\int\limits_{{{V}_{q}}}{\left\{ \frac{\overrightarrow{v}}{c}\left( \frac{R}{{{R}^{*}}}\left( \frac{{{v}^{2}}}{{{c}^{2}}}-\frac{\overrightarrow{a}\cdot \overrightarrow{R}}{{{c}^{2}}}-1 \right)+1 \right)-\frac{\overrightarrow{a}R}{{{c}^{2}}} \right\}\frac{\rho \left( {{r}_{q}} \right){{r}_{q}}^{2}\sin \left( {{\theta }_{q}} \right)}{{{R}^{*}}^{2}}\ }d{{\theta }_{q}}d{{\varphi }_{q}}d{{r}_{q}}\]

учитывая, что по теореме косинусов
${{R}^{2}}={{R}_{0}}^{2}+{{s}^{2}}-2{{R}_{0}}s\cos \left( \alpha  \right)={{R}_{0}}^{2}+{{s}^{2}}-2{{R}_{0}}s\frac{{{z}_{q'}}-{{z}_{a'}}}{{{R}_{0}}}={{R}_{0}}^{2}+{{s}^{2}}+2s\left( {{z}_{a'}}-{{z}_{q'}} \right)$
где  расстояние от точки источника заряда к точке наблюдения без учёта запаздывания
Действующая на заряд со стороны электрического поля самоиндукции инерционная сила равна

$\overrightarrow{{{F}_{z}}}=\int\limits_{{{V}_{a}}}{\overrightarrow{{{E}_{z}}}\rho \left( {{r}_{a}} \right){{r}_{a}}^{2}\sin \left( {{\theta }_{a}} \right)}\ d{{\theta }_{a}}d{{\varphi }_{a}}d{{r}_{a}}$


Из приведенных формул видно, что сила инерции электромагнитной массы зависит от вида функции распределения плотности заряда в пространстве, а также от скорости и ускорения заряда.
В приближении малых скоростей ${}^{v}/{}_{c}\ll 1$  и малых ускорений $a{{r}_{0}}\ll {{c}^{2}}$
\[\overrightarrow{E}=\int\limits_{{{V}_{q}}}{\left\{ -\frac{\overrightarrow{a}R}{{{c}^{2}}} \right\}\frac{\rho \left( {{r}_{q}} \right){{r}_{q}}^{2}\sin \left( {{\theta }_{q}} \right)}{{{R}^{*}}^{2}}\ }d{{\theta }_{q}}d{{\varphi }_{q}}d{{r}_{q}}\]
Откуда
${{F}_{z}}=-\frac{\overrightarrow{a}}{{{c}^{^{2}}}}\int\limits_{{{V}_{a}}}{\int\limits_{{{V}_{q}}}{\frac{\rho \left( {{r}_{q}} \right)\rho \left( {{r}_{a}} \right)}{R}}}\ d{{V}_{q}}d{{V}_{a}}$
Сопоставляя с законами Ньютона для электромагнитной массы получаем выражение
$m=\frac{1}{{{c}^{^{2}}}}\int\limits_{{{V}_{a}}}{\int\limits_{{{V}_{q}}}{\frac{\rho \left( {{r}_{q}} \right)\rho \left( {{r}_{a}} \right)}{R}}}\ d{{V}_{q}}d{{V}_{a}}$
Рассчитаем теперь электромагнитную массу равномерно заряженной сферы радиуса ${{r}_{0}}$
Расстояние между координатами заряда и точки наблюдения имеет вид

${{R}_{0}}=\sqrt{\\
  { { \left( {{x}_{a}}  -  {{x}_{q}}\right) }^{2}} + \\
  { { \left( {{y}_{a}}  -  {{y}_{q}}\right) }^{2}} + \\
  { { \left( {{z}_{a}}  -  {{z}_{q}}\right) }^{2}} }$

${{R}_{0}}=\left|\overrightarrow{r_{q}} - \overrightarrow{r_{a}}\right|$

В сферической системе координат \cite{flugge}

$\frac{1}{\left|\overrightarrow{r_{q}} - \overrightarrow{r_{a}}\right|} = $



${{x}_{a}} = {{r}_{a}}\sin \left( {{\theta }_{a}} \right)\cos \left( {{\varphi }_{a}} \right)$

${{x}_{q}} = {{r}_{q}}\sin \left( {{\theta }_{q}} \right)\cos \left( {{\varphi }_{q}} \right)$

${{y}_{a}} = {{r}_{a}}\sin \left( {{\theta }_{a}} \right)\sin \left( {{\varphi }_{a}} \right)$

${{y}_{q}} = {{r}_{q}}\sin \left( {{\theta }_{q}} \right)\sin \left( {{\varphi }_{q}} \right)$

${{z}_{a}} = {{r}_{a}}\cos \left( {{\theta }_{a}} \right)$

${{z}_{q}} = {{r}_{q}}\cos \left( {{\theta }_{q}} \right)$


Для его упрощения ввиду симметричности задачи поле не зависит от координаты ${{\varphi }_{a}}$поэтому мы можем положив ${{\varphi }_{a}}=0$ упростить формулу а вместо интегрирования по ${{\varphi }_{a}}$ просто применить умножение на $2\pi$

${{R}_{0}}=\sqrt{\\
 {{\left( {{r}_{a}}\sin \left( {{\theta }_{a}} \right)-{{r}_{q}}\sin \left( {{\theta }_{q}} \right)\cos \left( {{\varphi }_{q}} \right) \right)}^{2}}+ \\
 {{\left( -{{r}_{q}}\sin \left( {{\theta }_{q}} \right)\sin \left( {{\varphi }_{q}} \right) \right)}^{2}}+ \\
 {{\left( {{r}_{a}}\cos \left( {{\theta }_{a}} \right)-{{r}_{q}}\cos \left( {{\theta }_{q}} \right) \right)}^{2}} \\
}=\sqrt{
  {{\left( {{r}_{a}}\sin \left( {{\theta }_{a}} \right) \right)}^{2}}-2\left( {{r}_{a}}\sin \left( {{\theta }_{a}} \right){{r}_{q}}\sin \left( {{\theta }_{q}} \right)\cos \left( {{\varphi }_{q}} \right) \right)+ \\
  {{\left( {{r}_{q}}\sin \left( {{\theta }_{q}} \right)\cos \left( {{\varphi }_{q}} \right) \right)}^{2}}+{{\left( {{r}_{q}}\sin \left( {{\theta }_{q}} \right)\sin \left( {{\varphi }_{q}} \right) \right)}^{2}}+ \\
  {{\left( {{r}_{a}}\cos \left( {{\theta }_{a}} \right)-{{r}_{q}}\cos \left( {{\theta }_{q}} \right) \right)}^{2}} \\
} \\
  =\sqrt{{{\left( {{r}_{a}}\sin \left( {{\theta }_{a}} \right) \right)}^{2}}-2{{r}_{a}}\sin \left( {{\theta }_{a}} \right){{r}_{q}}\sin \left( {{\theta }_{q}} \right)\cos \left( {{\varphi }_{q}} \right)+{{\left( {{r}_{q}}\sin \left( {{\theta }_{q}} \right) \right)}^{2}}+{{\left( {{r}_{a}}\cos \left( {{\theta }_{a}} \right)-{{r}_{q}}\cos \left( {{\theta }_{q}} \right) \right)}^{2}}}$

Плотность заряда положим $\rho \left( r \right)=\frac{e}{{}^{4}/{}_{3}\pi {{r}_{0}}^{3}}$

В работе \cite{misyuchenko} приводится

%========================================================================

\begin{thebibliography}{99}

\bibitem{LL2}
\textit{Ландау Л.Д. Лившиц Е.М. Теория поля. М. 1973}

\bibitem{rustot}
\textit{Re: Как запаздывающий Лиенар-Вихерт становится "незапаздывающим". Визуализация}
\\\texttt{http://www.sciteclibrary.ru/cgi-bin/yabb2/YaBB.pl?num=1528093569/330\#330}

\bibitem{tamm}
\textit{И.Е.Тамм. Основы теории электричества. М. 1957}

\bibitem{flugge}
\textit{З.Флюгге Задачи по квантовой механике т.2 М. "Мир" 1974. стр. 296}

\bibitem{misyuchenko}
\textit{В. Ганкин, Ю. Ганкин, О. Куприянова, И. Мисюченко. История электромагнитной массы}
% \\\texttt{http://fphysics.com/d/232484/d/istoriya_em_massy1.pdf}


\bibitem{Haddad}
S. Haddad and S. Suleiman
\textit{NEUTRON CHARGE DISTRIBUTION AND CHARGE DENSITY DISTRIBUTIONS IN LEAD ISOTOPES}
\textit{ACTA PHYSICA POLONICA B, Vol. 30 (1999) No 1}
\\\texttt{http://www.actaphys.uj.edu.pl/fulltext?series=Reg\&vol=30\&page=119}



\end{thebibliography}


\end{document}

